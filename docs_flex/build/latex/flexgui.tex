%% Generated by Sphinx.
\def\sphinxdocclass{report}
\documentclass[letterpaper,10pt,english]{sphinxmanual}
\ifdefined\pdfpxdimen
   \let\sphinxpxdimen\pdfpxdimen\else\newdimen\sphinxpxdimen
\fi \sphinxpxdimen=.75bp\relax
\ifdefined\pdfimageresolution
    \pdfimageresolution= \numexpr \dimexpr1in\relax/\sphinxpxdimen\relax
\fi
%% let collapsible pdf bookmarks panel have high depth per default
\PassOptionsToPackage{bookmarksdepth=5}{hyperref}


\PassOptionsToPackage{warn}{textcomp}
\usepackage[utf8]{inputenc}
\ifdefined\DeclareUnicodeCharacter
% support both utf8 and utf8x syntaxes
  \ifdefined\DeclareUnicodeCharacterAsOptional
    \def\sphinxDUC#1{\DeclareUnicodeCharacter{"#1}}
  \else
    \let\sphinxDUC\DeclareUnicodeCharacter
  \fi
  \sphinxDUC{00A0}{\nobreakspace}
  \sphinxDUC{2500}{\sphinxunichar{2500}}
  \sphinxDUC{2502}{\sphinxunichar{2502}}
  \sphinxDUC{2514}{\sphinxunichar{2514}}
  \sphinxDUC{251C}{\sphinxunichar{251C}}
  \sphinxDUC{2572}{\textbackslash}
\fi
\usepackage{cmap}
\usepackage[T1]{fontenc}
\usepackage{amsmath,amssymb,amstext}
\usepackage{babel}



\usepackage{tgtermes}
\usepackage{tgheros}
\renewcommand{\ttdefault}{txtt}



\usepackage[Bjarne]{fncychap}
\usepackage{sphinx}

\fvset{fontsize=auto}
\usepackage{geometry}


% Include hyperref last.
\usepackage{hyperref}
% Fix anchor placement for figures with captions.
\usepackage{hypcap}% it must be loaded after hyperref.
% Set up styles of URL: it should be placed after hyperref.
\urlstyle{same}

\addto\captionsenglish{\renewcommand{\contentsname}{Contents:}}

\usepackage{sphinxmessages}
\setcounter{tocdepth}{1}



\title{Flex GUI}
\date{Jul 27, 2025}
\release{}
\author{John Thornton}
\newcommand{\sphinxlogo}{\vbox{}}
\renewcommand{\releasename}{}
\makeindex
\begin{document}

\ifdefined\shorthandoff
  \ifnum\catcode`\=\string=\active\shorthandoff{=}\fi
  \ifnum\catcode`\"=\active\shorthandoff{"}\fi
\fi

\pagestyle{empty}
\sphinxmaketitle
\pagestyle{plain}
\sphinxtableofcontents
\pagestyle{normal}
\phantomsection\label{\detokenize{index::doc}}


\sphinxstepscope


\chapter{Flex GUI Description}
\label{\detokenize{description:flex-gui-description}}\label{\detokenize{description::doc}}
\sphinxAtStartPar
Flex GUI is a flexible GUI that can be customized to suit your needs.
\begin{itemize}
\item {} 
\sphinxAtStartPar
Uses stock Qt Designer 5 or 6

\item {} 
\sphinxAtStartPar
Widget names are used to connect controls to the correct code

\item {} 
\sphinxAtStartPar
Widgets are auto\sphinxhyphen{}discovered at startup

\item {} 
\sphinxAtStartPar
Special widgets only need Dynamic Properties to be discovered and created

\item {} 
\sphinxAtStartPar
Your GUI can have exactly the controls and labels you want

\item {} 
\sphinxAtStartPar
You can create and use your own style sheet, changing fonts, colors, etc.

\item {} 
\sphinxAtStartPar
All Flex GUI configuration is done in the .ini file

\item {} 
\sphinxAtStartPar
Flex GUI remembers the size and position of your GUI

\end{itemize}

\sphinxAtStartPar
For more information on Dynamic Properties see {\hyperref[\detokenize{property::doc}]{\sphinxcrossref{\DUrole{doc}{Dynamic Properties}}}}


\section{Features}
\label{\detokenize{description:features}}\begin{itemize}
\item {} 
\sphinxAtStartPar
Status Labels

\item {} 
\sphinxAtStartPar
Menu Actions

\item {} 
\sphinxAtStartPar
Button Actions

\item {} 
\sphinxAtStartPar
Button Controls

\item {} 
\sphinxAtStartPar
Plotter

\item {} 
\sphinxAtStartPar
MDI Input and Buttons

\item {} 
\sphinxAtStartPar
Spindle Controls

\item {} 
\sphinxAtStartPar
Probing with Spindle Safety

\item {} 
\sphinxAtStartPar
Tool Change Controls

\item {} 
\sphinxAtStartPar
Coordinate System Controls

\item {} 
\sphinxAtStartPar
HAL Buttons, Spinboxes, Sliders, Labels and LCDs

\item {} 
\sphinxAtStartPar
HAL I/O Controls

\item {} 
\sphinxAtStartPar
Touch Screen Controls and Popups

\end{itemize}


\section{Acronyms}
\label{\detokenize{description:acronyms}}
\sphinxAtStartPar
The following acronyms will be found in this document
\begin{itemize}
\item {} 
\sphinxAtStartPar
NC = \sphinxtitleref{Numerical Control} code includes G, M and O codes

\item {} 
\sphinxAtStartPar
G = \sphinxtitleref{Geometric} code is used for geometric movements

\item {} 
\sphinxAtStartPar
M = \sphinxtitleref{Miscellaneous} code is used for non\sphinxhyphen{}movement functions

\item {} 
\sphinxAtStartPar
O = \sphinxtitleref{Organizing} code is used in CNC programs to control flow

\end{itemize}

\sphinxstepscope


\chapter{Installing}
\label{\detokenize{install:installing}}\label{\detokenize{install::doc}}

\section{Install with Apt}
\label{\detokenize{install:install-with-apt}}\label{\detokenize{install:install-apt}}
\sphinxAtStartPar
The advantage of using apt to install Flex GUI is when a new version of Flex GUI
is released apt will know a new version is available when you run
\sphinxtitleref{sudo apt update}. This will allow you to install the new version of Flex GUI
along with other Debian software.

\begin{sphinxadmonition}{note}{Note:}
\sphinxAtStartPar
on a brand\sphinxhyphen{}new Rpi5 using the linuxcnc.iso, use apt \sphinxtitleref{\textendash{}fix\sphinxhyphen{}broken
install} to install several qt6 sub\sphinxhyphen{}dependencies
\end{sphinxadmonition}

\sphinxAtStartPar
The first command will ask for your password. Neither command will print
anything in the terminal.

\sphinxAtStartPar
For a PC to create an apt sources file for Flex GUI copy and paste this command
in a terminal

\begin{sphinxVerbatim}[commandchars=\\\{\}]
\PYG{n}{echo} \PYG{l+s+s1}{\PYGZsq{}}\PYG{l+s+s1}{deb [arch=amd64] https://gnipsel.com/flexgui/apt\PYGZhy{}repo stable main}\PYG{l+s+s1}{\PYGZsq{}} \PYG{o}{|} \PYG{n}{sudo} \PYG{n}{tee} \PYG{o}{/}\PYG{n}{etc}\PYG{o}{/}\PYG{n}{apt}\PYG{o}{/}\PYG{n}{sources}\PYG{o}{.}\PYG{n}{list}\PYG{o}{.}\PYG{n}{d}\PYG{o}{/}\PYG{n}{flexgui}\PYG{o}{.}\PYG{n}{list}
\end{sphinxVerbatim}

\sphinxAtStartPar
For a Raspberry Pi create an apt sources file for Flex GUI copy and paste this
command in a terminal

\begin{sphinxVerbatim}[commandchars=\\\{\}]
\PYG{n}{echo} \PYG{l+s+s1}{\PYGZsq{}}\PYG{l+s+s1}{deb [arch=arm64] https://gnipsel.com/flexgui/apt\PYGZhy{}repo stable main}\PYG{l+s+s1}{\PYGZsq{}} \PYG{o}{|} \PYG{n}{sudo} \PYG{n}{tee} \PYG{o}{/}\PYG{n}{etc}\PYG{o}{/}\PYG{n}{apt}\PYG{o}{/}\PYG{n}{sources}\PYG{o}{.}\PYG{n}{list}\PYG{o}{.}\PYG{n}{d}\PYG{o}{/}\PYG{n}{flexgui}\PYG{o}{.}\PYG{n}{list}
\end{sphinxVerbatim}

\sphinxAtStartPar
To check the above command worked you can list the file with this command

\begin{sphinxVerbatim}[commandchars=\\\{\}]
\PYG{n}{ls} \PYG{o}{/}\PYG{n}{etc}\PYG{o}{/}\PYG{n}{apt}\PYG{o}{/}\PYG{n}{sources}\PYG{o}{.}\PYG{n}{list}\PYG{o}{.}\PYG{n}{d}
\end{sphinxVerbatim}

\noindent{\hspace*{\fill}\sphinxincludegraphics{{install-02}.png}\hspace*{\fill}}

\sphinxAtStartPar
Next get the public key for Flex GUI and copy it to trusted.gpg.d

\begin{sphinxVerbatim}[commandchars=\\\{\}]
\PYG{n}{sudo} \PYG{n}{curl} \PYG{o}{\PYGZhy{}}\PYG{o}{\PYGZhy{}}\PYG{n}{silent} \PYG{o}{\PYGZhy{}}\PYG{o}{\PYGZhy{}}\PYG{n}{show}\PYG{o}{\PYGZhy{}}\PYG{n}{error} \PYG{n}{https}\PYG{p}{:}\PYG{o}{/}\PYG{o}{/}\PYG{n}{gnipsel}\PYG{o}{.}\PYG{n}{com}\PYG{o}{/}\PYG{n}{flexgui}\PYG{o}{/}\PYG{n}{apt}\PYG{o}{\PYGZhy{}}\PYG{n}{repo}\PYG{o}{/}\PYG{n}{pgp}\PYG{o}{\PYGZhy{}}\PYG{n}{key}\PYG{o}{.}\PYG{n}{public} \PYG{o}{\PYGZhy{}}\PYG{n}{o} \PYG{o}{/}\PYG{n}{etc}\PYG{o}{/}\PYG{n}{apt}\PYG{o}{/}\PYG{n}{trusted}\PYG{o}{.}\PYG{n}{gpg}\PYG{o}{.}\PYG{n}{d}\PYG{o}{/}\PYG{n}{flexgui}\PYG{o}{.}\PYG{n}{asc}
\end{sphinxVerbatim}

\sphinxAtStartPar
If curl is not installed you can install it with the following command

\begin{sphinxVerbatim}[commandchars=\\\{\}]
\PYG{n}{sudo} \PYG{n}{apt} \PYG{n}{install} \PYG{n}{curl}
\end{sphinxVerbatim}

\sphinxAtStartPar
Next update apt

\begin{sphinxVerbatim}[commandchars=\\\{\}]
\PYG{n}{sudo} \PYG{n}{apt} \PYG{n}{update}
\end{sphinxVerbatim}

\sphinxAtStartPar
If you have Flex GUI installed you can see what packages can be upgraded with
the following command

\begin{sphinxVerbatim}[commandchars=\\\{\}]
\PYG{n}{apt} \PYG{n+nb}{list} \PYG{o}{\PYGZhy{}}\PYG{o}{\PYGZhy{}}\PYG{n}{upgradable}
\end{sphinxVerbatim}

\sphinxAtStartPar
If Flex GUI is not installed you can install it with the following command

\begin{sphinxVerbatim}[commandchars=\\\{\}]
\PYG{n}{sudo} \PYG{n}{apt} \PYG{n}{install} \PYG{n}{flexgui}
\end{sphinxVerbatim}


\section{Install the Deb}
\label{\detokenize{install:install-the-deb}}\label{\detokenize{install:install-deb}}
\sphinxAtStartPar
You can still download the deb from github and install with gdebi if that works
better for you. If you don’t have an internet connection this is the best way to
install Flex GUI

\begin{sphinxadmonition}{note}{Note:}
\sphinxAtStartPar
Between releases the deb will have the latest bug fixes
\end{sphinxadmonition}

\sphinxAtStartPar
\sphinxhref{https://youtu.be/F8mCt7JJDDM}{Installing Flex GUI Tutorial}

\sphinxAtStartPar
Download the latest deb file from
\sphinxhref{https://github.com/jethornton/flexgui/releases}{\textgreater{}HERE\textless{}}.

\sphinxAtStartPar
If the link is not clickable, copy and paste the following URL into your
browser

\begin{sphinxVerbatim}[commandchars=\\\{\}]
\PYG{n}{https}\PYG{p}{:}\PYG{o}{/}\PYG{o}{/}\PYG{n}{github}\PYG{o}{.}\PYG{n}{com}\PYG{o}{/}\PYG{n}{jethornton}\PYG{o}{/}\PYG{n}{flexgui}\PYG{o}{/}\PYG{n}{releases}
\end{sphinxVerbatim}

\sphinxAtStartPar
\textasciitilde{}amd64.deb is for PC’s and \textasciitilde{}arm64.deb is for Raspberry Pi.

\sphinxAtStartPar
Select the latest release and click on the .deb to start a download.

\sphinxAtStartPar
Right click on the deb file and select \sphinxtitleref{Open with GDebi Package Installer}.
If that option is not there then GDebi is not installed, open a terminal and run
this command to install it:

\begin{sphinxVerbatim}[commandchars=\\\{\}]
\PYG{n}{sudo} \PYG{n}{apt} \PYG{n}{install} \PYG{n}{gdebi}
\end{sphinxVerbatim}

\sphinxAtStartPar
An alternative is to install from the terminal outright using \sphinxtitleref{dpkg}. Make sure
the version number is correct for the deb you have the following command may be
an older version.

\begin{sphinxVerbatim}[commandchars=\\\{\}]
\PYG{n}{sudo} \PYG{n}{dpkg} \PYG{o}{\PYGZhy{}}\PYG{n}{i} \PYG{n}{flexgui\PYGZus{}1}\PYG{l+m+mf}{.1}\PYG{l+m+mf}{.0}\PYG{n}{\PYGZus{}amd64}\PYG{o}{.}\PYG{n}{deb}
\end{sphinxVerbatim}


\section{Build and Install}
\label{\detokenize{install:build-and-install}}\label{\detokenize{install:install-build}}
\sphinxAtStartPar
If you plan on changing code in Flex GUI you can clone the repository and build
the deb after making changes. The target directory is optional.

\begin{sphinxVerbatim}[commandchars=\\\{\}]
\PYG{n}{git} \PYG{n}{clone} \PYG{n}{https}\PYG{p}{:}\PYG{o}{/}\PYG{o}{/}\PYG{n}{github}\PYG{o}{.}\PYG{n}{com}\PYG{o}{/}\PYG{n}{jethornton}\PYG{o}{/}\PYG{n}{flexgui}\PYG{o}{.}\PYG{n}{git} \PYG{p}{(}\PYG{n}{target}\PYG{o}{/}\PYG{n}{directory}\PYG{p}{)}
\end{sphinxVerbatim}

\sphinxAtStartPar
Before building the deb you will need to install some programs that do the
building. Open a terminal and run the following to install devscripts

\begin{sphinxVerbatim}[commandchars=\\\{\}]
\PYG{n}{sudo} \PYG{n}{apt} \PYG{n}{install} \PYG{n}{devscripts}
\end{sphinxVerbatim}

\sphinxAtStartPar
Open a terminal in the top most flexgui directory and use this command to build
a deb file.

\begin{sphinxVerbatim}[commandchars=\\\{\}]
\PYG{n}{debuild} \PYG{o}{\PYGZhy{}}\PYG{n}{us} \PYG{o}{\PYGZhy{}}\PYG{n}{uc}
\end{sphinxVerbatim}


\section{Copy Example Files}
\label{\detokenize{install:copy-example-files}}\label{\detokenize{install:copy-examples}}
\sphinxAtStartPar
After installing Flex GUI, a menu item \sphinxtitleref{Copy Flex Examples} is added to the
\sphinxtitleref{CNC} menu. This will copy the Flex GUI example files to
\textasciitilde{}/linuxcnc/configs/\sphinxtitleref{flex\_examples}.

\begin{sphinxadmonition}{note}{Note:}
\sphinxAtStartPar
After updating the Flex GUI some examples may have changed. To get a
fresh copy of the examples delete the \sphinxtitleref{linuxcnc/configs/flex\_examples} or
rename it.
\end{sphinxadmonition}

\sphinxstepscope


\chapter{Qt Designer}
\label{\detokenize{designer:qt-designer}}\label{\detokenize{designer::doc}}

\section{Installing the Qt Designer}
\label{\detokenize{designer:installing-the-qt-designer}}\label{\detokenize{designer:install-designer}}
\sphinxAtStartPar
In a terminal, install Qt Designer 5 with

\begin{sphinxVerbatim}[commandchars=\\\{\}]
\PYG{n}{sudo} \PYG{n}{apt} \PYG{n}{install} \PYG{n}{qttools5}\PYG{o}{\PYGZhy{}}\PYG{n}{dev}\PYG{o}{\PYGZhy{}}\PYG{n}{tools}
\end{sphinxVerbatim}

\begin{sphinxadmonition}{note}{Note:}
\sphinxAtStartPar
The Qt6 Designer is not required nor better; Qt5 Designer is fine.
\end{sphinxadmonition}


\section{Building a GUI}
\label{\detokenize{designer:building-a-gui}}\label{\detokenize{designer:build-gui}}
\sphinxAtStartPar
Run the Qt Designer from the Applications \textgreater{} Programming menu and create a new
\sphinxtitleref{Main Window}

\noindent{\hspace*{\fill}\sphinxincludegraphics{{designer-01}.png}\hspace*{\fill}}

\sphinxAtStartPar
To add a Tool Bar, right click on the main window and select \sphinxtitleref{Add Tool Bar}

\noindent{\hspace*{\fill}\sphinxincludegraphics{{designer-02}.png}\hspace*{\fill}}

\sphinxAtStartPar
To add a Menu, type in the menu area and press enter

\noindent{\hspace*{\fill}\sphinxincludegraphics{{designer-03}.png}\hspace*{\fill}}

\sphinxAtStartPar
When you create a Menu item it creates an action; this action can be dragged to
the Tool Bar to create a tool bar button

\noindent{\hspace*{\fill}\sphinxincludegraphics{{designer-04}.png}\hspace*{\fill}}

\sphinxAtStartPar
Adding items from the Widget Box is drag\sphinxhyphen{}and\sphinxhyphen{}drop. To create a basic layout from
Containers, add two Frames and a Tab Widget

\noindent{\hspace*{\fill}\sphinxincludegraphics{{designer-05}.png}\hspace*{\fill}}

\sphinxAtStartPar
Right click in the QMainWindow and select Lay out \textendash{}\textgreater{} Lay out Vertically

\noindent{\hspace*{\fill}\sphinxincludegraphics{{designer-06}.png}\hspace*{\fill}}

\noindent{\hspace*{\fill}\sphinxincludegraphics{{designer-07}.png}\hspace*{\fill}}

\sphinxAtStartPar
Add a Push Button to the QFrame, then right click on the frame or the QFrame in
the Object Inspector and set the lay out to \sphinxtitleref{grid}

\noindent{\hspace*{\fill}\sphinxincludegraphics{{designer-08}.png}\hspace*{\fill}}

\noindent{\hspace*{\fill}\sphinxincludegraphics{{designer-09}.png}\hspace*{\fill}}

\sphinxAtStartPar
After dragging a widget into the window, make sure you use the correct
objectName for that widget. For example the E\sphinxhyphen{}Stop button is called estop\_pb.

\begin{sphinxadmonition}{note}{Note:}
\sphinxAtStartPar
Each object name must be unique; designer will not allow duplicate
names.
\end{sphinxadmonition}

\sphinxAtStartPar
Save the GUI in the configuration directory alongside your .ini file.

\sphinxAtStartPar
You can start Qt5 Designer from a terminal with \sphinxtitleref{designer \&} which spawns a new
process (gives you back the terminal prompt.)

\begin{sphinxadmonition}{note}{Note:}
\sphinxAtStartPar
There is an documented issue with Qt5 Designer and bold fonts not
appearing properly.
\end{sphinxadmonition}


\section{Qt6 Designer}
\label{\detokenize{designer:qt6-designer}}
\sphinxAtStartPar
Qt6 Designer can be installed from a terminal with

\begin{sphinxVerbatim}[commandchars=\\\{\}]
\PYG{n}{sudo} \PYG{n}{apt} \PYG{n}{install} \PYG{n}{designer}\PYG{o}{\PYGZhy{}}\PYG{n}{qt6}
\end{sphinxVerbatim}

\sphinxAtStartPar
To run Qt6 Designer you must use the full path to the executable

\begin{sphinxVerbatim}[commandchars=\\\{\}]
\PYG{o}{/}\PYG{n}{usr}\PYG{o}{/}\PYG{n}{lib}\PYG{o}{/}\PYG{n}{qt6}\PYG{o}{/}\PYG{n+nb}{bin}\PYG{o}{/}\PYG{n}{designer}
\end{sphinxVerbatim}

\sphinxstepscope


\chapter{Building a GUI}
\label{\detokenize{build:building-a-gui}}\label{\detokenize{build::doc}}
\sphinxAtStartPar
If you have not copied the examples from the CNC menu select Copy Flex Examples.
This will put the Flex examples in linuxcnc/configs/flex\_examples.

\sphinxAtStartPar
The starters have all the files needed to run a simulation without complicated
code. The starters have a very simple GUI to start, just enough to show you that
they work.


\section{Copy a Starter}
\label{\detokenize{build:copy-a-starter}}
\sphinxAtStartPar
From the linuxcnc/configs/flex\_examples/starters copy one of the starter types
to the linuxcnc/configs directory.
\begin{itemize}
\item {} 
\sphinxAtStartPar
Rename the directory to the name of your choice.

\item {} 
\sphinxAtStartPar
Rename the .ui and .ini files to the name of your choice.

\item {} 
\sphinxAtStartPar
Edit the .ini file and change the GUI to the name of your .ui file.

\item {} 
\sphinxAtStartPar
Edit the MIN\_LIMIT and MAX\_LIMIT for each axis and joint to match your machine

\item {} 
\sphinxAtStartPar
From the CNC menu select LinuxCNC and pick your configuration, check Create
Desktop Shortcut then clidk OK to run your configuration.

\end{itemize}

\begin{figure}[htbp]
\centering
\capstart

\noindent\sphinxincludegraphics{{build-01}.png}
\caption{Mill Starter Example}\label{\detokenize{build:id1}}\end{figure}

\sphinxstepscope


\chapter{Dynamic Properties}
\label{\detokenize{property:dynamic-properties}}\label{\detokenize{property::doc}}
\sphinxAtStartPar
Flex GUI uses a lot Dynamic Properties to customize widget behavior and add
more functionality to a widget.

\sphinxAtStartPar
To Create a Dynamic Property the first thing you do is select the widget.

\noindent{\hspace*{\fill}\sphinxincludegraphics{{property-01}.png}\hspace*{\fill}}

\sphinxAtStartPar
Next left click on the green plus sign in the Property Editor.

\noindent{\hspace*{\fill}\sphinxincludegraphics{{property-02}.png}\hspace*{\fill}}

\sphinxAtStartPar
Select a string type as Flex GUI always uses a string type Dynamic Property.

\noindent{\hspace*{\fill}\sphinxincludegraphics{{property-03}.png}\hspace*{\fill}}

\sphinxAtStartPar
Next enter the Dynamic Property Name, which must be exactly like shown in the
documents.

\noindent{\hspace*{\fill}\sphinxincludegraphics{{property-04}.png}\hspace*{\fill}}

\sphinxAtStartPar
Press OK then the Dynamic Property will show up in the Property Editor.

\noindent{\hspace*{\fill}\sphinxincludegraphics{{property-05}.png}\hspace*{\fill}}

\sphinxAtStartPar
Now you can enter the Value for that Dynamic Property.

\noindent{\hspace*{\fill}\sphinxincludegraphics{{property-06}.png}\hspace*{\fill}}

\begin{sphinxadmonition}{note}{Note:}
\sphinxAtStartPar
The Dynamic Properties section does not show up in the Property
Editor until you create a Dynamic Property.
\end{sphinxadmonition}

\sphinxstepscope


\chapter{INI Settings}
\label{\detokenize{ini:ini-settings}}\label{\detokenize{ini::doc}}
\sphinxAtStartPar
\sphinxhref{https://youtu.be/JQXG9I7fwSo}{Video Tutorial}

\begin{sphinxadmonition}{note}{Note:}
\sphinxAtStartPar
The following Flex GUI settings are all located in the {[}DISPLAY{]}
section of your LinuxCNC .ini file.
\end{sphinxadmonition}


\section{Using Flex GUI}
\label{\detokenize{ini:using-flex-gui}}\label{\detokenize{ini:using-flexgui}}
\sphinxAtStartPar
To use the Flex GUI (as opposed to Axis or others), change the DISPLAY value to

\begin{sphinxVerbatim}[commandchars=\\\{\}]
\PYG{n}{DISPLAY} \PYG{o}{=} \PYG{n}{flexgui}
\end{sphinxVerbatim}

\sphinxAtStartPar
If no GUI is specified then the default GUI will be used.

\begin{sphinxadmonition}{note}{Note:}
\sphinxAtStartPar
Any Flex GUI .ui and .qss files must be in the same LinuxCNC
configuration directory as the .ini file.
\end{sphinxadmonition}

\sphinxAtStartPar
To use your .ui file (created with Qt Designer), add a GUI key to the .ini
with its \sphinxtitleref{filename}:

\begin{sphinxVerbatim}[commandchars=\\\{\}]
\PYG{n}{GUI} \PYG{o}{=} \PYG{n}{my}\PYG{o}{\PYGZhy{}}\PYG{n}{file}\PYG{o}{\PYGZhy{}}\PYG{n}{name}\PYG{o}{.}\PYG{n}{ui}
\end{sphinxVerbatim}


\section{Themes}
\label{\detokenize{ini:themes}}\label{\detokenize{ini:installed-themes}}
\sphinxAtStartPar
Themes are just style sheets that get applied to the widgets. The theme files
are in the themes directory of the example files if you want to copy and
customize one of the themes.

\begin{sphinxVerbatim}[commandchars=\\\{\}]
\PYG{n}{blue}\PYG{o}{.}\PYG{n}{qss}
\PYG{n}{blue}\PYG{o}{\PYGZhy{}}\PYG{n}{touch}\PYG{o}{.}\PYG{n}{qss}
\PYG{n}{dark}\PYG{o}{.}\PYG{n}{qss}
\PYG{n}{dark}\PYG{o}{\PYGZhy{}}\PYG{n}{touch}\PYG{o}{.}\PYG{n}{qss}
\PYG{n}{keyboard}\PYG{o}{.}\PYG{n}{qss}
\PYG{n}{touch}\PYG{o}{.}\PYG{n}{qss}
\end{sphinxVerbatim}

\sphinxAtStartPar
To use a built\sphinxhyphen{}in theme with no color changes choose one of the
following

\begin{sphinxVerbatim}[commandchars=\\\{\}]
\PYG{n}{THEME} \PYG{o}{=} \PYG{n}{touch}
\PYG{n}{THEME} \PYG{o}{=} \PYG{n}{keyboard}
\end{sphinxVerbatim}

\sphinxAtStartPar
To use a built in theme with coloring choose one of the following

\begin{sphinxVerbatim}[commandchars=\\\{\}]
\PYG{n}{THEME} \PYG{o}{=} \PYG{n}{blue}
\PYG{n}{THEME} \PYG{o}{=} \PYG{n}{blue}\PYG{o}{\PYGZhy{}}\PYG{n}{touch}
\PYG{n}{THEME} \PYG{o}{=} \PYG{n}{dark}
\PYG{n}{THEME} \PYG{o}{=} \PYG{n}{dark}\PYG{o}{\PYGZhy{}}\PYG{n}{touch}
\end{sphinxVerbatim}

\begin{sphinxadmonition}{note}{Note:}
\sphinxAtStartPar
Touch themes use tabs set to South for rounding and non touch use tabs
set to North.
\end{sphinxadmonition}

\begin{sphinxadmonition}{note}{Note:}
\sphinxAtStartPar
THEME is checked first then QSS so the first entry found is used.
\end{sphinxadmonition}

\sphinxAtStartPar
To use a custom .qss style sheet you created named \sphinxtitleref{lightflex.qss}

\begin{sphinxVerbatim}[commandchars=\\\{\}]
\PYG{n}{QSS} \PYG{o}{=} \PYG{n}{lightflex}\PYG{o}{.}\PYG{n}{qss}
\end{sphinxVerbatim}

\sphinxAtStartPar
For more information on style sheets see {\hyperref[\detokenize{style::doc}]{\sphinxcrossref{\DUrole{doc}{StyleSheet}}}}


\section{Jog Increments}
\label{\detokenize{ini:jog-increments}}
\sphinxAtStartPar
The following settings can be used in the {[}DISPLAY{]} section of the ini file to
preset jog items

\begin{sphinxVerbatim}[commandchars=\\\{\}]
\PYG{n}{INCREMENTS} \PYG{o}{=} \PYG{l+m+mf}{0.100}\PYG{p}{,} \PYG{l+m+mf}{0.010}\PYG{p}{,} \PYG{l+m+mf}{0.001}
\PYG{o+ow}{or}
\PYG{n}{INCREMENTS} \PYG{o}{=} \PYG{l+m+mi}{1} \PYG{n}{inch}\PYG{p}{,} \PYG{l+m+mf}{0.5} \PYG{o+ow}{in}\PYG{p}{,} \PYG{l+m+mi}{1} \PYG{n}{cm}\PYG{p}{,} \PYG{l+m+mi}{1} \PYG{n}{mm}
\PYG{n}{MIN\PYGZus{}LINEAR\PYGZus{}VELOCITY} \PYG{o}{=} \PYG{l+m+mf}{0.1}
\PYG{n}{MAX\PYGZus{}LINEAR\PYGZus{}VELOCITY} \PYG{o}{=} \PYG{l+m+mf}{1.0}
\PYG{n}{DEFAULT\PYGZus{}LINEAR\PYGZus{}VELOCITY} \PYG{o}{=} \PYG{l+m+mf}{0.2}
\end{sphinxVerbatim}

\begin{sphinxadmonition}{warning}{Warning:}
\sphinxAtStartPar
{[}DISPLAY{]} INCREMENTS must be a comma seperated list or it will be
ignored.
\end{sphinxadmonition}

\begin{sphinxadmonition}{note}{Note:}
\sphinxAtStartPar
Jog incremnts can have unit lables, the following are valid unit
labels cm, mm, um, inch, in or mil. If no unit labels are found the the
configuration units are used.
\end{sphinxadmonition}


\section{Startup File}
\label{\detokenize{ini:startup-file}}
\sphinxAtStartPar
To automatically open a NC file on startup, add the OPEN\_FILE key with any
valid path. Use \textasciitilde{}/ as a shortcut to the users home directory. Use ./ to indicate
that the file is in the configuration directory

\begin{sphinxVerbatim}[commandchars=\\\{\}]
\PYG{n}{Full} \PYG{n}{Path} \PYG{n}{to} \PYG{n}{the} \PYG{n}{file}
\PYG{n}{OPEN\PYGZus{}FILE} \PYG{o}{=} \PYG{o}{/}\PYG{n}{home}\PYG{o}{/}\PYG{n}{john}\PYG{o}{/}\PYG{n}{linuxcnc}\PYG{o}{/}\PYG{n}{configs}\PYG{o}{/}\PYG{n}{myconfig}\PYG{o}{/}\PYG{n}{welcome}\PYG{o}{.}\PYG{n}{ngc}
\PYG{o+ow}{or} \PYG{n}{use} \PYG{n}{the} \PYG{o}{\PYGZti{}} \PYG{k}{for} \PYG{n}{the} \PYG{n}{users} \PYG{n}{home} \PYG{n}{directory}
\PYG{n}{OPEN\PYGZus{}FILE} \PYG{o}{=} \PYG{o}{\PYGZti{}}\PYG{o}{/}\PYG{n}{linuxcnc}\PYG{o}{/}\PYG{n}{configs}\PYG{o}{/}\PYG{n}{flex\PYGZus{}examples}\PYG{o}{/}\PYG{n}{probe\PYGZus{}sim}\PYG{o}{/}\PYG{n}{square}\PYG{o}{.}\PYG{n}{ngc}
\PYG{o+ow}{or} \PYG{n}{use} \PYG{n}{the} \PYG{o}{.}\PYG{o}{/} \PYG{n}{to} \PYG{n}{use} \PYG{n}{the} \PYG{n}{current} \PYG{n}{configuration} \PYG{n}{directory}
\PYG{n}{OPEN\PYGZus{}FILE} \PYG{o}{=} \PYG{o}{.}\PYG{o}{/}\PYG{n}{welcome}\PYG{o}{.}\PYG{n}{ngc}
\PYG{o+ow}{or} \PYG{n}{use} \PYG{n}{the} \PYG{o}{.}\PYG{o}{.}\PYG{o}{/} \PYG{n}{to} \PYG{n}{use} \PYG{n}{the} \PYG{n}{parent} \PYG{n}{directory} \PYG{n}{of} \PYG{n}{the} \PYG{n}{configuration}
\PYG{n}{OPEN\PYGZus{}FILE} \PYG{o}{=} \PYG{o}{.}\PYG{o}{.}\PYG{o}{/}\PYG{n}{welcome}\PYG{o}{.}\PYG{n}{ngc}
\end{sphinxVerbatim}


\section{File Location}
\label{\detokenize{ini:file-location}}
\sphinxAtStartPar
Likewise, to specify a default location for NC files, add the PROGRAM\_PREFIX
item.

\begin{sphinxVerbatim}[commandchars=\\\{\}]
\PYG{n}{PROGRAM\PYGZus{}PREFIX} \PYG{o}{=} \PYG{o}{/}\PYG{n}{home}\PYG{o}{/}\PYG{n}{john}\PYG{o}{/}\PYG{n}{linuxcnc}\PYG{o}{/}\PYG{n}{configs}\PYG{o}{/}\PYG{n}{myconfig}
\PYG{o+ow}{or}
\PYG{n}{PROGRAM\PYGZus{}PREFIX} \PYG{o}{=} \PYG{o}{\PYGZti{}}\PYG{o}{/}\PYG{n}{linuxcnc}\PYG{o}{/}\PYG{n}{configs}\PYG{o}{/}\PYG{n}{flex\PYGZus{}examples}\PYG{o}{/}\PYG{n}{probe\PYGZus{}sim}
\PYG{o+ow}{or}
\PYG{n}{PROGRAM\PYGZus{}PREFIX} \PYG{o}{=} \PYG{o}{.}\PYG{o}{/}
\PYG{o+ow}{or}
\PYG{n}{PROGRAM\PYGZus{}PREFIX} \PYG{o}{=} \PYG{o}{.}\PYG{o}{.}\PYG{o}{/}
\end{sphinxVerbatim}


\section{Tool Table Editor}
\label{\detokenize{ini:tool-table-editor}}
\sphinxAtStartPar
To specify a different tool table editor add an entry to the {[}DISPLAY{]} section.
If no entry is found then the default tool editor is used

\begin{sphinxVerbatim}[commandchars=\\\{\}]
\PYG{n}{TOOL\PYGZus{}EDITOR} \PYG{o}{=} \PYG{n}{tooledit}
\end{sphinxVerbatim}

\sphinxAtStartPar
To control the columns displayed by the default tool editor add any of the valid
column specifiers separated by a space.

\begin{sphinxVerbatim}[commandchars=\\\{\}]
\PYG{n}{TOOL\PYGZus{}EDITOR} \PYG{o}{=} \PYG{n}{tooledit} \PYG{n}{x} \PYG{n}{y} \PYG{n}{z} \PYG{n}{a} \PYG{n}{b} \PYG{n}{c} \PYG{n}{u} \PYG{n}{v} \PYG{n}{w} \PYG{n}{diam} \PYG{n}{front} \PYG{n}{back} \PYG{n}{orien}
\end{sphinxVerbatim}

\sphinxAtStartPar
If no entry is found then the axes in the configuration and diameter are shown.
Tool, Pocket and Comment are always shown.


\section{Resource File}
\label{\detokenize{ini:resource-file}}
\sphinxAtStartPar
To use a .py resource file (to add images to buttons with your qss stylesheet)
place the .py resource file in the configuration directory and add the
following line to the .ini file

\begin{sphinxVerbatim}[commandchars=\\\{\}]
\PYG{n}{RESOURCES} \PYG{o}{=} \PYG{n}{resources}\PYG{o}{.}\PYG{n}{py}
\end{sphinxVerbatim}

\sphinxAtStartPar
See the section on Resources for more info.


\section{File Extensions}
\label{\detokenize{ini:file-extensions}}
\sphinxAtStartPar
The keyboard file dialog defaults to \sphinxtitleref{*.ngc} and this ignores case. To
specify the file extensions you want the file dialog to show, add an
EXTENSIONS key with the desired extensions separated by a comma. The
extensions must be in the format \sphinxtitleref{*.ext} with the asterisk and dot

\begin{sphinxVerbatim}[commandchars=\\\{\}]
\PYG{n}{EXTENSIONS} \PYG{o}{=} \PYG{o}{*}\PYG{o}{.}\PYG{n}{nc}\PYG{p}{,} \PYG{o}{*}\PYG{o}{.}\PYG{n}{G}\PYG{o}{\PYGZhy{}}\PYG{n}{code}\PYG{p}{,} \PYG{o}{*}\PYG{o}{.}\PYG{n}{ngc}\PYG{p}{,} \PYG{o}{*}\PYG{o}{.}\PYG{n}{txt}
\end{sphinxVerbatim}


\section{Screen Size}
\label{\detokenize{ini:screen-size}}
\sphinxAtStartPar
To control the initial size of the screen, add either:

\begin{sphinxVerbatim}[commandchars=\\\{\}]
\PYG{n}{SIZE} \PYG{o}{=} \PYG{n}{minimized}
\PYG{n}{SIZE} \PYG{o}{=} \PYG{n}{normal}
\PYG{n}{SIZE} \PYG{o}{=} \PYG{n}{maximized}
\PYG{n}{SIZE} \PYG{o}{=} \PYG{n}{full}
\end{sphinxVerbatim}

\begin{sphinxadmonition}{warning}{Warning:}
\sphinxAtStartPar
Full size screen does not have any window controls. Make sure
there is a way to close the GUI like an Exit button or you may not be able to
close the application. As a last\sphinxhyphen{}resort, pressing ALT\sphinxhyphen{}F4 will close it.
\end{sphinxadmonition}


\section{Plotter}
\label{\detokenize{ini:plotter}}
\sphinxAtStartPar
The plotter background color can be set in the {[}FLEXGUI{]} section of the ini. The
value is the Red,Greed,Blue color numbers from 0 to 1 with no space. So an entry
of 0.0,0.0,0.0 is black and 1.0,1.0,1.0 is white. Use a RGB 0\sphinxhyphen{}1 Color Picker to
select the RGB values.

\begin{sphinxVerbatim}[commandchars=\\\{\}]
\PYG{p}{[}\PYG{n}{FLEXGUI}\PYG{p}{]}
\PYG{n}{PLOT\PYGZus{}BACKGROUND\PYGZus{}COLOR} \PYG{o}{=} \PYG{l+m+mf}{0.0}\PYG{p}{,}\PYG{l+m+mf}{0.0}\PYG{p}{,}\PYG{l+m+mf}{0.0}
\end{sphinxVerbatim}

\sphinxAtStartPar
The plotter orientation can be set to one of the following x, x2, y, y2, z, or p.

\begin{sphinxVerbatim}[commandchars=\\\{\}]
\PYG{p}{[}\PYG{n}{DISPLAY}\PYG{p}{]}
\PYG{n}{VIEW} \PYG{o}{=} \PYG{n}{x}
\end{sphinxVerbatim}


\section{Colors}
\label{\detokenize{ini:colors}}
\sphinxAtStartPar
The E\sphinxhyphen{}Stop can have a static color for Open and Closed.

\sphinxAtStartPar
The Power Button can have a static color for Off and On.

\sphinxAtStartPar
Create a key in the ini file called FLEXGUI and use the following to
control the static color of these items. The value can be any valid color
specification in the RGB, RGBA or Hex color format.

\begin{sphinxVerbatim}[commandchars=\\\{\}]
\PYG{p}{[}\PYG{n}{FLEXGUI}\PYG{p}{]}
\PYG{n}{ESTOP\PYGZus{}OPEN\PYGZus{}COLOR} \PYG{o}{=} \PYG{l+m+mi}{128}\PYG{p}{,} \PYG{l+m+mi}{255}\PYG{p}{,} \PYG{l+m+mi}{128}
\PYG{n}{ESTOP\PYGZus{}CLOSED\PYGZus{}COLOR} \PYG{o}{=} \PYG{l+m+mi}{255}\PYG{p}{,} \PYG{l+m+mi}{77}\PYG{p}{,} \PYG{l+m+mi}{77}
\PYG{n}{POWER\PYGZus{}OFF\PYGZus{}COLOR} \PYG{o}{=} \PYG{l+m+mi}{255}\PYG{p}{,} \PYG{l+m+mi}{128}\PYG{p}{,} \PYG{l+m+mi}{128}
\PYG{n}{POWER\PYGZus{}ON\PYGZus{}COLOR} \PYG{o}{=} \PYG{c+c1}{\PYGZsh{}00FF00}
\PYG{n}{PROBE\PYGZus{}ENABLE\PYGZus{}ON\PYGZus{}COLOR} \PYG{o}{=} \PYG{l+m+mi}{255}\PYG{p}{,} \PYG{l+m+mi}{0}\PYG{p}{,} \PYG{l+m+mi}{0}\PYG{p}{,} \PYG{l+m+mi}{255}
\PYG{n}{PROBE\PYGZus{}ENABLE\PYGZus{}OFF\PYGZus{}COLOR} \PYG{o}{=} \PYG{l+m+mi}{0}\PYG{p}{,} \PYG{l+m+mi}{125}\PYG{p}{,} \PYG{l+m+mi}{0}\PYG{p}{,} \PYG{l+m+mi}{125}
\end{sphinxVerbatim}

\begin{sphinxadmonition}{note}{Note:}
\sphinxAtStartPar
Color pairs need to have both colors specified or the color will only
toggle once.
\end{sphinxadmonition}

\sphinxAtStartPar
Another way to achieve this is is via adding and editing a .qss stylesheet
file. See the {\hyperref[\detokenize{style::doc}]{\sphinxcrossref{\DUrole{doc}{StyleSheet}}}} section for more info.


\section{LED Defaults}
\label{\detokenize{ini:led-defaults}}\label{\detokenize{ini:id1}}
\sphinxAtStartPar
LED buttons can have defaults set in the ini file. This makes it easier to have
consistent LED size, position and colors. These options go in the {[}FLEXGUI{]}
section.

\sphinxAtStartPar
The color options can be specified using HEX, RGB or RGBA.

\sphinxAtStartPar
Valid RGB(A) Red, Green, Blue (Alpha) values are 0 to 255.

\sphinxAtStartPar
Valid HEX values are \#000000 to \#ffffff

\sphinxAtStartPar
In PyQt6 the Alpha channel is 0 to 255. 0 represents a fully transparent color,
while 255 represents a fully opaque color. If Alpha is ommitted then it’s set to
fully opaque or 255.

\sphinxAtStartPar
The Diameter and Offset values are whole numbers only.

\begin{sphinxVerbatim}[commandchars=\\\{\}]
\PYG{p}{[}\PYG{n}{FLEXGUI}\PYG{p}{]}
\PYG{n}{LED\PYGZus{}DIAMETER} \PYG{o}{=} \PYG{l+m+mi}{15}
\PYG{n}{LED\PYGZus{}RIGHT\PYGZus{}OFFSET} \PYG{o}{=} \PYG{l+m+mi}{5}
\PYG{n}{LED\PYGZus{}TOP\PYGZus{}OFFSET} \PYG{o}{=} \PYG{l+m+mi}{5}
\PYG{n}{LED\PYGZus{}ON\PYGZus{}COLOR} \PYG{o}{=} \PYG{l+m+mi}{0}\PYG{p}{,} \PYG{l+m+mi}{255}\PYG{p}{,} \PYG{l+m+mi}{0}
\PYG{n}{LED\PYGZus{}OFF\PYGZus{}COLOR}\PYG{o}{=} \PYG{l+m+mi}{125}\PYG{p}{,} \PYG{l+m+mi}{0}\PYG{p}{,} \PYG{l+m+mi}{0}\PYG{p}{,} \PYG{l+m+mi}{255}
\end{sphinxVerbatim}

\sphinxAtStartPar
For more information on LED buttons see {\hyperref[\detokenize{controls:led-buttons}]{\sphinxcrossref{\DUrole{std,std-ref}{LED Buttons}}}}


\section{Touch Screens}
\label{\detokenize{ini:touch-screens}}
\sphinxAtStartPar
Options for touch screen users.

\sphinxAtStartPar
Set the touch screen file chooser to automatically adjust the width by adding the
following to the FLEXGUI section.

\begin{sphinxVerbatim}[commandchars=\\\{\}]
\PYG{p}{[}\PYG{n}{FLEXGUI}\PYG{p}{]}
\PYG{n}{TOUCH\PYGZus{}FILE\PYGZus{}WIDTH} \PYG{o}{=} \PYG{k+kc}{True}
\end{sphinxVerbatim}

\sphinxstepscope


\chapter{Status Labels}
\label{\detokenize{labels:status-labels}}\label{\detokenize{labels::doc}}
\sphinxAtStartPar
\sphinxhref{https://youtu.be/wwT9fDTUa0c}{Status Labels Tutorial}

\sphinxAtStartPar
Status labels are created using a QLabel and setting the \sphinxtitleref{Object Name}. Status
labels come in two forms. A single\sphinxhyphen{}status\sphinxhyphen{}label like \sphinxtitleref{Machine Status} only
contains one piece of information, such as \sphinxtitleref{OFF}, \sphinxtitleref{RUN}, etc.

\sphinxAtStartPar
A multiple\sphinxhyphen{}status\sphinxhyphen{}label like the \sphinxtitleref{axis} or \sphinxtitleref{joint} dictionaries have multiple
items and displays for all joints. Multiple\sphinxhyphen{}status\sphinxhyphen{}labels use a number
identifier to select the axis, joint, or spindle information wanted.

\sphinxAtStartPar
When creating a status label, set the \sphinxtitleref{objectName} to the status you want to
display

\noindent{\hspace*{\fill}\sphinxincludegraphics{{status-01}.png}\hspace*{\fill}}


\section{Precision}
\label{\detokenize{labels:precision}}
\sphinxAtStartPar
Labels that return float values default to 3 decimal places for metric and 4
for inch.

\sphinxAtStartPar
To override the default, select the label then click on the Green Plus sign
in the Property Editor to add a Dynamic Property and select String.
See {\hyperref[\detokenize{property::doc}]{\sphinxcrossref{\DUrole{doc}{Dynamic Properties}}}}

\noindent{\hspace*{\fill}\sphinxincludegraphics{{status-02}.png}\hspace*{\fill}}

\sphinxAtStartPar
Set the Property Name to \sphinxtitleref{precision}:

\noindent{\hspace*{\fill}\sphinxincludegraphics{{status-03}.png}\hspace*{\fill}}

\sphinxAtStartPar
Set the Value to how many decimal places you want for that status label

\noindent{\hspace*{\fill}\sphinxincludegraphics{{status-04}.png}\hspace*{\fill}}

\sphinxAtStartPar
For more information about status labels read the LinuxCNC \sphinxhref{http://linuxcnc.org/docs/stable/html/config/python-interface.html}{Python Interface
Status Attributes}


\section{Status Labels}
\label{\detokenize{labels:id1}}
\sphinxAtStartPar
Status Labels are created by adding a QLabel and changing the Object Name to one
of the following. Some status labels use a dictionary to look up the name instead
of displaying the integer the name is displayed.

\begin{sphinxVerbatim}[commandchars=\\\{\}]
acceleration\PYGZus{}lb \PYGZhy{} returns float
reflects the INI entry [TRAJ]DEFAULT\PYGZus{}LINEAR\PYGZus{}ACCELERATION, if that entry is not
found it returns 1e+99

active\PYGZus{}queue\PYGZus{}lb \PYGZhy{} returns integer
number of motions blending

actual\PYGZus{}position\PYGZus{}lb \PYGZhy{} returns tuple of floats
current trajectory position, (x y z a b c u v w) in machine units
\end{sphinxVerbatim}

\sphinxAtStartPar
See the {\hyperref[\detokenize{labels:axis-position-labels}]{\emph{Axis Position}}} labels for individual axis
positions.

\begin{sphinxVerbatim}[commandchars=\\\{\}]
adaptive\PYGZus{}feed\PYGZus{}enabled\PYGZus{}lb \PYGZhy{} returns boolean
status of adaptive feedrate override

ain\PYGZus{}lb \PYGZhy{} returns tuple of floats
current value of the analog input pins

angular\PYGZus{}units\PYGZus{}lb \PYGZhy{} returns float \PYGZhy{} precision can be set
machine angular units per deg, reflects [TRAJ]ANGULAR\PYGZus{}UNITS

aout\PYGZus{}lb \PYGZhy{} returns tuple of floats
current value of the analog output pins

axis\PYGZus{}lb \PYGZhy{} returns tuple of dicts
reflecting current axis values

axis\PYGZus{}mask\PYGZus{}lb \PYGZhy{} returns integer
sum of the axes by [TRAJ]COORDINATES X=1, Y=2, Z=4, A=8, B=16, C=32, U=64, V=128, W=256

block\PYGZus{}delete\PYGZus{}lb \PYGZhy{} returns boolean
block delete current status

call\PYGZus{}level\PYGZus{}lb \PYGZhy{} returns integer
current subroutine depth. \PYGZhy{} 0 If not in a subroutine

command\PYGZus{}lb \PYGZhy{} returns string
currently executing command

current\PYGZus{}line\PYGZus{}lb \PYGZhy{} returns integer
currently executing line

current\PYGZus{}vel\PYGZus{}lb \PYGZhy{} returns float \PYGZhy{} precision can be set
current velocity in user units per second

cycle\PYGZus{}time\PYGZus{}lb \PYGZhy{} returns float \PYGZhy{} precision can be set
thread period

debug\PYGZus{}lb \PYGZhy{} returns integer
debug flag from the INI file

delay\PYGZus{}left\PYGZus{}lb \PYGZhy{} returns float \PYGZhy{} precision can be set
remaining time on G4 dwell command in seconds

din\PYGZus{}lb \PYGZhy{} returns tuple of integers
current value of the digital input pins

distance\PYGZus{}to\PYGZus{}go\PYGZus{}lb \PYGZhy{} returns float \PYGZhy{} precision can be set
remaining distance of current move as reported by trajectory planner

echo\PYGZus{}serial\PYGZus{}number\PYGZus{}lb \PYGZhy{} returns integer
The serial number of the last completed command sent by a UI to task.
All commands carry a serial number. Once the command has been executed,
its serial number is reflected in echo\PYGZus{}serial\PYGZus{}number

enabled\PYGZus{}lb \PYGZhy{} returns boolean
trajectory planner enabled flag

estop\PYGZus{}lb \PYGZhy{} returns integer
Returns either STATE\PYGZus{}ESTOP = 1) or not = 0)

exec\PYGZus{}state\PYGZus{}lb \PYGZhy{} returns integer that is used to lookup the state name.
task execution state. One of EXEC\PYGZus{}ERROR = 1, EXEC\PYGZus{}DONE = 2,
EXEC\PYGZus{}WAITING\PYGZus{}FOR\PYGZus{}MOTION = 3, EXEC\PYGZus{}WAITING\PYGZus{}FOR\PYGZus{}MOTION\PYGZus{}QUEUE = 4,
EXEC\PYGZus{}WAITING\PYGZus{}FOR\PYGZus{}IO = 5, EXEC\PYGZus{}WAITING\PYGZus{}FOR\PYGZus{}MOTION\PYGZus{}AND\PYGZus{}IO = 7,
EXEC\PYGZus{}WAITING\PYGZus{}FOR\PYGZus{}DELAY = 8, EXEC\PYGZus{}WAITING\PYGZus{}FOR\PYGZus{}SYSTEM\PYGZus{}CMD = 9,
EXEC\PYGZus{}WAITING\PYGZus{}FOR\PYGZus{}SPINDLE\PYGZus{}ORIENTED = 10).

feed\PYGZus{}hold\PYGZus{}enabled\PYGZus{}lb \PYGZhy{} returns boolean
enable flag for feed hold

feed\PYGZus{}override\PYGZus{}lb \PYGZhy{} returns boolean
 enable flag for feed override

file\PYGZus{}lb \PYGZhy{} returns string
currently loaded G\PYGZhy{}code filename with path

flood\PYGZus{}lb \PYGZhy{} returns integer that is used to lookup the state of OFF or ON
Flood status, either FLOOD\PYGZus{}OFF = 0) or FLOOD\PYGZus{}ON = 1)

g5x\PYGZus{}index\PYGZus{}lb \PYGZhy{} returns integer that is used to lookup the coordinate system name
currently active coordinate system, G54=1, G55=2 etc

g5x\PYGZus{}offset\PYGZus{}lb \PYGZhy{} returns tuple of floats
offsets of the currently active coordinate system X, Y, Z, U, V, W, A, B, C

gcodes\PYGZus{}lb \PYGZhy{} returns tuple of integers
Active G\PYGZhy{}codes for each modal group.
The integer values reflect the nominal G\PYGZhy{}code numbers multiplied by 10.
(Examples: 10 = G1, 430 = G43, 923 = G92.3)

homed \PYGZhy{} returns tuple of integers
currently homed joints, 0 = not homed, 1 = homed
\end{sphinxVerbatim}

\sphinxAtStartPar
See the {\hyperref[\detokenize{labels:axis-homed-labels}]{\emph{Axis Homed}}} labels for individual axis home
status.

\begin{sphinxVerbatim}[commandchars=\\\{\}]
id \PYGZhy{} returns integer
currently executing motion id

ini\PYGZus{}filename\PYGZus{}lb \PYGZhy{} returns string
path to the INI file passed to linuxcnc

inpos\PYGZus{}lb \PYGZhy{} returns boolean
machine\PYGZhy{}in\PYGZhy{}position flag

input\PYGZus{}timeout\PYGZus{}lb \PYGZhy{} returns boolean
flag for M66 timer in progress

interp\PYGZus{}state\PYGZus{}lb \PYGZhy{} returns integer that is used to lookup the state name
current state of RS274NGC interpreter. One of INTERP\PYGZus{}IDLE = 1,
INTERP\PYGZus{}READING = 2, INTERP\PYGZus{}PAUSED = 3, INTERP\PYGZus{}WAITING = 4

interpreter\PYGZus{}errcode\PYGZus{}lb \PYGZhy{} returns integer that is used to lookup the error name
current RS274NGC interpreter return code
        INTERP\PYGZus{}OK = 0,
        INTERP\PYGZus{}EXIT = 1,
        INTERP\PYGZus{}EXECUTE\PYGZus{}FINISH = 2,
        INTERP\PYGZus{}ENDFILE = 3,
        INTERP\PYGZus{}FILE\PYGZus{}NOT\PYGZus{}OPEN = 4,
        INTERP\PYGZus{}ERROR = 5

joint \PYGZhy{} returns tuple of dicts
reflecting current joint values
\end{sphinxVerbatim}

\sphinxAtStartPar
See the {\hyperref[\detokenize{labels:joint-status-labels}]{\emph{Joint Status}}} labels for individual joint
status items.

\begin{sphinxVerbatim}[commandchars=\\\{\}]
joint\PYGZus{}actual\PYGZus{}position \PYGZhy{} returns tuple of floats
actual joint positions

joint\PYGZus{}position \PYGZhy{} returns tuple of floats
desired joint positions

joints\PYGZus{}lb \PYGZhy{} returns integer
number of joints. Reflects [KINS]JOINTS INI value

kinematics\PYGZus{}type\PYGZus{}lb \PYGZhy{} returns integer that is used to lookup the kinematics name
The type of kinematics
        KINEMATICS\PYGZus{}IDENTITY = 1
        KINEMATICS\PYGZus{}FORWARD\PYGZus{}ONLY = 2
        KINEMATICS\PYGZus{}INVERSE\PYGZus{}ONLY = 3
        KINEMATICS\PYGZus{}BOTH = 4

limit \PYGZhy{} returns tuple of integers
axis limit masks. minHardLimit=1, maxHardLimit=2, minSoftLimit=4, maxSoftLimit=8

linear\PYGZus{}units\PYGZus{}lb \PYGZhy{} returns float \PYGZhy{} precision can be set
machine linear units per mm, reflects [TRAJ]LINEAR\PYGZus{}UNITS INI value

lube\PYGZus{}lb \PYGZhy{} returns integer
lube on flag

lube\PYGZus{}level\PYGZus{}lb \PYGZhy{} returns integer
reflects iocontrol.0.lube\PYGZus{}level

max\PYGZus{}acceleration\PYGZus{}lb \PYGZhy{} returns float \PYGZhy{} precision can be set
maximum acceleration. Reflects [TRAJ]MAX\PYGZus{}ACCELERATION

max\PYGZus{}velocity\PYGZus{}lb \PYGZhy{} returns float \PYGZhy{} precision can be set
maximum velocity. Reflects the current maximum velocity. If not modified by
halui.max\PYGZhy{}velocity or similar it should reflect [TRAJ]MAX\PYGZus{}VELOCITY

min\PYGZus{}jog\PYGZus{}vel\PYGZus{}lb \PYGZhy{} returns int
minimum jog velocity slider setting. Reflects the [DISPLAY] MIN\PYGZus{}LINEAR\PYGZus{}VELOCITY
setting in user units per minute.

max\PYGZus{}jog\PYGZus{}vel\PYGZus{}lb \PYGZhy{} returns int
maximum jog velocity slider setting. Reflects the [DISPLAY] MAX\PYGZus{}LINEAR\PYGZus{}VELOCITY
setting in user units per minute.

mcodes\PYGZus{}lb \PYGZhy{} returns tuple of 10 integers
currently active M\PYGZhy{}codes

mist\PYGZus{}lb \PYGZhy{} returns integer
Mist status, either MIST\PYGZus{}OFF = 0 or MIST\PYGZus{}ON = 1

motion\PYGZus{}line\PYGZus{}lb \PYGZhy{} returns integer
source line number motion is currently executing

motion\PYGZus{}mode\PYGZus{}lb \PYGZhy{} returns integer that is used to lookup the motion mode name
This is the mode of the Motion controller.
        TRAJ\PYGZus{}MODE\PYGZus{}FREE = 1
        TRAJ\PYGZus{}MODE\PYGZus{}COORD = 2
        TRAJ\PYGZus{}MODE\PYGZus{}TELEOP = 3

motion\PYGZus{}type\PYGZus{}lb \PYGZhy{} returns integer that is used to lookup the motion type name
The type of the currently executing motion. One of:
        MOTION\PYGZus{}TYPE\PYGZus{}TRAVERSE = 1
        MOTION\PYGZus{}TYPE\PYGZus{}FEED = 2
        MOTION\PYGZus{}TYPE\PYGZus{}ARC = 3
        MOTION\PYGZus{}TYPE\PYGZus{}TOOLCHANGE = 4
        MOTION\PYGZus{}TYPE\PYGZus{}PROBING = 5
        MOTION\PYGZus{}TYPE\PYGZus{}INDEXROTARY = 6
        Or 0 if no motion is currently taking place.

optional\PYGZus{}stop\PYGZus{}lb \PYGZhy{} returns integer
option stop flag

paused\PYGZus{}lb \PYGZhy{} returns boolean
motion paused flag

pocket\PYGZus{}prepped\PYGZus{}lb \PYGZhy{} returns integer
A Tx command completed, and this pocket is prepared. \PYGZhy{}1 if no prepared pocket

position \PYGZhy{} returns tuple of floats
trajectory position

probe\PYGZus{}tripped\PYGZus{}lb \PYGZhy{} returns boolean
True if probe has tripped

probe\PYGZus{}val\PYGZus{}lb \PYGZhy{} returns integer
reflects value of the motion.probe\PYGZhy{}input pin

probed\PYGZus{}position\PYGZus{}lb \PYGZhy{} returns tuple of floats
position where probe tripped

probing\PYGZus{}lb \PYGZhy{} returns boolean
True if a probe operation is in progress

program\PYGZus{}units\PYGZus{}lb \PYGZhy{} returns integer that is used to lookup the units name
        CANON\PYGZus{}UNITS\PYGZus{}INCHES = 1,
        CANON\PYGZus{}UNITS\PYGZus{}MM = 2,
        CANON\PYGZus{}UNITS\PYGZus{}CM = 3

queue\PYGZus{}lb \PYGZhy{} returns integer
current size of the trajectory planner queue

queue\PYGZus{}full\PYGZus{}lb \PYGZhy{} returns boolean
the trajectory planner queue is full

rapid\PYGZus{}override\PYGZus{}lb \PYGZhy{} returns percent
rapid override percent

rapidrate\PYGZus{}lb \PYGZhy{} returns float \PYGZhy{} precision can be set
rapid override scale, 1.0 = 100\PYGZpc{}

read\PYGZus{}line\PYGZus{}lb \PYGZhy{} returns integer
line the RS274NGC interpreter is currently reading

rotation\PYGZus{}xy\PYGZus{}lb \PYGZhy{} returns float \PYGZhy{} precision can be set
current XY rotation angle around Z axis

settings\PYGZus{}lb \PYGZhy{} returns tuple of floats
current interpreter settings
settings[0] = sequence number
settings[1] = feed rate
settings[2] = speed
settings[3] = G64 P blend tolerance
settings[4] = G64 Q naive CAM tolerance

spindles\PYGZus{}lb \PYGZhy{} returns tuple of dicts
returns the current spindle status

state\PYGZus{}lb \PYGZhy{} returns integer that is used to lookup the state name
current command execution status
One of RCS\PYGZus{}DONE = 1, RCS\PYGZus{}EXEC = 2, RCS\PYGZus{}ERROR = 3

task\PYGZus{}mode\PYGZus{}lb \PYGZhy{} returns integer that is used to lookup the task mode name
current task mode
One of  MODE\PYGZus{}MANUAL = 1, MODE\PYGZus{}AUTO = 2, MODE\PYGZus{}MDI = 3

task\PYGZus{}paused\PYGZus{}lb \PYGZhy{} returns integer
task paused flag, not paused = 0, paused = 1

task\PYGZus{}state\PYGZus{}lb \PYGZhy{} returns integer that is used to lookup the task state name
current task state
One of STATE\PYGZus{}ESTOP = 1, STATE\PYGZus{}ESTOP\PYGZus{}RESET = 2, STATE\PYGZus{}OFF = 3 STATE\PYGZus{}ON = 4
STATE\PYGZus{}OFF is never seen

tool\PYGZus{}in\PYGZus{}spindle\PYGZus{}lb \PYGZhy{} returns integer
current tool number in spindle (0 if no tool loaded)

tool\PYGZus{}from\PYGZus{}pocket\PYGZus{}lb \PYGZhy{} returns integer
pocket number for the currently loaded tool (0 if no tool loaded)

tool\PYGZus{}offset\PYGZus{}lb \PYGZhy{} returns tuple of floats
offset values of the current tool

tool\PYGZus{}table\PYGZus{}lb \PYGZhy{} returns tuple of tool\PYGZus{}results
list of tool entries. Each entry is a sequence of the following fields: id,
xoffset, yoffset, zoffset, aoffset, boffset, coffset, uoffset, voffset,
woffset, diameter, frontangle, backangle, orientation. The id and orientation
are integers and the rest are floats.
If id = \PYGZhy{}1 no tools are in the tool table.
\end{sphinxVerbatim}

\begin{sphinxadmonition}{note}{Note:}
\sphinxAtStartPar
You don’t have to use all the labels; only use the ones you need.
\end{sphinxadmonition}


\section{Axis Status}
\label{\detokenize{labels:axis-status}}
\sphinxAtStartPar
The Axis status contains status items for all 9 axes. Replace the \sphinxtitleref{n} with
the number of the axis. Axis numbers start at 0 and go through 8. Returns a
float


\begin{savenotes}\sphinxattablestart
\sphinxthistablewithglobalstyle
\centering
\sphinxcapstartof{table}
\sphinxthecaptionisattop
\sphinxcaption{Axis Status Labels}\label{\detokenize{labels:id2}}
\sphinxaftertopcaption
\begin{tabular}[t]{|\X{60}{120}|\X{60}{120}|}
\sphinxtoprule
\sphinxtableatstartofbodyhook
\sphinxAtStartPar
axis\_n\_max\_position\_limit\_lb
&
\sphinxAtStartPar
axis\_n\_min\_position\_limit\_lb
\\
\sphinxhline
\sphinxAtStartPar
axis\_n\_velocity\_lb
&
\sphinxAtStartPar
axis\_n\_vel\_per\_min\_lb
\\
\sphinxbottomrule
\end{tabular}
\sphinxtableafterendhook\par
\sphinxattableend\end{savenotes}

\begin{sphinxadmonition}{note}{Note:}
\sphinxAtStartPar
The Axis velocity label only reports back \sphinxtitleref{jogging} speed; use the
joint velocity label for \sphinxtitleref{linear} speed.
\end{sphinxadmonition}


\section{Joint Status Labels}
\label{\detokenize{labels:joint-status-labels}}
\sphinxAtStartPar
The Joint status contains status items for 16 joints. Replace the \sphinxtitleref{n} with
the number of the joint. Joint numbers start at 0 and go through 15


\begin{savenotes}\sphinxattablestart
\sphinxthistablewithglobalstyle
\centering
\sphinxcapstartof{table}
\sphinxthecaptionisattop
\sphinxcaption{Joint Status Labels}\label{\detokenize{labels:id3}}
\sphinxaftertopcaption
\begin{tabular}[t]{|\X{60}{120}|\X{60}{120}|}
\sphinxtoprule
\sphinxtableatstartofbodyhook
\sphinxAtStartPar
joint\_backlash\_n\_lb
&
\sphinxAtStartPar
joint\_input\_n\_lb
\\
\sphinxhline
\sphinxAtStartPar
joint\_min\_position\_limit\_n\_lb
&
\sphinxAtStartPar
joint\_enabled\_n\_lb
\\
\sphinxhline
\sphinxAtStartPar
joint\_jointType\_n\_lb
&
\sphinxAtStartPar
joint\_in\_soft\_limit\_n\_lb
\\
\sphinxhline
\sphinxAtStartPar
joint\_fault\_n\_lb
&
\sphinxAtStartPar
joint\_max\_ferror\_n\_lb
\\
\sphinxhline
\sphinxAtStartPar
joint\_output\_n\_lb
&
\sphinxAtStartPar
joint\_ferror\_current\_n\_lb
\\
\sphinxhline
\sphinxAtStartPar
joint\_max\_hard\_limit\_n\_lb
&
\sphinxAtStartPar
joint\_override\_limits\_n\_lb
\\
\sphinxhline
\sphinxAtStartPar
joint\_ferror\_highmark\_n\_lb
&
\sphinxAtStartPar
joint\_max\_position\_limit\_n\_lb
\\
\sphinxhline
\sphinxAtStartPar
joint\_units\_n\_lb
&
\sphinxAtStartPar
joint\_homed\_n\_lb
\\
\sphinxhline
\sphinxAtStartPar
joint\_max\_soft\_limit\_n\_lb
&
\sphinxAtStartPar
joint\_vel\_sec\_n\_lb
\\
\sphinxhline
\sphinxAtStartPar
joint\_vel\_min\_n\_lb
&
\sphinxAtStartPar
joint\_homing\_n\_lb
\\
\sphinxhline
\sphinxAtStartPar
joint\_min\_ferror\_n\_lb
&
\sphinxAtStartPar
joint\_inpos\_n\_lb
\\
\sphinxhline
\sphinxAtStartPar
joint\_min\_hard\_limit\_n\_lb
&\\
\sphinxbottomrule
\end{tabular}
\sphinxtableafterendhook\par
\sphinxattableend\end{savenotes}


\section{Special Labels}
\label{\detokenize{labels:special-labels}}
\sphinxAtStartPar
Run from line label \sphinxtitleref{start\_line\_lb}


\section{Axis Position Labels}
\label{\detokenize{labels:axis-position-labels}}
\sphinxAtStartPar
Axis machine position labels (no offsets.) Returns a float


\begin{savenotes}\sphinxattablestart
\sphinxthistablewithglobalstyle
\centering
\sphinxcapstartof{table}
\sphinxthecaptionisattop
\sphinxcaption{Machine Absolute Position Status Labels}\label{\detokenize{labels:id4}}
\sphinxaftertopcaption
\begin{tabular}[t]{|\X{40}{120}|\X{40}{120}|\X{40}{120}|}
\sphinxtoprule
\sphinxtableatstartofbodyhook
\sphinxAtStartPar
actual\_lb\_x
&
\sphinxAtStartPar
actual\_lb\_y
&
\sphinxAtStartPar
actual\_lb\_z
\\
\sphinxhline
\sphinxAtStartPar
actual\_lb\_a
&
\sphinxAtStartPar
actual\_lb\_b
&
\sphinxAtStartPar
actual\_lb\_c
\\
\sphinxhline
\sphinxAtStartPar
actual\_lb\_u
&
\sphinxAtStartPar
actual\_lb\_v
&
\sphinxAtStartPar
actual\_lb\_w
\\
\sphinxbottomrule
\end{tabular}
\sphinxtableafterendhook\par
\sphinxattableend\end{savenotes}

\sphinxAtStartPar
Axis position labels \sphinxtitleref{including} all offsets. Returns a float


\begin{savenotes}\sphinxattablestart
\sphinxthistablewithglobalstyle
\centering
\sphinxcapstartof{table}
\sphinxthecaptionisattop
\sphinxcaption{DRO Relative Status Labels}\label{\detokenize{labels:id5}}
\sphinxaftertopcaption
\begin{tabular}[t]{|\X{40}{120}|\X{40}{120}|\X{40}{120}|}
\sphinxtoprule
\sphinxtableatstartofbodyhook
\sphinxAtStartPar
dro\_lb\_x
&
\sphinxAtStartPar
dro\_lb\_y
&
\sphinxAtStartPar
dro\_lb\_z
\\
\sphinxhline
\sphinxAtStartPar
dro\_lb\_a
&
\sphinxAtStartPar
dro\_lb\_b
&
\sphinxAtStartPar
dro\_lb\_c
\\
\sphinxhline
\sphinxAtStartPar
dro\_lb\_u
&
\sphinxAtStartPar
dro\_lb\_v
&
\sphinxAtStartPar
dro\_lb\_w
\\
\sphinxbottomrule
\end{tabular}
\sphinxtableafterendhook\par
\sphinxattableend\end{savenotes}


\section{Axis Distance to Go labels}
\label{\detokenize{labels:axis-distance-to-go-labels}}

\begin{savenotes}\sphinxattablestart
\sphinxthistablewithglobalstyle
\centering
\sphinxcapstartof{table}
\sphinxthecaptionisattop
\sphinxcaption{Distance to Go Labels}\label{\detokenize{labels:id6}}
\sphinxaftertopcaption
\begin{tabular}[t]{|\X{40}{120}|\X{40}{120}|\X{40}{120}|}
\sphinxtoprule
\sphinxtableatstartofbodyhook
\sphinxAtStartPar
dtg\_lb\_x
&
\sphinxAtStartPar
dtg\_lb\_y
&
\sphinxAtStartPar
dtg\_lb\_z
\\
\sphinxhline
\sphinxAtStartPar
dtg\_lb\_a
&
\sphinxAtStartPar
dtg\_lb\_b
&
\sphinxAtStartPar
dtg\_lb\_c
\\
\sphinxhline
\sphinxAtStartPar
dtg\_lb\_u
&
\sphinxAtStartPar
dtg\_lb\_v
&
\sphinxAtStartPar
dtg\_lb\_w
\\
\sphinxbottomrule
\end{tabular}
\sphinxtableafterendhook\par
\sphinxattableend\end{savenotes}


\section{Axis Homed Labels}
\label{\detokenize{labels:axis-homed-labels}}

\begin{savenotes}\sphinxattablestart
\sphinxthistablewithglobalstyle
\centering
\sphinxcapstartof{table}
\sphinxthecaptionisattop
\sphinxcaption{Axis Homed Labels}\label{\detokenize{labels:id7}}
\sphinxaftertopcaption
\begin{tabular}[t]{|\X{40}{120}|\X{40}{120}|\X{40}{120}|}
\sphinxtoprule
\sphinxtableatstartofbodyhook
\sphinxAtStartPar
home\_lb\_0
&
\sphinxAtStartPar
home\_lb\_1
&
\sphinxAtStartPar
home\_lb\_2
\\
\sphinxhline
\sphinxAtStartPar
home\_lb\_3
&
\sphinxAtStartPar
home\_lb\_4
&
\sphinxAtStartPar
home\_lb\_5
\\
\sphinxhline
\sphinxAtStartPar
home\_lb\_6
&
\sphinxAtStartPar
home\_lb\_7
&
\sphinxAtStartPar
home\_lb\_8
\\
\sphinxbottomrule
\end{tabular}
\sphinxtableafterendhook\par
\sphinxattableend\end{savenotes}


\section{Offset Labels}
\label{\detokenize{labels:offset-labels}}
\sphinxAtStartPar
Offsets for the currently active G5x coordinate system. Returns a float


\begin{savenotes}\sphinxattablestart
\sphinxthistablewithglobalstyle
\centering
\sphinxcapstartof{table}
\sphinxthecaptionisattop
\sphinxcaption{G5x Status Labels}\label{\detokenize{labels:id8}}
\sphinxaftertopcaption
\begin{tabular}[t]{|\X{40}{120}|\X{40}{120}|\X{40}{120}|}
\sphinxtoprule
\sphinxtableatstartofbodyhook
\sphinxAtStartPar
g5x\_lb\_x
&
\sphinxAtStartPar
g5x\_lb\_y
&
\sphinxAtStartPar
g5x\_lb\_z
\\
\sphinxhline
\sphinxAtStartPar
g5x\_lb\_a
&
\sphinxAtStartPar
g5x\_lb\_b
&
\sphinxAtStartPar
g5x\_lb\_c
\\
\sphinxhline
\sphinxAtStartPar
g5x\_lb\_u
&
\sphinxAtStartPar
g5x\_lb\_v
&
\sphinxAtStartPar
g5x\_lb\_w
\\
\sphinxbottomrule
\end{tabular}
\sphinxtableafterendhook\par
\sphinxattableend\end{savenotes}

\sphinxAtStartPar
Offsets for G92. Returns a float


\begin{savenotes}\sphinxattablestart
\sphinxthistablewithglobalstyle
\centering
\sphinxcapstartof{table}
\sphinxthecaptionisattop
\sphinxcaption{G92 Status Labels}\label{\detokenize{labels:id9}}
\sphinxaftertopcaption
\begin{tabular}[t]{|\X{40}{120}|\X{40}{120}|\X{40}{120}|}
\sphinxtoprule
\sphinxtableatstartofbodyhook
\sphinxAtStartPar
g92\_lb\_x
&
\sphinxAtStartPar
g92\_lb\_y
&
\sphinxAtStartPar
g92\_lb\_z
\\
\sphinxhline
\sphinxAtStartPar
g92\_lb\_a
&
\sphinxAtStartPar
g92\_lb\_b
&
\sphinxAtStartPar
g92\_lb\_c
\\
\sphinxhline
\sphinxAtStartPar
g92\_lb\_u
&
\sphinxAtStartPar
g92\_lb\_v
&
\sphinxAtStartPar
g92\_lb\_w
\\
\sphinxbottomrule
\end{tabular}
\sphinxtableafterendhook\par
\sphinxattableend\end{savenotes}


\section{Velocity Labels}
\label{\detokenize{labels:velocity-labels}}
\sphinxAtStartPar
Tool velocity using two perpendicular joint velocities.

\sphinxAtStartPar
Name the label \sphinxtitleref{two\_vel\_lb} and add two int type Dynamic Properties called
\sphinxtitleref{joint\_0} and \sphinxtitleref{joint\_1} and set the values to the perpendicular joint numbers
you want to calculate. Typically this would be for the X and Y axes.

\sphinxAtStartPar
To select an int type of Dynamic Property, select \sphinxtitleref{Other} after clicking on
the green plus sign

\noindent{\hspace*{\fill}\sphinxincludegraphics{{status-05}.png}\hspace*{\fill}}

\sphinxAtStartPar
Then select the Property Type of \sphinxtitleref{int}

\noindent{\hspace*{\fill}\sphinxincludegraphics{{status-06}.png}\hspace*{\fill}}

\sphinxAtStartPar
The two Dynamic Properties should look like this

\noindent{\hspace*{\fill}\sphinxincludegraphics{{status-07}.png}\hspace*{\fill}}

\sphinxAtStartPar
Tool velocity using \sphinxtitleref{three} perpendicular joint velocities.

\sphinxAtStartPar
Name the label \sphinxtitleref{three\_vel\_lb} and add three int type Dynamic Properties called
\sphinxtitleref{joint\_0}, \sphinxtitleref{joint\_1} and \sphinxtitleref{joint\_2} and set the values to the perpendicular
joint numbers you want to calculate. Typically this would be for the X, Y and
Z axes.


\section{I/O Status}
\label{\detokenize{labels:i-o-status}}
\sphinxAtStartPar
The I/O status contains status items for 64 I/O’s. Replace the \sphinxtitleref{n} with the
number of the I/O. I/O numbers start at 0 and go through 63. Analog I/O
returns a float. For example a QLabel with an object name of din\_5\_lb will
show the status of the \sphinxtitleref{motion.digital\sphinxhyphen{}in\sphinxhyphen{}05} HAL pin


\begin{savenotes}\sphinxattablestart
\sphinxthistablewithglobalstyle
\centering
\sphinxcapstartof{table}
\sphinxthecaptionisattop
\sphinxcaption{I/O Status Labels}\label{\detokenize{labels:id10}}
\sphinxaftertopcaption
\begin{tabular}[t]{|\X{40}{80}|\X{40}{80}|}
\sphinxtoprule
\sphinxtableatstartofbodyhook
\sphinxAtStartPar
HAL Pin
&
\sphinxAtStartPar
Label Name
\\
\sphinxhline
\sphinxAtStartPar
motion.analog\sphinxhyphen{}in\sphinxhyphen{}nn
&
\sphinxAtStartPar
ain\_n\_lb
\\
\sphinxhline
\sphinxAtStartPar
motion.analog\sphinxhyphen{}out\sphinxhyphen{}nn
&
\sphinxAtStartPar
aout\_n\_lb
\\
\sphinxhline
\sphinxAtStartPar
motion.digital\sphinxhyphen{}in\sphinxhyphen{}nn
&
\sphinxAtStartPar
din\_n\_lb
\\
\sphinxhline
\sphinxAtStartPar
motion.digital\sphinxhyphen{}out\sphinxhyphen{}nn
&
\sphinxAtStartPar
dout\_n\_lb
\\
\sphinxbottomrule
\end{tabular}
\sphinxtableafterendhook\par
\sphinxattableend\end{savenotes}

\sphinxstepscope


\chapter{Menu}
\label{\detokenize{menu:menu}}\label{\detokenize{menu::doc}}
\sphinxAtStartPar
\sphinxhref{https://youtu.be/ukwunHGCglk}{Adding Menu Items Tutorial}
\sphinxhref{https://youtu.be/X\_SMoJ9sYbI}{Tool Bar Buttons Tutorial}

\begin{sphinxadmonition}{note}{Note:}
\sphinxAtStartPar
Every menu item has a command button, so you don’t need to use any
menu items if you don’t want to.
\end{sphinxadmonition}

\sphinxAtStartPar
Adding a menu item creates an action. When you create File \textgreater{} Open menu, the
\sphinxtitleref{actionOpen} action is created.

\noindent{\hspace*{\fill}\sphinxincludegraphics{{menu-01}.png}\hspace*{\fill}}

\begin{sphinxadmonition}{warning}{Warning:}
\sphinxAtStartPar
If you use the full\sphinxhyphen{}screen option, you will not be able to exit the
application if you don’t have the Exit action or an Exit Push Button or Press
ALT\sphinxhyphen{}F4 to close the GUI.
\end{sphinxadmonition}

\sphinxAtStartPar
This shows the typical menu categories which are the first items in each menu.
The image is from the Qt Designer.

\noindent{\hspace*{\fill}\sphinxincludegraphics{{menu-02}.png}\hspace*{\fill}}

\sphinxAtStartPar
The following table shows the menu name you type into Qt Designer and the action
name that is created by the Qt Designer. Menu categories like \sphinxtitleref{File} don’t
create an action name.


\begin{savenotes}
\sphinxatlongtablestart
\sphinxthistablewithglobalstyle
\begin{longtable}[c]{|l|l|}
\sphinxthelongtablecaptionisattop
\caption{Menu Items\strut}\label{\detokenize{menu:id1}}\\*[\sphinxlongtablecapskipadjust]
\sphinxtoprule
\endfirsthead

\multicolumn{2}{c}{\sphinxnorowcolor
    \makebox[0pt]{\sphinxtablecontinued{\tablename\ \thetable{} \textendash{} continued from previous page}}%
}\\
\sphinxtoprule
\endhead

\sphinxbottomrule
\multicolumn{2}{r}{\sphinxnorowcolor
    \makebox[0pt][r]{\sphinxtablecontinued{continues on next page}}%
}\\
\endfoot

\endlastfoot
\sphinxtableatstartofbodyhook

\sphinxAtStartPar
\sphinxstylestrong{File}
&
\sphinxAtStartPar
\sphinxstylestrong{Action Name}
\\
\sphinxhline
\sphinxAtStartPar
Open
&
\sphinxAtStartPar
actionOpen
\\
\sphinxhline
\sphinxAtStartPar
Edit
&
\sphinxAtStartPar
actionEdit
\\
\sphinxhline
\sphinxAtStartPar
Reload
&
\sphinxAtStartPar
actionReload
\\
\sphinxhline
\sphinxAtStartPar
Save As
&
\sphinxAtStartPar
actionSave\_As
\\
\sphinxhline
\sphinxAtStartPar
Edit Tool Table
&
\sphinxAtStartPar
actionEdit\_Tool\_Table
\\
\sphinxhline
\sphinxAtStartPar
Reload Tool Table
&
\sphinxAtStartPar
actionReload\_Tool\_Table
\\
\sphinxhline
\sphinxAtStartPar
Ladder Editor
&
\sphinxAtStartPar
actionLadder\_Editor
\\
\sphinxhline
\sphinxAtStartPar
Quit
&
\sphinxAtStartPar
actionQuit
\\
\sphinxhline&\\
\sphinxhline
\sphinxAtStartPar
\sphinxstylestrong{Machine}
&
\sphinxAtStartPar
\sphinxstylestrong{Action Name}
\\
\sphinxhline
\sphinxAtStartPar
E Stop
&
\sphinxAtStartPar
actionE\_Stop
\\
\sphinxhline
\sphinxAtStartPar
Power
&
\sphinxAtStartPar
action\_Power
\\
\sphinxhline
\sphinxAtStartPar
Run
&
\sphinxAtStartPar
actionRun
\\
\sphinxhline
\sphinxAtStartPar
Run From Line
&
\sphinxAtStartPar
actionRun\_From\_Line
\\
\sphinxhline
\sphinxAtStartPar
Step
&
\sphinxAtStartPar
actionStep
\\
\sphinxhline
\sphinxAtStartPar
Pause
&
\sphinxAtStartPar
actionPause
\\
\sphinxhline
\sphinxAtStartPar
Resume
&
\sphinxAtStartPar
actionResume
\\
\sphinxhline
\sphinxAtStartPar
Stop
&
\sphinxAtStartPar
actionStop
\\
\sphinxhline
\sphinxAtStartPar
Clear MDI History
&
\sphinxAtStartPar
actionClear\_MDI\_History
\\
\sphinxhline
\sphinxAtStartPar
Copy MDI History
&
\sphinxAtStartPar
actionCopy\_MDI\_History
\\
\sphinxhline
\sphinxAtStartPar
Homing
&
\sphinxAtStartPar
this creates a home menu item for each axis
\\
\sphinxhline
\sphinxAtStartPar
Unhoming
&
\sphinxAtStartPar
this creates a unhome menu item for each axis
\\
\sphinxhline
\sphinxAtStartPar
Clear Offsets
&
\sphinxAtStartPar
this creates a clear offsets for each coordinate system
\\
\sphinxhline&\\
\sphinxhline
\sphinxAtStartPar
\sphinxstylestrong{Programs}
&
\sphinxAtStartPar
\sphinxstylestrong{Action Name}
\\
\sphinxhline
\sphinxAtStartPar
Show HAL
&
\sphinxAtStartPar
actionShow\_HAL
\\
\sphinxhline
\sphinxAtStartPar
HAL Meter
&
\sphinxAtStartPar
actionHAL\_Meter
\\
\sphinxhline
\sphinxAtStartPar
HAL Scope
&
\sphinxAtStartPar
actionHAL\_Scope
\\
\sphinxhline&\\
\sphinxhline
\sphinxAtStartPar
\sphinxstylestrong{View}
&
\sphinxAtStartPar
\sphinxstylestrong{Action Name}
\\
\sphinxhline
\sphinxAtStartPar
DRO
&
\sphinxAtStartPar
actionDRO
\\
\sphinxhline
\sphinxAtStartPar
Limits
&
\sphinxAtStartPar
actionLimits
\\
\sphinxhline
\sphinxAtStartPar
Extents Option
&
\sphinxAtStartPar
actionExtents\_Option
\\
\sphinxhline
\sphinxAtStartPar
Live Plot
&
\sphinxAtStartPar
actionLive\_Plot
\\
\sphinxhline
\sphinxAtStartPar
Velocity
&
\sphinxAtStartPar
actionVelocity
\\
\sphinxhline
\sphinxAtStartPar
Metric Units
&
\sphinxAtStartPar
actionMetric\_Units
\\
\sphinxhline
\sphinxAtStartPar
Program
&
\sphinxAtStartPar
actionProgram
\\
\sphinxhline
\sphinxAtStartPar
Rapids
&
\sphinxAtStartPar
actionRapids
\\
\sphinxhline
\sphinxAtStartPar
Tool
&
\sphinxAtStartPar
actionTool
\\
\sphinxhline
\sphinxAtStartPar
Lathe Radius
&
\sphinxAtStartPar
actionLathe\_Radius
\\
\sphinxhline
\sphinxAtStartPar
DTG
&
\sphinxAtStartPar
actionDTG
\\
\sphinxhline
\sphinxAtStartPar
Offsets
&
\sphinxAtStartPar
actionOffsets
\\
\sphinxhline
\sphinxAtStartPar
Overlay
&
\sphinxAtStartPar
actionOverlay
\\
\sphinxhline
\sphinxAtStartPar
Clear Live Plot
&
\sphinxAtStartPar
actionClear\_Live\_Plot
\\
\sphinxhline&\\
\sphinxhline
\sphinxAtStartPar
\sphinxstylestrong{Help}
&
\sphinxAtStartPar
\sphinxstylestrong{Action Name}
\\
\sphinxhline
\sphinxAtStartPar
About
&
\sphinxAtStartPar
actionAbout
\\
\sphinxhline
\sphinxAtStartPar
Quick Reference
&
\sphinxAtStartPar
actionQuick\_Reference
\\
\sphinxbottomrule
\end{longtable}
\sphinxtableafterendhook
\sphinxatlongtableend
\end{savenotes}


\section{Action Names}
\label{\detokenize{menu:action-names}}
\sphinxAtStartPar
When you add a menu item, it creates an action and the Object Name is created
from the menu name automatically.

\sphinxAtStartPar
The Object Name must match the above items \sphinxtitleref{exactly} in order to be discovered
by Flex GUI:

\noindent{\hspace*{\fill}\sphinxincludegraphics{{actions-01}.png}\hspace*{\fill}}


\section{Recent Files}
\label{\detokenize{menu:recent-files}}
\begin{sphinxadmonition}{note}{Note:}
\sphinxAtStartPar
The Recent menu item is added after the Open menu. There must be at
least one menu item after Open for the Recent menu to be added.
\end{sphinxadmonition}

\sphinxAtStartPar
Location of the Recent menu after the Open menu:

\noindent{\hspace*{\fill}\sphinxincludegraphics{{menu-03}.png}\hspace*{\fill}}


\section{Tool Bars}
\label{\detokenize{menu:tool-bars}}
\sphinxAtStartPar
If you right\sphinxhyphen{}click on the main window, you can add a Tool Bar:

\noindent{\hspace*{\fill}\sphinxincludegraphics{{tool-bar-01}.png}\hspace*{\fill}}

\sphinxAtStartPar
To add actions to the Tool Bar, drag them from the Action Editor and drop them
in the Tool Bar:

\noindent{\hspace*{\fill}\sphinxincludegraphics{{tool-bar-02}.png}\hspace*{\fill}}

\sphinxAtStartPar
To set the style of a Tool Bar Button, use the action name and replace action
with \sphinxtitleref{flex\_} for example the actionQuit would be \sphinxtitleref{flex\_Quit}. See \sphinxtitleref{Tool Bar
Buttons} in the stylesheet examples.


\section{Shortcut Keys}
\label{\detokenize{menu:shortcut-keys}}
\sphinxAtStartPar
Shortcut keys can be added in the Property Editor by clicking in the shortcut
Value box and pressing the key or key combination you want to use. You can
change text, icon Text, or tool Tip also.

\noindent{\hspace*{\fill}\sphinxincludegraphics{{actions-02}.png}\hspace*{\fill}}

\begin{sphinxadmonition}{note}{Note:}
\sphinxAtStartPar
A toolTip can be handy, however they might not work on touchscreens.
\end{sphinxadmonition}

\sphinxstepscope


\chapter{Controls}
\label{\detokenize{controls:controls}}\label{\detokenize{controls::doc}}
\sphinxAtStartPar
\sphinxhref{https://youtu.be/X\_SMoJ9sYbI}{Command Buttons Tutorial}
\sphinxhref{https://youtu.be/R8Z\_oCdaAXM}{Home Controls Tutorial}


\section{Push Buttons}
\label{\detokenize{controls:push-buttons}}
\sphinxAtStartPar
Controls are QPushButtons that can be placed anywhere you like. Use the Name
from the list below for each control widget objectName. Replace the \sphinxtitleref{(0\sphinxhyphen{}8)}
with the joint number or axis index. More controls are in {\hyperref[\detokenize{tools::doc}]{\sphinxcrossref{\DUrole{doc}{Tools}}}}.


\begin{savenotes}
\sphinxatlongtablestart
\sphinxthistablewithglobalstyle
\begin{longtable}[c]{|l|l|}
\sphinxthelongtablecaptionisattop
\caption{Control Push Buttons\strut}\label{\detokenize{controls:id2}}\\*[\sphinxlongtablecapskipadjust]
\sphinxtoprule
\endfirsthead

\multicolumn{2}{c}{\sphinxnorowcolor
    \makebox[0pt]{\sphinxtablecontinued{\tablename\ \thetable{} \textendash{} continued from previous page}}%
}\\
\sphinxtoprule
\endhead

\sphinxbottomrule
\multicolumn{2}{r}{\sphinxnorowcolor
    \makebox[0pt][r]{\sphinxtablecontinued{continues on next page}}%
}\\
\endfoot

\endlastfoot
\sphinxtableatstartofbodyhook

\sphinxAtStartPar
\sphinxstylestrong{Control Function}
&
\sphinxAtStartPar
\sphinxstylestrong{Object Name}
\\
\sphinxhline
\sphinxAtStartPar
Open a G\sphinxhyphen{}code File
&
\sphinxAtStartPar
open\_pb
\\
\sphinxhline
\sphinxAtStartPar
Edit a G\sphinxhyphen{}code File
&
\sphinxAtStartPar
edit\_pb
\\
\sphinxhline
\sphinxAtStartPar
Reload a G\sphinxhyphen{}code File
&
\sphinxAtStartPar
reload\_pb
\\
\sphinxhline
\sphinxAtStartPar
Edit Tool Table
&
\sphinxAtStartPar
edit\_tool\_table\_pb
\\
\sphinxhline
\sphinxAtStartPar
Edit Ladder
&
\sphinxAtStartPar
edit\_ladder\_pb
\\
\sphinxhline
\sphinxAtStartPar
Reload Tool Table
&
\sphinxAtStartPar
reload\_tool\_table\_pb
\\
\sphinxhline
\sphinxAtStartPar
Save
&
\sphinxAtStartPar
save\_pb
\\
\sphinxhline
\sphinxAtStartPar
Save As a New Name
&
\sphinxAtStartPar
save\_as\_pb
\\
\sphinxhline
\sphinxAtStartPar
Quit the Program
&
\sphinxAtStartPar
quit\_pb
\\
\sphinxhline
\sphinxAtStartPar
E\sphinxhyphen{}Stop Toggle
&
\sphinxAtStartPar
estop\_pb
\\
\sphinxhline
\sphinxAtStartPar
Power Toggle
&
\sphinxAtStartPar
power\_pb
\\
\sphinxhline
\sphinxAtStartPar
Run a Loaded G\sphinxhyphen{}code File
&
\sphinxAtStartPar
run\_pb
\\
\sphinxhline
\sphinxAtStartPar
Run From Line
&
\sphinxAtStartPar
run\_from\_line\_pb
\\
\sphinxhline
\sphinxAtStartPar
Step one Logical Line
&
\sphinxAtStartPar
step\_pb
\\
\sphinxhline
\sphinxAtStartPar
Pause a Running Program
&
\sphinxAtStartPar
pause\_pb
\\
\sphinxhline
\sphinxAtStartPar
Resume a Paused Program
&
\sphinxAtStartPar
resume\_pb
\\
\sphinxhline
\sphinxAtStartPar
Stop a Running Program
&
\sphinxAtStartPar
stop\_pb
\\
\sphinxhline
\sphinxAtStartPar
Home All Joints
&
\sphinxAtStartPar
home\_all\_pb
\\
\sphinxhline
\sphinxAtStartPar
Home a Joint (0\sphinxhyphen{}8)
&
\sphinxAtStartPar
home\_pb\_(0\sphinxhyphen{}8)
\\
\sphinxhline
\sphinxAtStartPar
Unhome All Joints
&
\sphinxAtStartPar
unhome\_all\_pb
\\
\sphinxhline
\sphinxAtStartPar
Unhome a Joint (0\sphinxhyphen{}8)
&
\sphinxAtStartPar
unhome\_pb\_(0\sphinxhyphen{}8)
\\
\sphinxhline
\sphinxAtStartPar
Zero an Axis
&
\sphinxAtStartPar
zero\_(axis letter)\_pb
\\
\sphinxhline
\sphinxAtStartPar
Manual Mode
&
\sphinxAtStartPar
manual\_mode\_pb
\\
\sphinxhline
\sphinxAtStartPar
Flood Toggle
&
\sphinxAtStartPar
flood\_pb
\\
\sphinxhline
\sphinxAtStartPar
Mist Toggle
&
\sphinxAtStartPar
mist\_pb
\\
\sphinxhline
\sphinxAtStartPar
Clear Error History
&
\sphinxAtStartPar
clear\_errors\_pb
\\
\sphinxhline
\sphinxAtStartPar
Copy Error History
&
\sphinxAtStartPar
copy\_errors\_pb
\\
\sphinxhline
\sphinxAtStartPar
Clear Information History
&
\sphinxAtStartPar
clear\_info\_pb
\\
\sphinxhline
\sphinxAtStartPar
Show HAL
&
\sphinxAtStartPar
show\_hal\_pb
\\
\sphinxhline
\sphinxAtStartPar
HAL Meter
&
\sphinxAtStartPar
hal\_meter\_pb
\\
\sphinxhline
\sphinxAtStartPar
HAL Scope
&
\sphinxAtStartPar
hal\_scope\_pb
\\
\sphinxhline
\sphinxAtStartPar
Help About
&
\sphinxAtStartPar
about\_pb
\\
\sphinxhline
\sphinxAtStartPar
Quick Reference
&
\sphinxAtStartPar
quick\_reference\_pb
\\
\sphinxbottomrule
\end{longtable}
\sphinxtableafterendhook
\sphinxatlongtableend
\end{savenotes}

\begin{sphinxadmonition}{note}{Note:}
\sphinxAtStartPar
You don’t have to use any of these controls; Flex GUI is flexible.
\end{sphinxadmonition}

\noindent{\hspace*{\fill}\sphinxincludegraphics{{controls-01}.png}\hspace*{\fill}}

\begin{sphinxadmonition}{note}{Note:}
\sphinxAtStartPar
Touch\sphinxhyphen{}Off buttons require a Double Spin Box named \sphinxtitleref{touchoff\_dsb}
\end{sphinxadmonition}

\begin{sphinxadmonition}{note}{Note:}
\sphinxAtStartPar
Tool\sphinxhyphen{}Touch\sphinxhyphen{}Off buttons require a Double Spin Box named
\sphinxtitleref{tool\_touchoff\_dsb}
\end{sphinxadmonition}


\section{E Stop and Power}
\label{\detokenize{controls:e-stop-and-power}}
\sphinxAtStartPar
The E Stop push button Open/Closed state text can be set by adding two string
type Dynamic Properties \sphinxtitleref{open\_text} and \sphinxtitleref{closed\_text}. The text in those two
properties will be used if found. See {\hyperref[\detokenize{property::doc}]{\sphinxcrossref{\DUrole{doc}{Dynamic Properties}}}}

\noindent{\hspace*{\fill}\sphinxincludegraphics{{estop-01}.png}\hspace*{\fill}}

\sphinxAtStartPar
The Power push button On/Off state text can be set by adding two string type
Dynamic Properties \sphinxtitleref{on\_text} and \sphinxtitleref{off\_text}. The text in those two properties
will be used if found. The default text is \sphinxtitleref{Power Off} and
\sphinxtitleref{Power On}.

\noindent{\hspace*{\fill}\sphinxincludegraphics{{power-01}.png}\hspace*{\fill}}

\begin{sphinxadmonition}{note}{Note:}
\sphinxAtStartPar
To have two words be above and below insert a backslasgh and n between the words.
\end{sphinxadmonition}

\noindent{\hspace*{\fill}\sphinxincludegraphics{{estop-02}.png}\hspace*{\fill}}

\sphinxAtStartPar
This is how the above looks in the GUI

\noindent{\hspace*{\fill}\sphinxincludegraphics{{estop-03}.png}\hspace*{\fill}}


\section{LED Buttons}
\label{\detokenize{controls:led-buttons}}\label{\detokenize{controls:id1}}
\begin{sphinxadmonition}{important}{Important:}
\sphinxAtStartPar
This is implmented in version 1.2.0.
\end{sphinxadmonition}

\sphinxAtStartPar
Some push buttons can have a LED to indicate on and off states. The LED is added
to the push button with a Bool type Dynamic Property called \sphinxtitleref{led\_indicator}.


\begin{savenotes}\sphinxattablestart
\sphinxthistablewithglobalstyle
\centering
\sphinxcapstartof{table}
\sphinxthecaptionisattop
\sphinxcaption{LED Buttons}\label{\detokenize{controls:id3}}
\sphinxaftertopcaption
\begin{tabulary}{\linewidth}[t]{|T|T|T|}
\sphinxtoprule
\sphinxtableatstartofbodyhook
\sphinxAtStartPar
\sphinxstylestrong{Control Name}
&
\sphinxAtStartPar
\sphinxstylestrong{Object Name}
&
\sphinxAtStartPar
\sphinxstylestrong{Function}
\\
\sphinxhline
\sphinxAtStartPar
Save
&
\sphinxAtStartPar
save\_pb
&
\sphinxAtStartPar
NC Code Save Button
\\
\sphinxhline
\sphinxAtStartPar
Reload
&
\sphinxAtStartPar
reload\_pb
&
\sphinxAtStartPar
Reload NC Code from current file
\\
\sphinxhline
\sphinxAtStartPar
E Stop
&
\sphinxAtStartPar
estop\_pb
&
\sphinxAtStartPar
E Stop Toggle Button
\\
\sphinxhline
\sphinxAtStartPar
Power
&
\sphinxAtStartPar
power\_pb
&
\sphinxAtStartPar
Power Toggle Button
\\
\sphinxhline
\sphinxAtStartPar
Run
&
\sphinxAtStartPar
run\_pb
&
\sphinxAtStartPar
Runs a loaded NC file
\\
\sphinxhline
\sphinxAtStartPar
Pause
&
\sphinxAtStartPar
pause\_pb
&
\sphinxAtStartPar
Pauses a running NC file
\\
\sphinxhline
\sphinxAtStartPar
Manual Mode
&
\sphinxAtStartPar
manual\_mode\_pb
&
\sphinxAtStartPar
Puts the control into Manual Mode
\\
\sphinxhline
\sphinxAtStartPar
MDI Mode
&
\sphinxAtStartPar
mdi\_mode\_pb
&
\sphinxAtStartPar
Puts the control into MDI Mode
\\
\sphinxhline
\sphinxAtStartPar
Flood
&
\sphinxAtStartPar
flood\_pb
&
\sphinxAtStartPar
Turns on the flood coolant
\\
\sphinxhline
\sphinxAtStartPar
Mist
&
\sphinxAtStartPar
mist\_pb
&
\sphinxAtStartPar
Turns on the mist coolant
\\
\sphinxhline
\sphinxAtStartPar
Probe Enable
&
\sphinxAtStartPar
probing\_enable\_pb
&
\sphinxAtStartPar
Enables Probing and disables other controls
\\
\sphinxbottomrule
\end{tabulary}
\sphinxtableafterendhook\par
\sphinxattableend\end{savenotes}

\sphinxAtStartPar
Adding the Bool type Dynamic Property \sphinxtitleref{led\_indicator} to one of the above
control buttons will add the default LED to that button. Each control button can
have different options.


\begin{savenotes}\sphinxattablestart
\sphinxthistablewithglobalstyle
\centering
\sphinxcapstartof{table}
\sphinxthecaptionisattop
\sphinxcaption{LED Button Dynamic Properties}\label{\detokenize{controls:id4}}
\sphinxaftertopcaption
\begin{tabulary}{\linewidth}[t]{|T|T|T|}
\sphinxtoprule
\sphinxtableatstartofbodyhook
\sphinxAtStartPar
\sphinxstylestrong{Property Name}
&
\sphinxAtStartPar
\sphinxstylestrong{Type}
&
\sphinxAtStartPar
\sphinxstylestrong{Function}
\\
\sphinxhline
\sphinxAtStartPar
led\_indicator
&
\sphinxAtStartPar
Bool
&
\sphinxAtStartPar
Creates a LED
\\
\sphinxhline
\sphinxAtStartPar
led\_diameter
&
\sphinxAtStartPar
Int
&
\sphinxAtStartPar
Sets the Diameter of the LED in pixels
\\
\sphinxhline
\sphinxAtStartPar
led\_right\_offset
&
\sphinxAtStartPar
Int
&
\sphinxAtStartPar
Sets the offset from the right edge in pixels
\\
\sphinxhline
\sphinxAtStartPar
led\_top\_offset
&
\sphinxAtStartPar
Int
&
\sphinxAtStartPar
Sets the offset from the top edge in pixels
\\
\sphinxhline
\sphinxAtStartPar
led\_on\_color
&
\sphinxAtStartPar
Color
&
\sphinxAtStartPar
Sets the color of the LED when on
\\
\sphinxhline
\sphinxAtStartPar
led\_off\_color
&
\sphinxAtStartPar
Color
&
\sphinxAtStartPar
Sets the color of the LED when off
\\
\sphinxbottomrule
\end{tabulary}
\sphinxtableafterendhook\par
\sphinxattableend\end{savenotes}

\sphinxAtStartPar
To change the LED default options they can be set in the INI file.
See {\hyperref[\detokenize{ini:led-defaults}]{\sphinxcrossref{\DUrole{std,std-ref}{LED Defaults}}}}

\begin{sphinxadmonition}{tip}{Tip:}
\sphinxAtStartPar
A space after the button text gives more room for the LED
\end{sphinxadmonition}


\section{Coordinate System Controls}
\label{\detokenize{controls:coordinate-system-controls}}
\sphinxAtStartPar
A QPushButton can be used to clear the current coordinate system by using 0 as
the index or any one of the 9 coordinate systems with (1\sphinxhyphen{}9).

\sphinxAtStartPar
To clear the G92 coordinate system use 10 as the index.


\begin{savenotes}\sphinxattablestart
\sphinxthistablewithglobalstyle
\centering
\sphinxcapstartof{table}
\sphinxthecaptionisattop
\sphinxcaption{Coordinate System Buttons}\label{\detokenize{controls:id5}}
\sphinxaftertopcaption
\begin{tabulary}{\linewidth}[t]{|T|T|}
\sphinxtoprule
\sphinxtableatstartofbodyhook
\sphinxAtStartPar
\sphinxstylestrong{Control Function}
&
\sphinxAtStartPar
\sphinxstylestrong{Object Name}
\\
\sphinxhline
\sphinxAtStartPar
Clear Current G5x
&
\sphinxAtStartPar
clear\_coord\_0
\\
\sphinxhline
\sphinxAtStartPar
Clear G5x Coordinate System
&
\sphinxAtStartPar
clear\_coord\_(1\sphinxhyphen{}9)
\\
\sphinxhline
\sphinxAtStartPar
Clear G92 Coordinate System
&
\sphinxAtStartPar
clear\_coord\_10
\\
\sphinxbottomrule
\end{tabulary}
\sphinxtableafterendhook\par
\sphinxattableend\end{savenotes}


\section{Options}
\label{\detokenize{controls:options}}
\sphinxAtStartPar
The QPushButton options are toggle\sphinxhyphen{}type buttons; press to turn on, press again
to turn off. They are normal push buttons but Flex automatically makes them
\sphinxtitleref{checkable}.


\begin{savenotes}\sphinxattablestart
\sphinxthistablewithglobalstyle
\raggedright
\sphinxcapstartof{table}
\sphinxthecaptionisattop
\sphinxcaption{Options}\label{\detokenize{controls:id6}}
\sphinxaftertopcaption
\begin{tabulary}{\linewidth}[t]{|T|T|T|}
\sphinxtoprule
\sphinxtableatstartofbodyhook
\sphinxAtStartPar
\sphinxstylestrong{Function}
&
\sphinxAtStartPar
\sphinxstylestrong{Widget}
&
\sphinxAtStartPar
\sphinxstylestrong{Name}
\\
\sphinxhline
\sphinxAtStartPar
Flood Toggle
&
\sphinxAtStartPar
QPushButton
&
\sphinxAtStartPar
flood\_pb
\\
\sphinxhline
\sphinxAtStartPar
Mist Toggle
&
\sphinxAtStartPar
QPushButton
&
\sphinxAtStartPar
mist\_pb
\\
\sphinxhline
\sphinxAtStartPar
Optional Stop at M1
&
\sphinxAtStartPar
QPushButton
&
\sphinxAtStartPar
optional\_stop\_pb
\\
\sphinxhline
\sphinxAtStartPar
Block Delete line that starts with /
&
\sphinxAtStartPar
QPushButton
&
\sphinxAtStartPar
block\_delete\_pb
\\
\sphinxhline
\sphinxAtStartPar
Feed Override Enable/Disable
&
\sphinxAtStartPar
QPushButton
&
\sphinxAtStartPar
feed\_override\_pb
\\
\sphinxbottomrule
\end{tabulary}
\sphinxtableafterendhook\par
\sphinxattableend\end{savenotes}


\section{Axis Index}
\label{\detokenize{controls:axis-index}}
\begin{sphinxVerbatim}[commandchars=\\\{\}]
\PYG{n}{X} \PYG{l+m+mi}{0}
\PYG{n}{Y} \PYG{l+m+mi}{1}
\PYG{n}{Z} \PYG{l+m+mi}{2}
\PYG{n}{A} \PYG{l+m+mi}{3}
\PYG{n}{B} \PYG{l+m+mi}{4}
\PYG{n}{C} \PYG{l+m+mi}{5}
\PYG{n}{U} \PYG{l+m+mi}{6}
\PYG{n}{V} \PYG{l+m+mi}{7}
\PYG{n}{W} \PYG{l+m+mi}{8}
\end{sphinxVerbatim}


\section{Jog Controls}
\label{\detokenize{controls:jog-controls}}
\sphinxAtStartPar
\sphinxhref{https://youtu.be/ReVeEB5tEYM}{Jog Controls Tutorial}

\sphinxAtStartPar
Jogging requires a \sphinxtitleref{Jog Velocity Slider} and \sphinxtitleref{Jog Mode Selector}. If either
is not found, Jog Buttons will be disabled. This type of jog controls provides
a button for each axis and jog direction.


\begin{savenotes}\sphinxattablestart
\sphinxthistablewithglobalstyle
\raggedright
\sphinxcapstartof{table}
\sphinxthecaptionisattop
\sphinxcaption{Jog Widgets}\label{\detokenize{controls:id7}}
\sphinxaftertopcaption
\begin{tabulary}{\linewidth}[t]{|T|T|T|}
\sphinxtoprule
\sphinxtableatstartofbodyhook
\sphinxAtStartPar
\sphinxstylestrong{Function}
&
\sphinxAtStartPar
\sphinxstylestrong{Widget}
&
\sphinxAtStartPar
\sphinxstylestrong{Name}
\\
\sphinxhline
\sphinxAtStartPar
Jog Plus Axis (0\sphinxhyphen{}8)
&
\sphinxAtStartPar
QPushButton
&
\sphinxAtStartPar
jog\_plus\_pb\_(0\sphinxhyphen{}8)
\\
\sphinxhline
\sphinxAtStartPar
Jog Minus Axis (0\sphinxhyphen{}8)
&
\sphinxAtStartPar
QPushButton
&
\sphinxAtStartPar
jog\_minus\_pb\_(0\sphinxhyphen{}8)
\\
\sphinxhline
\sphinxAtStartPar
Jog Velocity Slider
&
\sphinxAtStartPar
QSlider
&
\sphinxAtStartPar
jog\_vel\_sl
\\
\sphinxhline
\sphinxAtStartPar
Jog Velocity Label
&
\sphinxAtStartPar
QLabel
&
\sphinxAtStartPar
jog\_vel\_lb
\\
\sphinxhline
\sphinxAtStartPar
Jog Mode Selector
&
\sphinxAtStartPar
QComboBox
&
\sphinxAtStartPar
jog\_modes\_cb
\\
\sphinxbottomrule
\end{tabulary}
\sphinxtableafterendhook\par
\sphinxattableend\end{savenotes}

\begin{sphinxadmonition}{note}{Note:}
\sphinxAtStartPar
Jog Plus/Minus buttons use the {\hyperref[\detokenize{controls:axis-index}]{\sphinxcrossref{Axis Index}}}. So \sphinxtitleref{Jog Y Plus} is
\sphinxtitleref{jog\_plus\_pb\_1}.
\end{sphinxadmonition}

\begin{sphinxadmonition}{note}{Note:}
\sphinxAtStartPar
\sphinxtitleref{Jog Mode Selector} reads the ini entry {[}DISPLAY{]} INCREMENTS and if
not found, only \sphinxtitleref{Continuous} will be an option.
\end{sphinxadmonition}


\section{Jog Selected Axis Controls}
\label{\detokenize{controls:jog-selected-axis-controls}}
\sphinxAtStartPar
To add Axis style jog controls where you select an axis then the plus/minus
buttons jog the selected axis add a QRadioButton for each axis and a QPushButton
for Plus and Minus. Axes are 0\sphinxhyphen{}8 for X, Y, Z, A, B, C, U, V, W.


\begin{savenotes}\sphinxattablestart
\sphinxthistablewithglobalstyle
\raggedright
\sphinxcapstartof{table}
\sphinxthecaptionisattop
\sphinxcaption{Jog Selected Widgets}\label{\detokenize{controls:id8}}
\sphinxaftertopcaption
\begin{tabulary}{\linewidth}[t]{|T|T|T|}
\sphinxtoprule
\sphinxtableatstartofbodyhook
\sphinxAtStartPar
\sphinxstylestrong{Function}
&
\sphinxAtStartPar
\sphinxstylestrong{Widget}
&
\sphinxAtStartPar
\sphinxstylestrong{Name}
\\
\sphinxhline
\sphinxAtStartPar
Axis Select (0\sphinxhyphen{}8)
&
\sphinxAtStartPar
QRadioButton
&
\sphinxAtStartPar
axis\_select\_(0\sphinxhyphen{}8)
\\
\sphinxhline
\sphinxAtStartPar
Jog Plus
&
\sphinxAtStartPar
QPushButton
&
\sphinxAtStartPar
jog\_selected\_plus
\\
\sphinxhline
\sphinxAtStartPar
Jog Minus
&
\sphinxAtStartPar
QPushButton
&
\sphinxAtStartPar
jog\_selected\_minus
\\
\sphinxbottomrule
\end{tabulary}
\sphinxtableafterendhook\par
\sphinxattableend\end{savenotes}


\section{Overrides}
\label{\detokenize{controls:overrides}}
\sphinxAtStartPar
\sphinxhref{https://youtu.be/taAtYf3ebDE}{Overrides Tutorial}

\sphinxAtStartPar
A QSlider is used to control the following functions and the corresponding
label shows the value of the slider:


\begin{savenotes}\sphinxattablestart
\sphinxthistablewithglobalstyle
\raggedright
\sphinxcapstartof{table}
\sphinxthecaptionisattop
\sphinxcaption{Overrides}\label{\detokenize{controls:id9}}
\sphinxaftertopcaption
\begin{tabulary}{\linewidth}[t]{|T|T|T|}
\sphinxtoprule
\sphinxtableatstartofbodyhook
\sphinxAtStartPar
\sphinxstylestrong{Function}
&
\sphinxAtStartPar
\sphinxstylestrong{Widget}
&
\sphinxAtStartPar
\sphinxstylestrong{Object Name}
\\
\sphinxhline
\sphinxAtStartPar
Feed Override Slider
&
\sphinxAtStartPar
QSlider
&
\sphinxAtStartPar
feed\_override\_sl
\\
\sphinxhline
\sphinxAtStartPar
Feed Override Percent
&
\sphinxAtStartPar
QLabel
&
\sphinxAtStartPar
feed\_override\_lb
\\
\sphinxhline
\sphinxAtStartPar
Rapid Override Slider
&
\sphinxAtStartPar
QSlider
&
\sphinxAtStartPar
rapid\_override\_sl
\\
\sphinxhline
\sphinxAtStartPar
Rapid Override Percent
&
\sphinxAtStartPar
QLabel
&
\sphinxAtStartPar
rapid\_override\_lb
\\
\sphinxhline
\sphinxAtStartPar
Spindle Override Slider
&
\sphinxAtStartPar
QSlider
&
\sphinxAtStartPar
spindle\_override\_sl
\\
\sphinxhline
\sphinxAtStartPar
Spindle Override Percent
&
\sphinxAtStartPar
QLabel
&
\sphinxAtStartPar
spindle\_override\_0\_lb
\\
\sphinxhline
\sphinxAtStartPar
Maximum Velocity
&
\sphinxAtStartPar
QSlider
&
\sphinxAtStartPar
max\_vel\_sl
\\
\sphinxhline
\sphinxAtStartPar
Override Limits
&
\sphinxAtStartPar
QCheckBox
&
\sphinxAtStartPar
override\_limits\_cb
\\
\sphinxbottomrule
\end{tabulary}
\sphinxtableafterendhook\par
\sphinxattableend\end{savenotes}

\sphinxAtStartPar
The following settings can be used in the DISPLAY section of the ini file:

\begin{sphinxVerbatim}[commandchars=\\\{\}]
\PYG{n}{Feed} \PYG{n}{Override} \PYG{n}{maximum}             \PYG{n}{MAX\PYGZus{}FEED\PYGZus{}OVERRIDE}
\PYG{n}{Spindle} \PYG{n}{Override} \PYG{n}{maximum}          \PYG{n}{MAX\PYGZus{}SPINDLE\PYGZus{}OVERRIDE}
\end{sphinxVerbatim}


\section{Override Presets}
\label{\detokenize{controls:override-presets}}
\sphinxAtStartPar
Feed, Rapid and Spindle overrides can have a preset button(s) for different
preset amounts. Replace the nnn with the percent of override you want that
button to use.


\begin{savenotes}\sphinxattablestart
\sphinxthistablewithglobalstyle
\raggedright
\sphinxcapstartof{table}
\sphinxthecaptionisattop
\sphinxcaption{Override Presets}\label{\detokenize{controls:id10}}
\sphinxaftertopcaption
\begin{tabulary}{\linewidth}[t]{|T|T|T|}
\sphinxtoprule
\sphinxtableatstartofbodyhook
\sphinxAtStartPar
\sphinxstylestrong{Function}
&
\sphinxAtStartPar
\sphinxstylestrong{Widget}
&
\sphinxAtStartPar
\sphinxstylestrong{Object Name}
\\
\sphinxhline
\sphinxAtStartPar
Feed Override Preset
&
\sphinxAtStartPar
QPushButton
&
\sphinxAtStartPar
feed\_percent\_nnn
\\
\sphinxhline
\sphinxAtStartPar
Rapid Override Preset
&
\sphinxAtStartPar
QPushButton
&
\sphinxAtStartPar
rapid\_percent\_nnn
\\
\sphinxhline
\sphinxAtStartPar
Spindle Override Preset
&
\sphinxAtStartPar
QPushButton
&
\sphinxAtStartPar
spindle\_percent\_nnn
\\
\sphinxbottomrule
\end{tabulary}
\sphinxtableafterendhook\par
\sphinxattableend\end{savenotes}

\begin{sphinxadmonition}{note}{Note:}
\sphinxAtStartPar
The maximum override for Rapid is 100
\end{sphinxadmonition}


\section{Stacked Widget}
\label{\detokenize{controls:stacked-widget}}
\sphinxAtStartPar
To change to a specific page on a QStackedWidget add a QPushButton on each page
and set a couple of Dynamic Properties. See {\hyperref[\detokenize{property::doc}]{\sphinxcrossref{\DUrole{doc}{Dynamic Properties}}}}


\begin{savenotes}\sphinxattablestart
\sphinxthistablewithglobalstyle
\raggedright
\sphinxcapstartof{table}
\sphinxthecaptionisattop
\sphinxcaption{Stacked Widget Change Page}\label{\detokenize{controls:id11}}
\sphinxaftertopcaption
\begin{tabulary}{\linewidth}[t]{|T|T|}
\sphinxtoprule
\sphinxtableatstartofbodyhook
\sphinxAtStartPar
\sphinxstylestrong{Dynamic Property Name}
&
\sphinxAtStartPar
\sphinxstylestrong{Value}
\\
\sphinxhline
\sphinxAtStartPar
change\_page
&
\sphinxAtStartPar
QStackedWidget Object Name
\\
\sphinxhline
\sphinxAtStartPar
index
&
\sphinxAtStartPar
index of page to change to
\\
\sphinxbottomrule
\end{tabulary}
\sphinxtableafterendhook\par
\sphinxattableend\end{savenotes}

\noindent{\hspace*{\fill}\sphinxincludegraphics{{stacked-01}.png}\hspace*{\fill}}

\sphinxAtStartPar
To create a Next Page and Previous Page buttons for a QStackedWidget add two
QPushButtons with the following Dynamic Properties. See {\hyperref[\detokenize{property::doc}]{\sphinxcrossref{\DUrole{doc}{Dynamic Properties}}}}


\begin{savenotes}\sphinxattablestart
\sphinxthistablewithglobalstyle
\raggedright
\sphinxcapstartof{table}
\sphinxthecaptionisattop
\sphinxcaption{Stacked Widget Next/Previous Page}\label{\detokenize{controls:id12}}
\sphinxaftertopcaption
\begin{tabulary}{\linewidth}[t]{|T|T|T|}
\sphinxtoprule
\sphinxtableatstartofbodyhook
\sphinxAtStartPar
\sphinxstylestrong{Button Function}
&
\sphinxAtStartPar
\sphinxstylestrong{Dynamic Property Name}
&
\sphinxAtStartPar
\sphinxstylestrong{Value}
\\
\sphinxhline
\sphinxAtStartPar
Next Page
&
\sphinxAtStartPar
next\_page
&
\sphinxAtStartPar
QStackedWidget Object Name
\\
\sphinxhline
\sphinxAtStartPar
Previous Page
&
\sphinxAtStartPar
previous\_page
&
\sphinxAtStartPar
QStackedWidget Object Name
\\
\sphinxbottomrule
\end{tabulary}
\sphinxtableafterendhook\par
\sphinxattableend\end{savenotes}

\begin{sphinxadmonition}{note}{Note:}
\sphinxAtStartPar
The Forward and Backward Buttons should not be in the QStackedWidget
\end{sphinxadmonition}


\section{File Load Buttons}
\label{\detokenize{controls:file-load-buttons}}
\sphinxAtStartPar
To create a QPushButton to load a specific file add two Dynamic Properties.


\begin{savenotes}\sphinxattablestart
\sphinxthistablewithglobalstyle
\raggedright
\sphinxcapstartof{table}
\sphinxthecaptionisattop
\sphinxcaption{File Load Buttons}\label{\detokenize{controls:id13}}
\sphinxaftertopcaption
\begin{tabulary}{\linewidth}[t]{|T|T|}
\sphinxtoprule
\sphinxtableatstartofbodyhook
\sphinxAtStartPar
\sphinxstylestrong{Dynamic Property Name}
&
\sphinxAtStartPar
\sphinxstylestrong{Value}
\\
\sphinxhline
\sphinxAtStartPar
function
&
\sphinxAtStartPar
load\_file
\\
\sphinxhline
\sphinxAtStartPar
filename
&
\sphinxAtStartPar
file to load
\\
\sphinxbottomrule
\end{tabulary}
\sphinxtableafterendhook\par
\sphinxattableend\end{savenotes}

\sphinxAtStartPar
File Name in the PROGRAM\_PREFIX path.

\noindent{\hspace*{\fill}\sphinxincludegraphics{{controls-02}.png}\hspace*{\fill}}

\sphinxAtStartPar
File Name with Full Path

\noindent{\hspace*{\fill}\sphinxincludegraphics{{controls-03}.png}\hspace*{\fill}}

\sphinxAtStartPar
File Name in the Configuration Directory

\noindent{\hspace*{\fill}\sphinxincludegraphics{{controls-04}.png}\hspace*{\fill}}

\sphinxAtStartPar
File Name relative to the Users Home Directory

\noindent{\hspace*{\fill}\sphinxincludegraphics{{controls-05}.png}\hspace*{\fill}}

\sphinxAtStartPar
The file name can be just the name and extension if it’s in the PROGRAM\_PREFIX
path. Or it can be any valid path and file name.

\sphinxAtStartPar
File Name Examples

\begin{sphinxVerbatim}[commandchars=\\\{\}]
\PYG{n}{A} \PYG{n}{file} \PYG{o+ow}{in} \PYG{n}{the} \PYG{n}{PROGRAM\PYGZus{}PREFIX} \PYG{n}{path}
\PYG{n}{somefile}\PYG{o}{.}\PYG{n}{ngc}

\PYG{n}{A} \PYG{n}{file} \PYG{o+ow}{in} \PYG{n}{the} \PYG{n}{configuration} \PYG{n}{directory}
\PYG{o}{.}\PYG{o}{/}\PYG{n}{anotherfile}\PYG{o}{.}\PYG{n}{ngc}

\PYG{n}{A} \PYG{n}{file} \PYG{n}{up} \PYG{n}{one} \PYG{n}{directory} \PYG{k+kn}{from} \PYG{n+nn}{the} \PYG{n}{configuration} \PYG{n}{directory}
\PYG{o}{.}\PYG{o}{.}\PYG{o}{/}\PYG{n}{up\PYGZus{}one}\PYG{o}{.}\PYG{n}{ngc}

\PYG{n}{A} \PYG{n}{file} \PYG{n}{relative} \PYG{n}{to} \PYG{n}{the} \PYG{n}{users} \PYG{n}{home} \PYG{n}{directory}
\PYG{o}{/}\PYG{n}{home}\PYG{o}{/}\PYG{n}{fred}\PYG{o}{/}\PYG{n}{linuxcnc}\PYG{o}{/}\PYG{n}{my\PYGZus{}files}\PYG{o}{/}\PYG{n}{afile}\PYG{o}{.}\PYG{n}{ngc}
\PYG{n}{could} \PYG{n}{be}
\PYG{o}{\PYGZti{}}\PYG{o}{/}\PYG{n}{linuxcnc}\PYG{o}{/}\PYG{n}{my\PYGZus{}files}\PYG{o}{/}\PYG{n}{afile}\PYG{o}{.}\PYG{n}{ngc}
\end{sphinxVerbatim}

\begin{sphinxadmonition}{warning}{Warning:}
\sphinxAtStartPar
The file must be in the directory specified by the INI entry
PROGRAM\_PREFIX in the {[}DISPLAY{]} section or have a valid path.
\end{sphinxadmonition}

\sphinxAtStartPar
This is useful for probe routine buttons to load the nc code so the
path can be viewed in the plotter and for programs that are ran frequently.

\begin{sphinxadmonition}{note}{Note:}
\sphinxAtStartPar
The file is not added to the Recent Files list.
\end{sphinxadmonition}

\sphinxstepscope


\chapter{Plotter}
\label{\detokenize{plotter:plotter}}\label{\detokenize{plotter::doc}}
\sphinxAtStartPar
\sphinxhref{https://youtu.be/\_f9sQWPe\_XI}{Plotter Tutorial}

\sphinxAtStartPar
To add a live G\sphinxhyphen{}code plotter, add a QWidget or QFrame and name it \sphinxtitleref{plot\_widget}.


\section{Controls}
\label{\detokenize{plotter:controls}}
\sphinxAtStartPar
If you’re using a touch\sphinxhyphen{}screen, add pan, zoom, and rotate controls for the
plotter


\begin{savenotes}\sphinxattablestart
\sphinxthistablewithglobalstyle
\centering
\sphinxcapstartof{table}
\sphinxthecaptionisattop
\sphinxcaption{Display Controls}\label{\detokenize{plotter:id1}}
\sphinxaftertopcaption
\begin{tabular}[t]{|\X{40}{120}|\X{40}{120}|\X{40}{120}|}
\sphinxtoprule
\sphinxtableatstartofbodyhook
\sphinxAtStartPar
\sphinxstylestrong{Control}
&
\sphinxAtStartPar
\sphinxstylestrong{Widget}
&
\sphinxAtStartPar
\sphinxstylestrong{Object Name}
\\
\sphinxhline
\sphinxAtStartPar
Rotate View Up
&
\sphinxAtStartPar
QPushButton
&
\sphinxAtStartPar
view\_rotate\_up\_pb
\\
\sphinxhline
\sphinxAtStartPar
Rotate View Down
&
\sphinxAtStartPar
QPushButton
&
\sphinxAtStartPar
view\_rotate\_down\_pb
\\
\sphinxhline
\sphinxAtStartPar
Rotate View Left
&
\sphinxAtStartPar
QPushButton
&
\sphinxAtStartPar
view\_rotate\_left\_pb
\\
\sphinxhline
\sphinxAtStartPar
Rotate View Right
&
\sphinxAtStartPar
QPushButton
&
\sphinxAtStartPar
view\_rotate\_right\_pb
\\
\sphinxhline
\sphinxAtStartPar
Pan View Up
&
\sphinxAtStartPar
QPushButton
&
\sphinxAtStartPar
view\_pan\_up\_pb
\\
\sphinxhline
\sphinxAtStartPar
Pan View Down
&
\sphinxAtStartPar
QPushButton
&
\sphinxAtStartPar
view\_pan\_down\_pb
\\
\sphinxhline
\sphinxAtStartPar
Pan View Left
&
\sphinxAtStartPar
QPushButton
&
\sphinxAtStartPar
view\_pan\_left\_pb
\\
\sphinxhline
\sphinxAtStartPar
Pan View Right
&
\sphinxAtStartPar
QPushButton
&
\sphinxAtStartPar
view\_pan\_right\_pb
\\
\sphinxhline
\sphinxAtStartPar
Zoom View In
&
\sphinxAtStartPar
QPushButton
&
\sphinxAtStartPar
view\_zoom\_in\_pb
\\
\sphinxhline
\sphinxAtStartPar
Zoom View Out
&
\sphinxAtStartPar
QPushButton
&
\sphinxAtStartPar
view\_zoom\_out\_pb
\\
\sphinxbottomrule
\end{tabular}
\sphinxtableafterendhook\par
\sphinxattableend\end{savenotes}

\sphinxAtStartPar
The following controls set\sphinxhyphen{}predefined views


\begin{savenotes}\sphinxattablestart
\sphinxthistablewithglobalstyle
\centering
\sphinxcapstartof{table}
\sphinxthecaptionisattop
\sphinxcaption{Display Views}\label{\detokenize{plotter:id2}}
\sphinxaftertopcaption
\begin{tabular}[t]{|\X{40}{120}|\X{40}{120}|\X{40}{120}|}
\sphinxtoprule
\sphinxtableatstartofbodyhook
\sphinxAtStartPar
\sphinxstylestrong{Control}
&
\sphinxAtStartPar
\sphinxstylestrong{Widget}
&
\sphinxAtStartPar
\sphinxstylestrong{Object Name}
\\
\sphinxhline
\sphinxAtStartPar
View Perspective
&
\sphinxAtStartPar
QPushButton
&
\sphinxAtStartPar
view\_p\_pb
\\
\sphinxhline
\sphinxAtStartPar
View X
&
\sphinxAtStartPar
QPushButton
&
\sphinxAtStartPar
view\_x\_pb
\\
\sphinxhline
\sphinxAtStartPar
View Y
&
\sphinxAtStartPar
QPushButton
&
\sphinxAtStartPar
view\_y\_pb
\\
\sphinxhline
\sphinxAtStartPar
View Y2
&
\sphinxAtStartPar
QPushButton
&
\sphinxAtStartPar
view\_y2\_pb
\\
\sphinxhline
\sphinxAtStartPar
View Z
&
\sphinxAtStartPar
QPushButton
&
\sphinxAtStartPar
view\_z\_pb
\\
\sphinxhline
\sphinxAtStartPar
View Z2
&
\sphinxAtStartPar
QPushButton
&
\sphinxAtStartPar
view\_z2\_pb
\\
\sphinxbottomrule
\end{tabular}
\sphinxtableafterendhook\par
\sphinxattableend\end{savenotes}

\sphinxAtStartPar
To clear the Live plot


\begin{savenotes}\sphinxattablestart
\sphinxthistablewithglobalstyle
\centering
\sphinxcapstartof{table}
\sphinxthecaptionisattop
\sphinxcaption{Display Functions}\label{\detokenize{plotter:id3}}
\sphinxaftertopcaption
\begin{tabular}[t]{|\X{40}{120}|\X{40}{120}|\X{40}{120}|}
\sphinxtoprule
\sphinxtableatstartofbodyhook
\sphinxAtStartPar
Control
&
\sphinxAtStartPar
Widget
&
\sphinxAtStartPar
Name
\\
\sphinxhline
\sphinxAtStartPar
Clear Live Plot
&
\sphinxAtStartPar
QPushButton
&
\sphinxAtStartPar
view\_clear\_pb
\\
\sphinxbottomrule
\end{tabular}
\sphinxtableafterendhook\par
\sphinxattableend\end{savenotes}


\section{Display}
\label{\detokenize{plotter:display}}
\sphinxAtStartPar
The DRO overlaid onto the plotter can be customized by turning on or off
various features. Use either a QCheckbox or a QPushButton to toggle these


\begin{savenotes}\sphinxattablestart
\sphinxthistablewithglobalstyle
\centering
\sphinxcapstartof{table}
\sphinxthecaptionisattop
\sphinxcaption{Display Checkbox Options}\label{\detokenize{plotter:id4}}
\sphinxaftertopcaption
\begin{tabular}[t]{|\X{40}{120}|\X{40}{120}|\X{40}{120}|}
\sphinxtoprule
\sphinxtableatstartofbodyhook
\sphinxAtStartPar
\sphinxstylestrong{Function}
&
\sphinxAtStartPar
\sphinxstylestrong{Widget}
&
\sphinxAtStartPar
\sphinxstylestrong{Object Name}
\\
\sphinxhline
\sphinxAtStartPar
View DRO
&
\sphinxAtStartPar
QCheckBox
&
\sphinxAtStartPar
view\_dro\_cb
\\
\sphinxhline
\sphinxAtStartPar
View Machine Limits
&
\sphinxAtStartPar
QCheckBox
&
\sphinxAtStartPar
view\_limits\_cb
\\
\sphinxhline
\sphinxAtStartPar
View Extents Option
&
\sphinxAtStartPar
QCheckBox
&
\sphinxAtStartPar
view\_extents\_option\_cb
\\
\sphinxhline
\sphinxAtStartPar
View Live Plot
&
\sphinxAtStartPar
QCheckBox
&
\sphinxAtStartPar
view\_live\_plot\_cb
\\
\sphinxhline
\sphinxAtStartPar
View Velocity
&
\sphinxAtStartPar
QCheckBox
&
\sphinxAtStartPar
view\_velocity\_cb
\\
\sphinxhline
\sphinxAtStartPar
Use Metric Units
&
\sphinxAtStartPar
QCheckBox
&
\sphinxAtStartPar
view\_metric\_units\_cb
\\
\sphinxhline
\sphinxAtStartPar
View Program
&
\sphinxAtStartPar
QCheckBox
&
\sphinxAtStartPar
view\_program\_cb
\\
\sphinxhline
\sphinxAtStartPar
View Rapids
&
\sphinxAtStartPar
QCheckBox
&
\sphinxAtStartPar
view\_rapids\_cb
\\
\sphinxhline
\sphinxAtStartPar
View Tool
&
\sphinxAtStartPar
QCheckBox
&
\sphinxAtStartPar
view\_tool\_cb
\\
\sphinxhline
\sphinxAtStartPar
View Lathe Radius
&
\sphinxAtStartPar
QCheckBox
&
\sphinxAtStartPar
view\_lathe\_radius\_cb
\\
\sphinxhline
\sphinxAtStartPar
View Distance to Go
&
\sphinxAtStartPar
QCheckBox
&
\sphinxAtStartPar
view\_dtg\_cb
\\
\sphinxhline
\sphinxAtStartPar
View Offsets
&
\sphinxAtStartPar
QCheckBox
&
\sphinxAtStartPar
view\_offsets\_cb
\\
\sphinxhline
\sphinxAtStartPar
View Overlay
&
\sphinxAtStartPar
QCheckBox
&
\sphinxAtStartPar
view\_overlay\_cb
\\
\sphinxbottomrule
\end{tabular}
\sphinxtableafterendhook\par
\sphinxattableend\end{savenotes}


\begin{savenotes}\sphinxattablestart
\sphinxthistablewithglobalstyle
\centering
\sphinxcapstartof{table}
\sphinxthecaptionisattop
\sphinxcaption{Display PushButton Options}\label{\detokenize{plotter:id5}}
\sphinxaftertopcaption
\begin{tabular}[t]{|\X{40}{120}|\X{40}{120}|\X{40}{120}|}
\sphinxtoprule
\sphinxtableatstartofbodyhook
\sphinxAtStartPar
\sphinxstylestrong{Function}
&
\sphinxAtStartPar
\sphinxstylestrong{Widget}
&
\sphinxAtStartPar
\sphinxstylestrong{Object Name}
\\
\sphinxhline
\sphinxAtStartPar
View DRO
&
\sphinxAtStartPar
QPushButton
&
\sphinxAtStartPar
view\_dro\_pb
\\
\sphinxhline
\sphinxAtStartPar
View Machine Limits
&
\sphinxAtStartPar
QPushButton
&
\sphinxAtStartPar
view\_limits\_pb
\\
\sphinxhline
\sphinxAtStartPar
View Extents Option
&
\sphinxAtStartPar
QPushButton
&
\sphinxAtStartPar
view\_extents\_option\_pb
\\
\sphinxhline
\sphinxAtStartPar
View Live Plot
&
\sphinxAtStartPar
QPushButton
&
\sphinxAtStartPar
view\_live\_plot\_pb
\\
\sphinxhline
\sphinxAtStartPar
View Velocity
&
\sphinxAtStartPar
QPushButton
&
\sphinxAtStartPar
view\_velocity\_pb
\\
\sphinxhline
\sphinxAtStartPar
Use Metric Units
&
\sphinxAtStartPar
QPushButton
&
\sphinxAtStartPar
view\_metric\_units\_pb
\\
\sphinxhline
\sphinxAtStartPar
View Program
&
\sphinxAtStartPar
QPushButton
&
\sphinxAtStartPar
view\_program\_pb
\\
\sphinxhline
\sphinxAtStartPar
View Rapids
&
\sphinxAtStartPar
QPushButton
&
\sphinxAtStartPar
view\_rapids\_pb
\\
\sphinxhline
\sphinxAtStartPar
View Tool
&
\sphinxAtStartPar
QPushButton
&
\sphinxAtStartPar
view\_tool\_pb
\\
\sphinxhline
\sphinxAtStartPar
View Lathe Radius
&
\sphinxAtStartPar
QPushButton
&
\sphinxAtStartPar
view\_lathe\_radius\_pb
\\
\sphinxhline
\sphinxAtStartPar
View Distance to Go
&
\sphinxAtStartPar
QPushButton
&
\sphinxAtStartPar
view\_dtg\_pb
\\
\sphinxhline
\sphinxAtStartPar
View Offsets
&
\sphinxAtStartPar
QPushButton
&
\sphinxAtStartPar
view\_offsets\_pb
\\
\sphinxhline
\sphinxAtStartPar
View Overlay
&
\sphinxAtStartPar
QPushButton
&
\sphinxAtStartPar
view\_overlay\_pb
\\
\sphinxbottomrule
\end{tabular}
\sphinxtableafterendhook\par
\sphinxattableend\end{savenotes}

\begin{sphinxadmonition}{note}{Note:}
\sphinxAtStartPar
Don’t set the checked property to checked in Qt Designer as this is
already handled in the code. Once you check an option it is remembered.
\end{sphinxadmonition}


\section{Menu}
\label{\detokenize{plotter:menu}}
\sphinxAtStartPar
The following menu items can set display options. \sphinxtitleref{Menu Name} is what you
type when creating the Menu, then press enter. All the items are checkbox type
menu items that stay coordinated with the checkbox of the same option.


\begin{savenotes}\sphinxattablestart
\sphinxthistablewithglobalstyle
\centering
\sphinxcapstartof{table}
\sphinxthecaptionisattop
\sphinxcaption{Plot Menu Options}\label{\detokenize{plotter:id6}}
\sphinxaftertopcaption
\begin{tabular}[t]{|\X{40}{120}|\X{40}{120}|\X{40}{120}|}
\sphinxtoprule
\sphinxtableatstartofbodyhook
\sphinxAtStartPar
\sphinxstylestrong{Function}
&
\sphinxAtStartPar
\sphinxstylestrong{Menu Name}
&
\sphinxAtStartPar
\sphinxstylestrong{Object Name}
\\
\sphinxhline
\sphinxAtStartPar
View DRO
&
\sphinxAtStartPar
DRO
&
\sphinxAtStartPar
actionDRO
\\
\sphinxhline
\sphinxAtStartPar
View Machine Limits
&
\sphinxAtStartPar
Limits
&
\sphinxAtStartPar
actionLimits
\\
\sphinxhline
\sphinxAtStartPar
View Extents Option
&
\sphinxAtStartPar
Extents Option
&
\sphinxAtStartPar
actionExtents\_Option
\\
\sphinxhline
\sphinxAtStartPar
View Live Plot
&
\sphinxAtStartPar
Live Plot
&
\sphinxAtStartPar
actionLive\_Plot
\\
\sphinxhline
\sphinxAtStartPar
View Velocity
&
\sphinxAtStartPar
Velocity
&
\sphinxAtStartPar
actionVelocity
\\
\sphinxhline
\sphinxAtStartPar
Use Metric Units
&
\sphinxAtStartPar
Metric Units
&
\sphinxAtStartPar
actionMetric\_Units
\\
\sphinxhline
\sphinxAtStartPar
View Program
&
\sphinxAtStartPar
Program
&
\sphinxAtStartPar
actionProgram
\\
\sphinxhline
\sphinxAtStartPar
View Rapids
&
\sphinxAtStartPar
Rapids
&
\sphinxAtStartPar
actionRapids
\\
\sphinxhline
\sphinxAtStartPar
View Tool
&
\sphinxAtStartPar
Tool
&
\sphinxAtStartPar
actionTool
\\
\sphinxhline
\sphinxAtStartPar
View Lathe Radius
&
\sphinxAtStartPar
Lathe Radius
&
\sphinxAtStartPar
actionLathe\_Radius
\\
\sphinxhline
\sphinxAtStartPar
View Distance to Go
&
\sphinxAtStartPar
DTG
&
\sphinxAtStartPar
actionDTG
\\
\sphinxhline
\sphinxAtStartPar
View Offsets
&
\sphinxAtStartPar
Offsets
&
\sphinxAtStartPar
actionOffsets
\\
\sphinxhline
\sphinxAtStartPar
View Overlay
&
\sphinxAtStartPar
Overlay
&
\sphinxAtStartPar
actionOverlay
\\
\sphinxbottomrule
\end{tabular}
\sphinxtableafterendhook\par
\sphinxattableend\end{savenotes}

\begin{sphinxadmonition}{note}{Note:}
\sphinxAtStartPar
Once a view selection has been set, Flex GUI remembers it.
\end{sphinxadmonition}

\sphinxAtStartPar
The live plot can be cleared from the menu with this menu item.


\begin{savenotes}\sphinxattablestart
\sphinxthistablewithglobalstyle
\centering
\sphinxcapstartof{table}
\sphinxthecaptionisattop
\sphinxcaption{Plot Menu}\label{\detokenize{plotter:id7}}
\sphinxaftertopcaption
\begin{tabular}[t]{|\X{40}{120}|\X{40}{120}|\X{40}{120}|}
\sphinxtoprule
\sphinxtableatstartofbodyhook
\sphinxAtStartPar
\sphinxstylestrong{Function}
&
\sphinxAtStartPar
\sphinxstylestrong{Menu Name}
&
\sphinxAtStartPar
\sphinxstylestrong{Object Name}
\\
\sphinxhline
\sphinxAtStartPar
Clear Live Plot
&
\sphinxAtStartPar
Clear Live Plot
&
\sphinxAtStartPar
actionClear\_Live\_Plot
\\
\sphinxbottomrule
\end{tabular}
\sphinxtableafterendhook\par
\sphinxattableend\end{savenotes}


\section{DRO}
\label{\detokenize{plotter:dro}}
\sphinxAtStartPar
The font size can be set in the ini file by adding in the {[}DISPLAY{]} section
DRO\_FONT\_SIZE = n where n is an integer. The default size is 12.

\sphinxstepscope


\chapter{Manual Data Input (MDI)}
\label{\detokenize{mdi:manual-data-input-mdi}}\label{\detokenize{mdi::doc}}
\sphinxAtStartPar
\sphinxhref{https://youtu.be/fHkyWxfZiKs}{MDI Tutorial}


\section{MDI Interface}
\label{\detokenize{mdi:mdi-interface}}
\sphinxAtStartPar
The MDI Interface uses a QLineEdit named \sphinxtitleref{mdi\_command\_le} to enter commands.

\sphinxAtStartPar
For touch screens there are two options a \sphinxtitleref{NC Popup} is a touch screen that has
G and M words and a number keypad or a \sphinxtitleref{Keyboard Popup} has a full keyboard.

\sphinxAtStartPar
To enable a popup add a Dynamic string type Property to the \sphinxtitleref{mdi\_command\_le}
QLineEdit and name it \sphinxtitleref{input} and set the value to either \sphinxtitleref{nccode} or
\sphinxtitleref{keyboard}.

\sphinxAtStartPar
Dynamic Property

\noindent{\hspace*{\fill}\sphinxincludegraphics{{dynamic-property-01}.png}\hspace*{\fill}}

\sphinxAtStartPar
Setting the value

\noindent{\hspace*{\fill}\sphinxincludegraphics{{dynamic-property-02}.png}\hspace*{\fill}}

\sphinxAtStartPar
NC code popup window

\noindent{\hspace*{\fill}\sphinxincludegraphics{{nccode-popup}.png}\hspace*{\fill}}

\sphinxAtStartPar
Keyboard popup window

\noindent{\hspace*{\fill}\sphinxincludegraphics{{keyboard-popup}.png}\hspace*{\fill}}


\section{MDI History}
\label{\detokenize{mdi:mdi-history}}
\sphinxAtStartPar
MDI history uses a QListWidget named \sphinxtitleref{mdi\_history\_lw} to display the MDI
history. You can click on a line in the history display to copy the command to
the MDI Interface, ready for running.

\noindent{\hspace*{\fill}\sphinxincludegraphics{{mdi-history}.png}\hspace*{\fill}}

\sphinxAtStartPar
The MDI history is kept in a file named \sphinxtitleref{mdi\_history.txt} in the configuration
directory.


\section{MDI Controls}
\label{\detokenize{mdi:mdi-controls}}
\sphinxAtStartPar
The following QPushButtons can be used to execute, copy, and clear MDI command
history


\begin{savenotes}\sphinxattablestart
\sphinxthistablewithglobalstyle
\centering
\sphinxcapstartof{table}
\sphinxthecaptionisattop
\sphinxcaption{MDI Push Buttons}\label{\detokenize{mdi:id1}}
\sphinxaftertopcaption
\begin{tabulary}{\linewidth}[t]{|T|T|T|}
\sphinxtoprule
\sphinxtableatstartofbodyhook
\sphinxAtStartPar
\sphinxstylestrong{Function}
&
\sphinxAtStartPar
\sphinxstylestrong{Widget}
&
\sphinxAtStartPar
\sphinxstylestrong{Object Name}
\\
\sphinxhline
\sphinxAtStartPar
Run MDI Command
&
\sphinxAtStartPar
QPushButton
&
\sphinxAtStartPar
run\_mdi\_pb
\\
\sphinxhline
\sphinxAtStartPar
Copy the MDI History to the Clipboard
&
\sphinxAtStartPar
QPushButton
&
\sphinxAtStartPar
copy\_mdi\_history\_pb
\\
\sphinxhline
\sphinxAtStartPar
Save the MDI History to a file
&
\sphinxAtStartPar
QPushButton
&
\sphinxAtStartPar
save\_mdi\_history\_pb
\\
\sphinxhline
\sphinxAtStartPar
Clear the MDI History
&
\sphinxAtStartPar
QPushButton
&
\sphinxAtStartPar
clear\_mdi\_history\_pb
\\
\sphinxbottomrule
\end{tabulary}
\sphinxtableafterendhook\par
\sphinxattableend\end{savenotes}


\section{MDI Button}
\label{\detokenize{mdi:mdi-button}}\label{\detokenize{mdi:mdibuttontag}}
\sphinxAtStartPar
MDI buttons execute a MDI command when the button is pressed. These are
created by adding two dynamic properties called \sphinxtitleref{function} and \sphinxtitleref{command} to a
QPushButton.

\begin{sphinxadmonition}{note}{Note:}
\sphinxAtStartPar
If the \sphinxtitleref{command} property is not found, the button will not be
enabled!
\end{sphinxadmonition}

\sphinxAtStartPar
Select the button then create a dynamic property by pressing the green plus
sign in the Property Editor

\noindent{\hspace*{\fill}\sphinxincludegraphics{{mdi-01}.png}\hspace*{\fill}}

\sphinxAtStartPar
Then select \sphinxtitleref{string}:

\noindent{\hspace*{\fill}\sphinxincludegraphics{{mdi-02}.png}\hspace*{\fill}}

\sphinxAtStartPar
Name the property \sphinxtitleref{function} and click OK

\noindent{\hspace*{\fill}\sphinxincludegraphics{{mdi-03}.png}\hspace*{\fill}}

\sphinxAtStartPar
Set the value of the property to \sphinxtitleref{mdi}

\noindent{\hspace*{\fill}\sphinxincludegraphics{{mdi-04}.png}\hspace*{\fill}}

\sphinxAtStartPar
Add a property called \sphinxtitleref{command}

\noindent{\hspace*{\fill}\sphinxincludegraphics{{mdi-05}.png}\hspace*{\fill}}

\sphinxAtStartPar
Set the value of the property to your valid MDI command

\noindent{\hspace*{\fill}\sphinxincludegraphics{{mdi-06}.png}\hspace*{\fill}}

\sphinxstepscope


\chapter{Spindle}
\label{\detokenize{spindle:spindle}}\label{\detokenize{spindle::doc}}
\sphinxAtStartPar
\sphinxhref{https://youtu.be/37Sh-lieq9Y}{Video Tutorial}


\section{Spindle Status}
\label{\detokenize{spindle:spindle-status}}
\sphinxAtStartPar
Spindle status labels show the current status of the item.


\begin{savenotes}\sphinxattablestart
\sphinxthistablewithglobalstyle
\centering
\sphinxcapstartof{table}
\sphinxthecaptionisattop
\sphinxcaption{Spindle Status Labels}\label{\detokenize{spindle:id1}}
\sphinxaftertopcaption
\begin{tabulary}{\linewidth}[t]{|T|T|T|}
\sphinxtoprule
\sphinxtableatstartofbodyhook
\sphinxAtStartPar
\sphinxstylestrong{Control Function}
&
\sphinxAtStartPar
\sphinxstylestrong{Object Type}
&
\sphinxAtStartPar
\sphinxstylestrong{Object Name}
\\
\sphinxhline
\sphinxAtStartPar
Spindle Brake
&
\sphinxAtStartPar
QLabel
&
\sphinxAtStartPar
spindle\_brake\_0\_lb
\\
\sphinxhline
\sphinxAtStartPar
Spindle Direction
&
\sphinxAtStartPar
QLabel
&
\sphinxAtStartPar
spindle\_direction\_0\_lb
\\
\sphinxhline
\sphinxAtStartPar
Spindle Enabled
&
\sphinxAtStartPar
QLabel
&
\sphinxAtStartPar
spindle\_enabled\_0\_lb
\\
\sphinxhline
\sphinxAtStartPar
Spindle Override Enabled
&
\sphinxAtStartPar
QLabel
&
\sphinxAtStartPar
spindle\_override\_enabled\_0\_lb
\\
\sphinxhline
\sphinxAtStartPar
Spindle Commanded Speed
&
\sphinxAtStartPar
QLabel
&
\sphinxAtStartPar
spindle\_speed\_0\_lb
\\
\sphinxhline
\sphinxAtStartPar
Spindle Speed LCD
&
\sphinxAtStartPar
QLCDNumber
&
\sphinxAtStartPar
spindle\_speed\_0\_lcd
\\
\sphinxhline
\sphinxAtStartPar
Spindle Override Percent
&
\sphinxAtStartPar
QLabel
&
\sphinxAtStartPar
spindle\_override\_0\_lb
\\
\sphinxhline
\sphinxAtStartPar
Spindle Homed
&
\sphinxAtStartPar
QLabel
&
\sphinxAtStartPar
spindle\_homed\_0\_lb
\\
\sphinxhline
\sphinxAtStartPar
Spindle Orient State
&
\sphinxAtStartPar
QLabel
&
\sphinxAtStartPar
spindle\_orient\_state\_0\_lb
\\
\sphinxhline
\sphinxAtStartPar
Spindle Orient Fault
&
\sphinxAtStartPar
QLabel
&
\sphinxAtStartPar
spindle\_orient\_fault\_0\_lb
\\
\sphinxhline
\sphinxAtStartPar
Current S word Setting
&
\sphinxAtStartPar
QLabel
&
\sphinxAtStartPar
settings\_speed\_lb
\\
\sphinxhline
\sphinxAtStartPar
Spindle Actual Speed
&
\sphinxAtStartPar
QLabel
&
\sphinxAtStartPar
spindle\_actual\_speed\_lb
\\
\sphinxhline
\sphinxAtStartPar
Spindle Speed Setting
&
\sphinxAtStartPar
QLabel
&
\sphinxAtStartPar
spindle\_speed\_setting\_lb
\\
\sphinxbottomrule
\end{tabulary}
\sphinxtableafterendhook\par
\sphinxattableend\end{savenotes}

\begin{sphinxadmonition}{note}{Note:}
\sphinxAtStartPar
Spindle Commanded Speed does not show override. Spindle Actual Speed
is actual speed including override.
\end{sphinxadmonition}

\begin{sphinxadmonition}{note}{Note:}
\sphinxAtStartPar
The digitCount property of the LCD must be large enough to display the
whole number.
\end{sphinxadmonition}

\sphinxAtStartPar
On start\sphinxhyphen{}up, Flex will check for the following items in the {[}SPINDLE\_0{]} section
of the .ini file

\sphinxAtStartPar
If \sphinxtitleref{INCREMENT} is not found, Flex will look in the .ini {[}DISPLAY{]} section for
\sphinxtitleref{SPINDLE\_INCREMENT} and if not there will default the increment to 10 for
spindle faster/slower control buttons and spindle step for the QSpinBox.

\sphinxAtStartPar
If \sphinxtitleref{MIN\_FORWARD\_VELOCITY} is found, it will be used to set the QSpinBox minimum
setting. If not found, the minimum setting will be 0.

\sphinxAtStartPar
If \sphinxtitleref{MAX\_FORWARD\_VELOCITY} is found, it will set the QSpinBox maximum setting.
If not found, the maximum setting will be 1000.

\sphinxAtStartPar
\sphinxtitleref{INCREMENT} will also set the QSpinBox single step when using the up/down
arrows.


\section{Spindle Controls}
\label{\detokenize{spindle:spindle-controls}}
\sphinxAtStartPar
The following items control the spindle on/off direction and speed


\begin{savenotes}\sphinxattablestart
\sphinxthistablewithglobalstyle
\centering
\sphinxcapstartof{table}
\sphinxthecaptionisattop
\sphinxcaption{Spindle Status Labels}\label{\detokenize{spindle:id2}}
\sphinxaftertopcaption
\begin{tabulary}{\linewidth}[t]{|T|T|T|}
\sphinxtoprule
\sphinxtableatstartofbodyhook
\sphinxAtStartPar
\sphinxstylestrong{Control Function}
&
\sphinxAtStartPar
\sphinxstylestrong{Object Type}
&
\sphinxAtStartPar
\sphinxstylestrong{Object Name}
\\
\sphinxhline
\sphinxAtStartPar
Spindle Forward
&
\sphinxAtStartPar
QPushButton
&
\sphinxAtStartPar
spindle\_fwd\_pb
\\
\sphinxhline
\sphinxAtStartPar
Spindle Reverse
&
\sphinxAtStartPar
QPushButton
&
\sphinxAtStartPar
spindle\_rev\_pb
\\
\sphinxhline
\sphinxAtStartPar
Spindle Stop
&
\sphinxAtStartPar
QPushButton
&
\sphinxAtStartPar
spindle\_stop\_pb
\\
\sphinxhline
\sphinxAtStartPar
Spindle Faster
&
\sphinxAtStartPar
QPushButton
&
\sphinxAtStartPar
spindle\_plus\_pb
\\
\sphinxhline
\sphinxAtStartPar
Spindle Slower
&
\sphinxAtStartPar
QPushButton
&
\sphinxAtStartPar
spindle\_minus\_pb
\\
\sphinxhline
\sphinxAtStartPar
Spindle Speed
&
\sphinxAtStartPar
QSpinBox
&
\sphinxAtStartPar
spindle\_speed\_sb
\\
\sphinxbottomrule
\end{tabulary}
\sphinxtableafterendhook\par
\sphinxattableend\end{savenotes}

\begin{sphinxadmonition}{note}{Note:}
\sphinxAtStartPar
The spindle can not be started with a spindle speed of zero.
\end{sphinxadmonition}


\section{Spindle Overrides}
\label{\detokenize{spindle:spindle-overrides}}
\sphinxAtStartPar
The spindle speed override is set using a QSlider. See the Status Labels above
for spindle override status labels.


\begin{savenotes}\sphinxattablestart
\sphinxthistablewithglobalstyle
\centering
\sphinxcapstartof{table}
\sphinxthecaptionisattop
\sphinxcaption{Spindle Override}\label{\detokenize{spindle:id3}}
\sphinxaftertopcaption
\begin{tabulary}{\linewidth}[t]{|T|T|T|}
\sphinxtoprule
\sphinxtableatstartofbodyhook
\sphinxAtStartPar
\sphinxstylestrong{Control Function}
&
\sphinxAtStartPar
\sphinxstylestrong{Object Type}
&
\sphinxAtStartPar
\sphinxstylestrong{Object Name}
\\
\sphinxhline
\sphinxAtStartPar
Spindle Override
&
\sphinxAtStartPar
QSlider
&
\sphinxAtStartPar
spindle\_override\_sl
\\
\sphinxbottomrule
\end{tabulary}
\sphinxtableafterendhook\par
\sphinxattableend\end{savenotes}

\sphinxstepscope


\chapter{Parameters}
\label{\detokenize{parameters:parameters}}\label{\detokenize{parameters::doc}}
\sphinxAtStartPar
Parameter values in the var file can be set and watched for changes.


\section{Setting Parameters}
\label{\detokenize{parameters:setting-parameters}}
\sphinxAtStartPar
To set a user parameter value in the var file with a QDoubleSpinBox add a couple
of string type Dynamic Properties. See {\hyperref[\detokenize{property::doc}]{\sphinxcrossref{\DUrole{doc}{Dynamic Properties}}}} The parameters 31 \sphinxhyphen{} 5000
are available for use in NC code programs. Replace \sphinxtitleref{nnnn} with the variable
number.

\begin{sphinxVerbatim}[commandchars=\\\{\}]
function set\PYGZus{}var
variable `nnnn`
\end{sphinxVerbatim}

\noindent{\hspace*{\fill}\sphinxincludegraphics{{parameters-01}.png}\hspace*{\fill}}

\sphinxAtStartPar
The user parameter must be in the var file. On startup Flex reads the var file
and sets the value of the QDoubleSpinBox to that value. When you change the
value of the QDoubleSpinBox the var file is updated with the new value. There is
a 0.5 second timeout before the var file is updated to give time to type in a
number.

\sphinxAtStartPar
The configuration must be out of E\sphinxhyphen{}Stop, Power On and Homed before the
QDoubleSpinBox is enabled.


\section{Watching Parameters}
\label{\detokenize{parameters:watching-parameters}}
\sphinxAtStartPar
To watch the value of a user parameter a QLabel can be used with the following
string type Dynamic Properties. See {\hyperref[\detokenize{property::doc}]{\sphinxcrossref{\DUrole{doc}{Dynamic Properties}}}}

\begin{sphinxVerbatim}[commandchars=\\\{\}]
function watch\PYGZus{}var
variable `nnnn`
\end{sphinxVerbatim}

\noindent{\hspace*{\fill}\sphinxincludegraphics{{parameters-02}.png}\hspace*{\fill}}

\sphinxAtStartPar
The value is updated at startup and any time the var file is updated.

\sphinxAtStartPar
The default precision for a parameter value is 6, to change the precision add a
Dynamic Property called \sphinxtitleref{precision} and set the value to the number of digits
after the dot.

\noindent{\hspace*{\fill}\sphinxincludegraphics{{parameters-03}.png}\hspace*{\fill}}

\begin{sphinxadmonition}{note}{Note:}
\sphinxAtStartPar
The var file is not updated until the end of a NC file.
\end{sphinxadmonition}

\sphinxAtStartPar
Example of setting and watching parameters

\noindent{\hspace*{\fill}\sphinxincludegraphics{{parameters-04}.png}\hspace*{\fill}}

\sphinxstepscope


\chapter{Python Module}
\label{\detokenize{python:python-module}}\label{\detokenize{python::doc}}
\sphinxAtStartPar
\sphinxhref{https://youtu.be/QC4K\_8VMc6Y}{Python Import Tutorial}

\sphinxAtStartPar
To import a python module add the following to the INI {[}FLEXGUI{]} section using the
name of the python file without the.py extension. The file name must be unique
and can not be any python module name. You can have as many imports as you need
to simplify your code.

\begin{sphinxVerbatim}[commandchars=\\\{\}]
\PYG{p}{[}\PYG{n}{FLEXGUI}\PYG{p}{]}
\PYG{n}{IMPORT\PYGZus{}PYTHON} \PYG{o}{=} \PYG{n}{testpy}
\end{sphinxVerbatim}

\begin{sphinxadmonition}{note}{Note:}
\sphinxAtStartPar
The module requires the .py extension to be able to import so the
above module would be named testpy.py.
\end{sphinxadmonition}

\sphinxAtStartPar
In each python file you import you must have a \sphinxtitleref{startup} function where you make
any connections from objects in the ui file to code in your module. The parent
is passed to the startup function to give you access to all the objects in the
GUI.

\begin{sphinxVerbatim}[commandchars=\\\{\}]
\PYG{k+kn}{from} \PYG{n+nn}{functools} \PYG{k+kn}{import} \PYG{n}{partial}

\PYG{k}{def} \PYG{n+nf}{startup}\PYG{p}{(}\PYG{n}{parent}\PYG{p}{)}\PYG{p}{:}
        \PYG{c+c1}{\PYGZsh{} connect a pushbutton without passing parent}
        \PYG{n}{parent}\PYG{o}{.}\PYG{n}{my\PYGZus{}test\PYGZus{}pb}\PYG{o}{.}\PYG{n}{clicked}\PYG{o}{.}\PYG{n}{connect}\PYG{p}{(}\PYG{n}{test\PYGZus{}1}\PYG{p}{)}
        \PYG{n}{parent}\PYG{o}{.}\PYG{n}{get\PYGZus{}names\PYGZus{}pb}\PYG{o}{.}\PYG{n}{clicked}\PYG{o}{.}\PYG{n}{connect}\PYG{p}{(}\PYG{n}{partial}\PYG{p}{(}\PYG{n}{get\PYGZus{}names}\PYG{p}{,} \PYG{n}{parent}\PYG{p}{)}\PYG{p}{)}

        \PYG{c+c1}{\PYGZsh{} connect a pushbutton and pass parent to the function}
        \PYG{n}{parent}\PYG{o}{.}\PYG{n}{another\PYGZus{}test\PYGZus{}pb}\PYG{o}{.}\PYG{n}{clicked}\PYG{o}{.}\PYG{n}{connect}\PYG{p}{(}\PYG{n}{partial}\PYG{p}{(}\PYG{n}{test\PYGZus{}2}\PYG{p}{,} \PYG{n}{parent}\PYG{p}{)}\PYG{p}{)}

\PYG{k}{def} \PYG{n+nf}{test\PYGZus{}1}\PYG{p}{(}\PYG{p}{)}\PYG{p}{:}
        \PYG{n+nb}{print}\PYG{p}{(}\PYG{l+s+s1}{\PYGZsq{}}\PYG{l+s+s1}{test 1}\PYG{l+s+s1}{\PYGZsq{}}\PYG{p}{)}

\PYG{k}{def} \PYG{n+nf}{test\PYGZus{}2}\PYG{p}{(}\PYG{n}{parent}\PYG{p}{)}\PYG{p}{:}
        \PYG{c+c1}{\PYGZsh{} in this function you have access to all the objects in parent}
        \PYG{n+nb}{print}\PYG{p}{(}\PYG{l+s+sa}{f}\PYG{l+s+s1}{\PYGZsq{}}\PYG{l+s+s1}{test 2 }\PYG{l+s+si}{\PYGZob{}}\PYG{n}{parent}\PYG{o}{.}\PYG{n}{another\PYGZus{}test\PYGZus{}pb}\PYG{o}{.}\PYG{n}{text}\PYG{p}{(}\PYG{p}{)}\PYG{l+s+si}{\PYGZcb{}}\PYG{l+s+s1}{\PYGZsq{}}\PYG{p}{)}

\PYG{k}{def} \PYG{n+nf}{get\PYGZus{}names}\PYG{p}{(}\PYG{n}{parent}\PYG{p}{)}\PYG{p}{:}
        \PYG{c+c1}{\PYGZsh{} get all the object names from the parent}
        \PYG{n+nb}{print}\PYG{p}{(}\PYG{n+nb}{dir}\PYG{p}{(}\PYG{n}{parent}\PYG{p}{)}\PYG{p}{)}
\end{sphinxVerbatim}


\section{Timer}
\label{\detokenize{python:timer}}
\sphinxAtStartPar
A user timer is provided for use in the user python module.

\begin{sphinxVerbatim}[commandchars=\\\{\}]
\PYG{k+kn}{from} \PYG{n+nn}{functools} \PYG{k+kn}{import} \PYG{n}{partial}

\PYG{k}{def} \PYG{n+nf}{startup}\PYG{p}{(}\PYG{n}{parent}\PYG{p}{)}\PYG{p}{:}
        \PYG{n}{parent}\PYG{o}{.}\PYG{n}{user\PYGZus{}timer}\PYG{o}{.}\PYG{n}{timeout}\PYG{o}{.}\PYG{n}{connect}\PYG{p}{(}\PYG{n}{testit}\PYG{p}{)}
        \PYG{n}{parent}\PYG{o}{.}\PYG{n}{conn\PYGZus{}pb}\PYG{o}{.}\PYG{n}{setEnabled}\PYG{p}{(}\PYG{k+kc}{False}\PYG{p}{)} \PYG{c+c1}{\PYGZsh{} prevent another connection}
        \PYG{n}{parent}\PYG{o}{.}\PYG{n}{disc\PYGZus{}pb}\PYG{o}{.}\PYG{n}{clicked}\PYG{o}{.}\PYG{n}{connect}\PYG{p}{(}\PYG{n}{partial}\PYG{p}{(}\PYG{n}{disc}\PYG{p}{,} \PYG{n}{parent}\PYG{p}{)}\PYG{p}{)}
        \PYG{n}{parent}\PYG{o}{.}\PYG{n}{conn\PYGZus{}pb}\PYG{o}{.}\PYG{n}{clicked}\PYG{o}{.}\PYG{n}{connect}\PYG{p}{(}\PYG{n}{partial}\PYG{p}{(}\PYG{n}{conn}\PYG{p}{,} \PYG{n}{parent}\PYG{p}{)}\PYG{p}{)}
        \PYG{n}{parent}\PYG{o}{.}\PYG{n}{start\PYGZus{}pb}\PYG{o}{.}\PYG{n}{clicked}\PYG{o}{.}\PYG{n}{connect}\PYG{p}{(}\PYG{n}{partial}\PYG{p}{(}\PYG{n}{start}\PYG{p}{,} \PYG{n}{parent}\PYG{p}{)}\PYG{p}{)}
        \PYG{n}{parent}\PYG{o}{.}\PYG{n}{stop\PYGZus{}pb}\PYG{o}{.}\PYG{n}{clicked}\PYG{o}{.}\PYG{n}{connect}\PYG{p}{(}\PYG{n}{partial}\PYG{p}{(}\PYG{n}{stop}\PYG{p}{,} \PYG{n}{parent}\PYG{p}{)}\PYG{p}{)}

\PYG{k}{def} \PYG{n+nf}{testit}\PYG{p}{(}\PYG{p}{)}\PYG{p}{:}
        \PYG{n+nb}{print}\PYG{p}{(}\PYG{l+s+s1}{\PYGZsq{}}\PYG{l+s+s1}{testing}\PYG{l+s+s1}{\PYGZsq{}}\PYG{p}{)}

\PYG{k}{def} \PYG{n+nf}{disc}\PYG{p}{(}\PYG{n}{parent}\PYG{p}{)}\PYG{p}{:}
        \PYG{n}{parent}\PYG{o}{.}\PYG{n}{user\PYGZus{}timer}\PYG{o}{.}\PYG{n}{timeout}\PYG{o}{.}\PYG{n}{disconnect}\PYG{p}{(}\PYG{n}{testit}\PYG{p}{)}
        \PYG{n}{parent}\PYG{o}{.}\PYG{n}{conn\PYGZus{}pb}\PYG{o}{.}\PYG{n}{setEnabled}\PYG{p}{(}\PYG{k+kc}{True}\PYG{p}{)} \PYG{c+c1}{\PYGZsh{} allow a connection}
        \PYG{n}{parent}\PYG{o}{.}\PYG{n}{disc\PYGZus{}pb}\PYG{o}{.}\PYG{n}{setEnabled}\PYG{p}{(}\PYG{k+kc}{False}\PYG{p}{)} \PYG{c+c1}{\PYGZsh{} prevent trying to disconnect}

\PYG{k}{def} \PYG{n+nf}{conn}\PYG{p}{(}\PYG{n}{parent}\PYG{p}{)}\PYG{p}{:}
        \PYG{n}{parent}\PYG{o}{.}\PYG{n}{user\PYGZus{}timer}\PYG{o}{.}\PYG{n}{timeout}\PYG{o}{.}\PYG{n}{connect}\PYG{p}{(}\PYG{n}{testit}\PYG{p}{)}
        \PYG{n}{parent}\PYG{o}{.}\PYG{n}{conn\PYGZus{}pb}\PYG{o}{.}\PYG{n}{setEnabled}\PYG{p}{(}\PYG{k+kc}{False}\PYG{p}{)} \PYG{c+c1}{\PYGZsh{} prevent trying to connect}
        \PYG{n}{parent}\PYG{o}{.}\PYG{n}{disc\PYGZus{}pb}\PYG{o}{.}\PYG{n}{setEnabled}\PYG{p}{(}\PYG{k+kc}{True}\PYG{p}{)} \PYG{c+c1}{\PYGZsh{} allow a disconnect}

\PYG{k}{def} \PYG{n+nf}{start}\PYG{p}{(}\PYG{n}{parent}\PYG{p}{)}\PYG{p}{:}
        \PYG{n}{parent}\PYG{o}{.}\PYG{n}{user\PYGZus{}timer}\PYG{o}{.}\PYG{n}{start}\PYG{p}{(}\PYG{l+m+mi}{1000}\PYG{p}{)} \PYG{c+c1}{\PYGZsh{} milliseconds}

\PYG{k}{def} \PYG{n+nf}{stop}\PYG{p}{(}\PYG{n}{parent}\PYG{p}{)}\PYG{p}{:}
        \PYG{n}{parent}\PYG{o}{.}\PYG{n}{user\PYGZus{}timer}\PYG{o}{.}\PYG{n}{stop}\PYG{p}{(}\PYG{p}{)}
\end{sphinxVerbatim}

\sphinxstepscope


\chapter{Probing}
\label{\detokenize{probe:probing}}\label{\detokenize{probe::doc}}

\section{Probe Enable}
\label{\detokenize{probe:probe-enable}}
\sphinxAtStartPar
Add a QPushButton named \sphinxtitleref{probing\_enable\_pb} and if it is found it will be set as
a toggle button. The button will only be enabled when the machine is homed and
not running a program. The button is set to checkable in code so it can be
styled with :checked and :enabled pseudo\sphinxhyphen{}states among others.

\begin{sphinxadmonition}{note}{Note:}
\sphinxAtStartPar
The \sphinxtitleref{probing\_enable\_pb} requires at least one object that starts with
\sphinxtitleref{probe\_} to be enabled.
\end{sphinxadmonition}

\begin{sphinxVerbatim}[commandchars=\\\{\}]
\PYG{n+nt}{QPushButton}\PYG{p}{\PYGZsh{}}\PYG{n+nn}{probing\PYGZus{}enable\PYGZus{}pb}\PYG{p}{:}\PYG{n+nd}{enabled}\PYG{p}{:}\PYG{n+nd}{checked}\PYG{+w}{ }\PYG{p}{\PYGZob{}}
\PYG{+w}{        }\PYG{k}{color}\PYG{p}{:}\PYG{+w}{ }\PYG{k+kc}{white}\PYG{p}{;}
\PYG{+w}{        }\PYG{k}{background\PYGZhy{}color}\PYG{p}{:}\PYG{+w}{ }\PYG{k+kc}{red}\PYG{p}{;}
\PYG{p}{\PYGZcb{}}

\PYG{n+nt}{QPushButton}\PYG{p}{\PYGZsh{}}\PYG{n+nn}{probing\PYGZus{}enable\PYGZus{}pb}\PYG{p}{:}\PYG{n+nd}{enabled}\PYG{+w}{ }\PYG{p}{\PYGZob{}}
\PYG{+w}{        }\PYG{k}{color}\PYG{p}{:}\PYG{+w}{ }\PYG{k+kc}{green}\PYG{p}{;}
\PYG{+w}{        }\PYG{k}{background\PYGZhy{}color}\PYG{p}{:}\PYG{+w}{ }\PYG{k+kc}{yellow}\PYG{p}{;}
\PYG{p}{\PYGZcb{}}
\end{sphinxVerbatim}

\begin{sphinxadmonition}{note}{Note:}
\sphinxAtStartPar
A QPushButton with checkable set to true does not have an unchecked
pseudo\sphinxhyphen{}state
\end{sphinxadmonition}

\sphinxAtStartPar
For more style sheet options see the {\hyperref[\detokenize{style::doc}]{\sphinxcrossref{\DUrole{doc}{StyleSheet}}}} section

\sphinxAtStartPar
The text on the Probe Enable button can be set by adding two Dynamic Properties
named \sphinxtitleref{on\_text} and \sphinxtitleref{off\_text}. Both must be present or no change will take
place as there is no default text for the Probe Enable button.
See {\hyperref[\detokenize{property::doc}]{\sphinxcrossref{\DUrole{doc}{Dynamic Properties}}}}

\noindent{\hspace*{\fill}\sphinxincludegraphics{{probe-00}.png}\hspace*{\fill}}

\sphinxAtStartPar
This is what the button would look like with the above settings.

\noindent{\hspace*{\fill}\sphinxincludegraphics{{probe-01}.png}\hspace*{\fill}}


\section{Function}
\label{\detokenize{probe:function}}
\sphinxAtStartPar
When the \sphinxtitleref{probing\_enable\_pb} is toggled \sphinxtitleref{OFF}, any widget with an object name
that starts with \sphinxtitleref{probe\_} will be disabled.

\sphinxAtStartPar
When the \sphinxtitleref{probing\_enable\_pb} is toggled \sphinxtitleref{ON} the widgets that start with
\sphinxtitleref{probe\_} will be enabled. In addition, spindle controls will disabled, spindle
speed set to 0, run controls will be disabled, and MDI controls will be disabled.

\sphinxAtStartPar
QPushButtons with an objectName that start with \sphinxtitleref{probe\_} and configured as a
{\hyperref[\detokenize{mdi:mdibuttontag}]{\sphinxcrossref{\DUrole{std,std-ref}{MDI Button}}}} (to launch the probing subroutines) will be enabled when
probing is enabled and disabled when probing is disabled.

\sphinxAtStartPar
You can create a HAL {\hyperref[\detokenize{hal:spinboxtag}]{\sphinxcrossref{\DUrole{std,std-ref}{Spinbox}}}} to use in your probing subroutine. Set
the objectName to start with \sphinxtitleref{probe\_} and it will be enabled and disabled with
the probe buttons.

\sphinxAtStartPar
If you’re using a touch\sphinxhyphen{}screen, add a Dynamic Property named \sphinxtitleref{input} and set
the value to \sphinxtitleref{touch}.


\section{Example}
\label{\detokenize{probe:example}}
\sphinxAtStartPar
A minimal example is in the Flex Examples in the Features directory

\noindent{\hspace*{\fill}\sphinxincludegraphics{{probe-02}.png}\hspace*{\fill}}

\sphinxAtStartPar
To run the example, close the E Stop, turn Power on, Home all, then toggle the
Enable button. When probing is enabled many other controls are disabled
including the spindle.

\sphinxAtStartPar
To test the probe routine press the Probe X Plus button and the X axis will
start to move in the positive direction. If you do nothing when the Search
Distance is reached you will get an error that the G38.2 move finished without
making contact, which is expected.

\sphinxAtStartPar
Press the Probe X Plus button again and after it starts to move, press the Probe
Trip button, the X axis will back off the Backoff Distance and start to move in
the positive direction again. Pressing the Probe Trip button again will end the
probe simulation. The debug information will show up in the Information window.


\section{Subroutine}
\label{\detokenize{probe:subroutine}}
\sphinxAtStartPar
The probe subroutines use the values from the Probe Settings spin boxes. To use
these values, you need to make the spin box a HAL pin. See the {\hyperref[\detokenize{hal:spinboxtag}]{\sphinxcrossref{\DUrole{std,std-ref}{Spinbox}}}}
example in the HAL section.

\sphinxAtStartPar
The subroutine is located in a directory called \sphinxtitleref{subroutines} that is in the
configuration directory. The ini’s {[}RS274NGC{]} SUBROUTINE\_PATH sets the path
that LinuxCNC looks for subroutines. Notice the leading ./ specifices that the
path to the current directory is where the subroutine directory is.

\begin{sphinxVerbatim}[commandchars=\\\{\}]
\PYG{n}{SUBROUTINE\PYGZus{}PATH} \PYG{o}{=} \PYG{o}{.}\PYG{o}{/}\PYG{n}{subroutines}
\end{sphinxVerbatim}

\sphinxAtStartPar
The example files used are the following; notice that the xplus.ngc is in
the ./subroutines directory

\begin{sphinxVerbatim}[commandchars=\\\{\}]
├── main.hal
├── parameters.var
├── parameters.var.bak
├── postgui.hal
├── probe.ini
├── probe.ui
├── README
├── sim\PYGZus{}axis\PYGZus{}probe.ini
├── subroutines
│   └── xplus.ngc
└── tool.tbl
\end{sphinxVerbatim}

\sphinxAtStartPar
In your subroutine you can use user parameters instead of using HAL pins.
See the {\hyperref[\detokenize{parameters::doc}]{\sphinxcrossref{\DUrole{doc}{Parameters}}}} section.

\sphinxAtStartPar
The subroutine is a normal LinuxCNC subroutine. The magic is how you get the
values from HAL pins with \sphinxtitleref{\#\textless{}\_hal{[}pin\_name{]}\textgreater{}} where pin\_name is the actual
pin name in HAL.

\begin{sphinxVerbatim}[commandchars=\\\{\}]
\PYG{p}{(}\PYG{n}{filename} \PYG{n}{xplus}\PYG{o}{.}\PYG{n}{ngc}\PYG{p}{)}
\PYG{p}{(}\PYG{n}{HAL} \PYG{n}{pins} \PYG{c+c1}{\PYGZsh{}\PYGZlt{}\PYGZus{}hal[pin\PYGZus{}name]\PYGZgt{})}
\PYG{p}{(}\PYG{n}{G90} \PYG{n}{absolute} \PYG{n}{distance} \PYG{n}{mode} \PYG{n}{G91} \PYG{n}{incremental} \PYG{n}{distance} \PYG{n}{mode}\PYG{p}{)}
\PYG{n}{o}\PYG{o}{\PYGZlt{}}\PYG{n}{xplus}\PYG{o}{\PYGZgt{}} \PYG{n}{sub}
        \PYG{p}{(}\PYG{n}{msg}\PYG{p}{,} \PYG{n}{xplus} \PYG{n}{subroutine}\PYG{p}{)}
        \PYG{n}{G20}
        \PYG{p}{;} \PYG{n}{initial} \PYG{n}{search}
        \PYG{n}{G91} \PYG{n}{G38}\PYG{l+m+mf}{.2} \PYG{n}{F}\PYG{c+c1}{\PYGZsh{}\PYGZlt{}\PYGZus{}hal[flexhal.search\PYGZhy{}speed]\PYGZgt{} X\PYGZsh{}\PYGZlt{}\PYGZus{}hal[flexhal.search\PYGZhy{}distance]\PYGZgt{}}
        \PYG{p}{;}\PYG{l+m+mi}{5061}\PYG{o}{\PYGZhy{}}\PYG{l+m+mi}{5069} \PYG{o}{\PYGZhy{}} \PYG{n}{Coordinates} \PYG{n}{of} \PYG{n}{a} \PYG{n}{G38} \PYG{n}{probe} \PYG{n}{result} \PYG{p}{(}\PYG{n}{X}\PYG{p}{,} \PYG{n}{Y}\PYG{p}{,} \PYG{n}{Z}\PYG{p}{,} \PYG{n}{A}\PYG{p}{,} \PYG{n}{B}\PYG{p}{,} \PYG{n}{C}\PYG{p}{,} \PYG{n}{U}\PYG{p}{,} \PYG{n}{V} \PYG{o}{\PYGZam{}} \PYG{n}{W}\PYG{p}{)}
        \PYG{p}{(}\PYG{n}{debug}\PYG{p}{,} \PYG{n}{Probe} \PYG{n}{Contact} \PYG{n}{at} \PYG{c+c1}{\PYGZsh{}5061)}
        \PYG{p}{;} \PYG{n}{back} \PYG{n}{off} \PYG{n}{using} \PYG{c+c1}{\PYGZsh{}5061 to compensate for over travel on the probe}
        \PYG{n}{G90} \PYG{n}{G0} \PYG{n}{X}\PYG{p}{[}\PYG{c+c1}{\PYGZsh{}5061\PYGZhy{}\PYGZsh{}\PYGZlt{}\PYGZus{}hal[flexhal.backoff\PYGZhy{}distance]\PYGZgt{}]}
        \PYG{p}{;} \PYG{n}{final} \PYG{n}{probe} \PYG{n}{at} \PYG{n}{latch} \PYG{n}{speed}
        \PYG{n}{G91} \PYG{n}{G38}\PYG{l+m+mf}{.2} \PYG{n}{F}\PYG{c+c1}{\PYGZsh{}\PYGZlt{}\PYGZus{}hal[flexhal.latch\PYGZhy{}speed]\PYGZgt{} X[\PYGZsh{}\PYGZlt{}\PYGZus{}hal[flexhal.backoff\PYGZhy{}distance]\PYGZgt{} + 0.02]}
        \PYG{p}{(}\PYG{n}{debug}\PYG{p}{,} \PYG{n}{Probe} \PYG{n}{Contact} \PYG{n}{at} \PYG{c+c1}{\PYGZsh{}5061)}
\PYG{n}{o}\PYG{o}{\PYGZlt{}}\PYG{n}{xplus}\PYG{o}{\PYGZgt{}} \PYG{n}{endsub}
\PYG{n}{M2}
\end{sphinxVerbatim}

\sphinxAtStartPar
Looking at the Halshow window which pops up when you press the Show HAL button,
you can see the flexhal pin names for each spin box and for the Probe Trip
button. Also notice that the Probe Trip button is connected to a signal which is
connected to motion.probe\sphinxhyphen{}input in the postgui.hal file

\noindent{\hspace*{\fill}\sphinxincludegraphics{{probe-03}.png}\hspace*{\fill}}

\sphinxstepscope


\chapter{Tools}
\label{\detokenize{tools:tools}}\label{\detokenize{tools::doc}}
\sphinxAtStartPar
\sphinxhref{https://youtu.be/SQZ6RJj9hP8}{Tools Tutorial}


\section{Tool Change}
\label{\detokenize{tools:tool-change}}
\noindent{\hspace*{\fill}\sphinxincludegraphics{{tools-01}.png}\hspace*{\fill}}

\sphinxAtStartPar
A tool change QPushButton, with a QComboBox to select the tool number to change
to, is done with QPushButton named \sphinxtitleref{tool\_change\_pb} and a QComboBox named
\sphinxtitleref{tool\_change\_cb}. The tool change combobox will automatically be populated with
all the tools found in the tool table.


\begin{savenotes}\sphinxattablestart
\sphinxthistablewithglobalstyle
\centering
\sphinxcapstartof{table}
\sphinxthecaptionisattop
\sphinxcaption{Tool Change Controls}\label{\detokenize{tools:id1}}
\sphinxaftertopcaption
\begin{tabulary}{\linewidth}[t]{|T|T|T|}
\sphinxtoprule
\sphinxtableatstartofbodyhook
\sphinxAtStartPar
\sphinxstylestrong{Control Function}
&
\sphinxAtStartPar
\sphinxstylestrong{Object Type}
&
\sphinxAtStartPar
\sphinxstylestrong{Object Name}
\\
\sphinxhline
\sphinxAtStartPar
Tool Change
&
\sphinxAtStartPar
QPushButton
&
\sphinxAtStartPar
tool\_change\_pb
\\
\sphinxhline
\sphinxAtStartPar
Tool Selector
&
\sphinxAtStartPar
QComboBox
&
\sphinxAtStartPar
tool\_change\_cb
\\
\sphinxbottomrule
\end{tabulary}
\sphinxtableafterendhook\par
\sphinxattableend\end{savenotes}

\sphinxAtStartPar
To add the description of the tools to the tool change combo box add a Dynamic
Property named \sphinxtitleref{option} and set the value to \sphinxtitleref{description}. See {\hyperref[\detokenize{property::doc}]{\sphinxcrossref{\DUrole{doc}{Dynamic Properties}}}}

\noindent{\hspace*{\fill}\sphinxincludegraphics{{tools-02}.png}\hspace*{\fill}}

\sphinxAtStartPar
The description from the tool table will be appended to the tool number.

\noindent{\hspace*{\fill}\sphinxincludegraphics{{tools-03}.png}\hspace*{\fill}}

\sphinxAtStartPar
If you have limited space you can define the tool prefix by adding a Dynamic
Property named \sphinxtitleref{prefix} and set the value to the prefix you want.

\noindent{\hspace*{\fill}\sphinxincludegraphics{{tools-04}.png}\hspace*{\fill}}

\sphinxAtStartPar
The tool number will follow the prefix.

\noindent{\hspace*{\fill}\sphinxincludegraphics{{tools-05}.png}\hspace*{\fill}}

\begin{sphinxadmonition}{note}{Note:}
\sphinxAtStartPar
Only one option can be used, if option is found it is used and prefix
will be ignored.
\end{sphinxadmonition}


\section{Manual Tool Change}
\label{\detokenize{tools:manual-tool-change}}
\sphinxAtStartPar
All that is needed to add a manual tool change is to add the following to the
ini file in the {[}FLEXGUI{]} section.

\begin{sphinxVerbatim}[commandchars=\\\{\}]
\PYG{p}{[}\PYG{n}{FLEXGUI}\PYG{p}{]}
\PYG{n}{MANUAL\PYGZus{}TOOL\PYGZus{}CHANGE} \PYG{o}{=} \PYG{k+kc}{True}
\end{sphinxVerbatim}

\begin{figure}[htbp]
\centering
\capstart

\noindent\sphinxincludegraphics{{tools-06}.png}
\caption{This is without using a theme.}\label{\detokenize{tools:id2}}\end{figure}

\begin{figure}[htbp]
\centering
\capstart

\noindent\sphinxincludegraphics{{tools-07}.png}
\caption{This is the blue\sphinxhyphen{}touch theme.}\label{\detokenize{tools:id3}}\end{figure}

\begin{figure}[htbp]
\centering
\capstart

\noindent\sphinxincludegraphics{{tools-08}.png}
\caption{This is the dark\sphinxhyphen{}touch theme.}\label{\detokenize{tools:id4}}\end{figure}

\begin{sphinxadmonition}{warning}{Warning:}
\sphinxAtStartPar
You can’t use the hal\_manualtoolchange at the same time as the
built in Flex Manual Tool Change, you must comment out all the
hal\_manualtoolchange lines or remove them.
\end{sphinxadmonition}


\section{Manual Tool Change Error}
\label{\detokenize{tools:manual-tool-change-error}}
\sphinxAtStartPar
If you get an error that hal\_manualtoolchange component exists look in your hal
files for the hal\_manualtoolchange lines in the Manual Tool Change Option below.

\sphinxAtStartPar
If you’re using a copy of one of the Axis sims the hal\_manualtoolchange
component can be hard to find. It’s recommended that you start with a simple
configuration like the Flex example simple\sphinxhyphen{}sim.


\section{Manual Tool Change Option}
\label{\detokenize{tools:manual-tool-change-option}}
\sphinxAtStartPar
The HAL Manual Tool Change requires at least the following HAL code in the main
hal file if not using the builtin Flex Manual Tool Change above.

\begin{sphinxVerbatim}[commandchars=\\\{\}]
\PYG{c+c1}{\PYGZsh{} manual tool change}
\PYG{n}{loadusr} \PYG{o}{\PYGZhy{}}\PYG{n}{W} \PYG{n}{hal\PYGZus{}manualtoolchange}
\PYG{n}{net} \PYG{n}{tool}\PYG{o}{\PYGZhy{}}\PYG{n}{change} \PYG{n}{iocontrol}\PYG{l+m+mf}{.0}\PYG{o}{.}\PYG{n}{tool}\PYG{o}{\PYGZhy{}}\PYG{n}{change} \PYG{o}{=}\PYG{o}{\PYGZgt{}} \PYG{n}{hal\PYGZus{}manualtoolchange}\PYG{o}{.}\PYG{n}{change}
\PYG{n}{net} \PYG{n}{tool}\PYG{o}{\PYGZhy{}}\PYG{n}{changed} \PYG{n}{iocontrol}\PYG{l+m+mf}{.0}\PYG{o}{.}\PYG{n}{tool}\PYG{o}{\PYGZhy{}}\PYG{n}{changed} \PYG{o}{\PYGZlt{}}\PYG{o}{=} \PYG{n}{hal\PYGZus{}manualtoolchange}\PYG{o}{.}\PYG{n}{changed}
\PYG{n}{net} \PYG{n}{tool}\PYG{o}{\PYGZhy{}}\PYG{n}{number} \PYG{n}{iocontrol}\PYG{l+m+mf}{.0}\PYG{o}{.}\PYG{n}{tool}\PYG{o}{\PYGZhy{}}\PYG{n}{prep}\PYG{o}{\PYGZhy{}}\PYG{n}{number} \PYG{o}{=}\PYG{o}{\PYGZgt{}} \PYG{n}{hal\PYGZus{}manualtoolchange}\PYG{o}{.}\PYG{n}{number}
\PYG{n}{net} \PYG{n}{tool}\PYG{o}{\PYGZhy{}}\PYG{n}{prepare}\PYG{o}{\PYGZhy{}}\PYG{n}{loopback} \PYG{n}{iocontrol}\PYG{l+m+mf}{.0}\PYG{o}{.}\PYG{n}{tool}\PYG{o}{\PYGZhy{}}\PYG{n}{prepare} \PYG{o}{=}\PYG{o}{\PYGZgt{}} \PYG{n}{iocontrol}\PYG{l+m+mf}{.0}\PYG{o}{.}\PYG{n}{tool}\PYG{o}{\PYGZhy{}}\PYG{n}{prepared}
\end{sphinxVerbatim}


\section{Tool Change Button}
\label{\detokenize{tools:tool-change-button}}
\sphinxAtStartPar
Tool change QPushButtons can be used to change tools without a spinbox by adding
up to 99 QPushButtons named \sphinxtitleref{tool\_change\_pb\_n}. With \sphinxtitleref{n} being the number of
the tool you wish to change to using that button


\begin{savenotes}\sphinxattablestart
\sphinxthistablewithglobalstyle
\centering
\sphinxcapstartof{table}
\sphinxthecaptionisattop
\sphinxcaption{Tool Change Buttons}\label{\detokenize{tools:id5}}
\sphinxaftertopcaption
\begin{tabulary}{\linewidth}[t]{|T|T|T|}
\sphinxtoprule
\sphinxtableatstartofbodyhook
\sphinxAtStartPar
\sphinxstylestrong{Control Function}
&
\sphinxAtStartPar
\sphinxstylestrong{Object Type}
&
\sphinxAtStartPar
\sphinxstylestrong{Object Name}
\\
\sphinxhline
\sphinxAtStartPar
Tool Change Button
&
\sphinxAtStartPar
QPushButton
&
\sphinxAtStartPar
tool\_change\_pb\_(n)
\\
\sphinxbottomrule
\end{tabulary}
\sphinxtableafterendhook\par
\sphinxattableend\end{savenotes}


\section{Tool Touchoff}
\label{\detokenize{tools:tool-touchoff}}
\sphinxAtStartPar
To touch\sphinxhyphen{}off a tool to an axis, use a tool\sphinxhyphen{}touch\sphinxhyphen{}off QPushButton and a QLineEdit
to enter the value of the touch off.


\begin{savenotes}\sphinxattablestart
\sphinxthistablewithglobalstyle
\centering
\sphinxcapstartof{table}
\sphinxthecaptionisattop
\sphinxcaption{Tool Touchoff Controls}\label{\detokenize{tools:id6}}
\sphinxaftertopcaption
\begin{tabulary}{\linewidth}[t]{|T|T|T|}
\sphinxtoprule
\sphinxtableatstartofbodyhook
\sphinxAtStartPar
\sphinxstylestrong{Control Function}
&
\sphinxAtStartPar
\sphinxstylestrong{Object Type}
&
\sphinxAtStartPar
\sphinxstylestrong{Object Name}
\\
\sphinxhline
\sphinxAtStartPar
Tool Touch Off Value
&
\sphinxAtStartPar
QLineEdit
&
\sphinxAtStartPar
tool\_touchoff\_le
\\
\sphinxhline
\sphinxAtStartPar
Tool Touch Off
&
\sphinxAtStartPar
QPushButton
&
\sphinxAtStartPar
tool\_touchoff\_(axis letter)
\\
\sphinxbottomrule
\end{tabulary}
\sphinxtableafterendhook\par
\sphinxattableend\end{savenotes}

\sphinxAtStartPar
Optionally you can have a QLineEdit for each axis for tool touch off. Add a
Dynamic Property named \sphinxtitleref{source} to the tool touch off button and set the value
to the name of the QLineEdit that is the source for that touch off button.
See {\hyperref[\detokenize{property::doc}]{\sphinxcrossref{\DUrole{doc}{Dynamic Properties}}}}

\noindent{\hspace*{\fill}\sphinxincludegraphics{{tools-08}.png}\hspace*{\fill}}

\sphinxAtStartPar
Tool touch off QLineEdit for each axis.

\noindent{\hspace*{\fill}\sphinxincludegraphics{{tools-09}.png}\hspace*{\fill}}


\section{Tool Touchoff Selected Axis}
\label{\detokenize{tools:tool-touchoff-selected-axis}}
\sphinxAtStartPar
To have Axis style tool touch off add a QPushButton named
\sphinxtitleref{tool\_touchoff\_selected\_pb}. You must have at least one QRadiobutton for an axis
to select.


\begin{savenotes}\sphinxattablestart
\sphinxthistablewithglobalstyle
\raggedright
\sphinxcapstartof{table}
\sphinxthecaptionisattop
\sphinxcaption{Tool Touchoff Selected Widgets}\label{\detokenize{tools:id7}}
\sphinxaftertopcaption
\begin{tabulary}{\linewidth}[t]{|T|T|T|}
\sphinxtoprule
\sphinxtableatstartofbodyhook
\sphinxAtStartPar
\sphinxstylestrong{Function}
&
\sphinxAtStartPar
\sphinxstylestrong{Widget}
&
\sphinxAtStartPar
\sphinxstylestrong{Name}
\\
\sphinxhline
\sphinxAtStartPar
Axis Select (0\sphinxhyphen{}8)
&
\sphinxAtStartPar
QRadioButton
&
\sphinxAtStartPar
axis\_select\_(0\sphinxhyphen{}8)
\\
\sphinxhline
\sphinxAtStartPar
Tool Touchoff
&
\sphinxAtStartPar
QPushButton
&
\sphinxAtStartPar
tool\_touchoff\_selected\_pb
\\
\sphinxbottomrule
\end{tabulary}
\sphinxtableafterendhook\par
\sphinxattableend\end{savenotes}


\section{Current Tool Status}
\label{\detokenize{tools:current-tool-status}}
\sphinxAtStartPar
Current Tool status of the tool loaded in the spindle. All the labels can have a
Dynamic Property called \sphinxtitleref{precision} with the number of digits you wish to show.
The \sphinxtitleref{tool\_id\_lb} and the \sphinxtitleref{tool\_orientation\_lb} are integers.


\begin{savenotes}\sphinxattablestart
\sphinxthistablewithglobalstyle
\centering
\sphinxcapstartof{table}
\sphinxthecaptionisattop
\sphinxcaption{Tool Table Status Labels}\label{\detokenize{tools:id8}}
\sphinxaftertopcaption
\begin{tabular}[t]{|\X{40}{120}|\X{40}{120}|\X{40}{120}|}
\sphinxtoprule
\sphinxtableatstartofbodyhook
\sphinxAtStartPar
tool\_id\_lb
&
\sphinxAtStartPar
tool\_xoffset\_lb
&
\sphinxAtStartPar
tool\_yoffset\_lb
\\
\sphinxhline
\sphinxAtStartPar
tool\_zoffset\_lb
&
\sphinxAtStartPar
tool\_aoffset\_lb
&
\sphinxAtStartPar
tool\_boffset\_lb
\\
\sphinxhline
\sphinxAtStartPar
tool\_coffset\_lb
&
\sphinxAtStartPar
tool\_uoffset\_lb
&
\sphinxAtStartPar
tool\_voffset\_lb
\\
\sphinxhline
\sphinxAtStartPar
tool\_woffset\_lb
&
\sphinxAtStartPar
tool\_diameter\_lb
&
\sphinxAtStartPar
tool\_frontangle\_lb
\\
\sphinxhline
\sphinxAtStartPar
tool\_backangle\_lb
&
\sphinxAtStartPar
tool\_orientation\_lb
&\\
\sphinxbottomrule
\end{tabular}
\sphinxtableafterendhook\par
\sphinxattableend\end{savenotes}

\sphinxstepscope


\chapter{Coordinate Systems}
\label{\detokenize{coordinates:coordinate-systems}}\label{\detokenize{coordinates::doc}}
\sphinxAtStartPar
\sphinxhref{https://youtu.be/Bsk7\_Ij7tVc/}{Coordinate System Tutorial}


\section{Coordinate System Touchoff}
\label{\detokenize{coordinates:coordinate-system-touchoff}}
\sphinxAtStartPar
To touch\sphinxhyphen{}off an axis, use a QPushButton and QLineEdit to set the touch\sphinxhyphen{}off value.
Optionally you can have a QComboBox to select the Coordinate System to touch off
to.


\begin{savenotes}\sphinxattablestart
\sphinxthistablewithglobalstyle
\centering
\sphinxcapstartof{table}
\sphinxthecaptionisattop
\sphinxcaption{Coordinate System Touch Off Controls}\label{\detokenize{coordinates:id1}}
\sphinxaftertopcaption
\begin{tabulary}{\linewidth}[t]{|T|T|T|}
\sphinxtoprule
\sphinxtableatstartofbodyhook
\sphinxAtStartPar
\sphinxstylestrong{Control Function}
&
\sphinxAtStartPar
\sphinxstylestrong{Object Type}
&
\sphinxAtStartPar
\sphinxstylestrong{Object Name}
\\
\sphinxhline
\sphinxAtStartPar
Touch Off Axis
&
\sphinxAtStartPar
QPushButton
&
\sphinxAtStartPar
touchoff\_pb\_(axis letter)
\\
\sphinxhline
\sphinxAtStartPar
Touch Off Value
&
\sphinxAtStartPar
QLineEdit
&
\sphinxAtStartPar
touchoff\_le
\\
\sphinxhline
\sphinxAtStartPar
Coordinate System
&
\sphinxAtStartPar
QComboBox
&
\sphinxAtStartPar
touchoff\_system\_cb
\\
\sphinxbottomrule
\end{tabulary}
\sphinxtableafterendhook\par
\sphinxattableend\end{savenotes}

\sphinxAtStartPar
Optionally you can have a QLineEdit for any axis by adding a string type Dynamic
Property named \sphinxtitleref{source} to the QPushButton and the value contains the object
name of the QLineEdit that you want to use. See {\hyperref[\detokenize{property::doc}]{\sphinxcrossref{\DUrole{doc}{Dynamic Properties}}}}

\noindent{\hspace*{\fill}\sphinxincludegraphics{{coordinate-01}.png}\hspace*{\fill}}

\sphinxAtStartPar
As you can see you can have a QLineEdit for each axis.

\noindent{\hspace*{\fill}\sphinxincludegraphics{{coordinate-02}.png}\hspace*{\fill}}


\section{Change Coordinate System}
\label{\detokenize{coordinates:change-coordinate-system}}
\sphinxAtStartPar
To change the coordinate system via a button, use a change\_cs\_\textasciigrave{}n\textasciigrave{} QPushButton
where \sphinxtitleref{n} is 1\sphinxhyphen{}9 for G54 through G59.3


\begin{savenotes}\sphinxattablestart
\sphinxthistablewithglobalstyle
\centering
\sphinxcapstartof{table}
\sphinxthecaptionisattop
\sphinxcaption{Coordinate System Change Buttons}\label{\detokenize{coordinates:id2}}
\sphinxaftertopcaption
\begin{tabulary}{\linewidth}[t]{|T|T|T|}
\sphinxtoprule
\sphinxtableatstartofbodyhook
\sphinxAtStartPar
\sphinxstylestrong{Control Function}
&
\sphinxAtStartPar
\sphinxstylestrong{Object Type}
&
\sphinxAtStartPar
\sphinxstylestrong{Object Name}
\\
\sphinxhline
\sphinxAtStartPar
Change Coordinate System
&
\sphinxAtStartPar
QPushButton
&
\sphinxAtStartPar
change\_cs\_(n)
\\
\sphinxbottomrule
\end{tabulary}
\sphinxtableafterendhook\par
\sphinxattableend\end{savenotes}

\sphinxstepscope


\chapter{Miscellaneous Items}
\label{\detokenize{misc:miscellaneous-items}}\label{\detokenize{misc::doc}}

\section{File Selector}
\label{\detokenize{misc:file-selector}}
\sphinxAtStartPar
Add a QListWidget and name it \sphinxtitleref{file\_lw}, this can be used with a touch screen by
specifying the touch input. A single left\sphinxhyphen{}click or touch is all that’s needed to
use the \sphinxtitleref{File Selector}. A left\sphinxhyphen{}click or touch on a directory will change to
that directory. A left\sphinxhyphen{}click or touch on the up or down arrow will move the list
by one. A left\sphinxhyphen{}click or touch in between an arrow and the slider will move the
list by one page. Touch\sphinxhyphen{}and\sphinxhyphen{}hold to move the slider.

\sphinxAtStartPar
If you use the touch input, the selector looks like this

\noindent{\hspace*{\fill}\sphinxincludegraphics{{file-selector-01}.png}\hspace*{\fill}}

\begin{sphinxadmonition}{note}{Note:}
\sphinxAtStartPar
Make sure you use a QListWidget and not a QListView for the file
selector.
\end{sphinxadmonition}

\sphinxAtStartPar
\sphinxhref{https://youtu.be/kTFMM71VFuU}{File, Error and Information Tutorial}


\section{Code Viewer}
\label{\detokenize{misc:code-viewer}}
\sphinxAtStartPar
To add a code viewer, add a \sphinxtitleref{QPlainTextEdit} from Input Widgets and name it
\sphinxtitleref{gcode\_pte}

\noindent{\hspace*{\fill}\sphinxincludegraphics{{gcode-viewer-01}.png}\hspace*{\fill}}


\section{Code Viewer Controls}
\label{\detokenize{misc:code-viewer-controls}}
\sphinxAtStartPar
The Code Viewer allows you to edit the file in Flex GUI without using an external
text editor. You can save the current code to the current file name, save the
current code with a new file name and you can search the code.


\begin{savenotes}\sphinxattablestart
\sphinxthistablewithglobalstyle
\centering
\sphinxcapstartof{table}
\sphinxthecaptionisattop
\sphinxcaption{Code Viewer Controls}\label{\detokenize{misc:id1}}
\sphinxaftertopcaption
\begin{tabulary}{\linewidth}[t]{|T|T|T|}
\sphinxtoprule
\sphinxtableatstartofbodyhook
\sphinxAtStartPar
\sphinxstylestrong{Function}
&
\sphinxAtStartPar
\sphinxstylestrong{Type}
&
\sphinxAtStartPar
\sphinxstylestrong{Object Name}
\\
\sphinxhline
\sphinxAtStartPar
Save
&
\sphinxAtStartPar
QPushButton
&
\sphinxAtStartPar
save\_pb
\\
\sphinxhline
\sphinxAtStartPar
Save As
&
\sphinxAtStartPar
QPushButton
&
\sphinxAtStartPar
save\_as\_pb
\\
\sphinxhline
\sphinxAtStartPar
Search
&
\sphinxAtStartPar
QPushButton
&
\sphinxAtStartPar
search\_pb
\\
\sphinxbottomrule
\end{tabulary}
\sphinxtableafterendhook\par
\sphinxattableend\end{savenotes}


\section{MDI Viewer}
\label{\detokenize{misc:mdi-viewer}}
\sphinxAtStartPar
To add a MDI viewer, add a \sphinxtitleref{QListWidget} from Item Widgets and name it
\sphinxtitleref{mdi\_history\_lw}

\noindent{\hspace*{\fill}\sphinxincludegraphics{{mdi-viewer-01}.png}\hspace*{\fill}}

\sphinxAtStartPar
To enter MDI commands, add a Line Edit and name it \sphinxtitleref{mdi\_command\_le}.


\section{Error Viewer}
\label{\detokenize{misc:error-viewer}}
\sphinxAtStartPar
To add an error viewer, add a \sphinxtitleref{QPlainTextEdit} from Input Widgets and name it
\sphinxtitleref{errors\_pte}

\noindent{\hspace*{\fill}\sphinxincludegraphics{{error-viewer-01}.png}\hspace*{\fill}}

\sphinxAtStartPar
To clear the error history, add a QPushButton and set the objectName to
\sphinxtitleref{clear\_errors\_pb}.

\sphinxAtStartPar
To copy the errors to the clipboard, add a QPushButton and set the object name
to \sphinxtitleref{copy\_errors\_pb}.

\begin{sphinxadmonition}{warning}{Warning:}
\sphinxAtStartPar
The error viewer must be a QPlainTextEdit not a QTextEdit.
\end{sphinxadmonition}


\section{Information Viewer}
\label{\detokenize{misc:information-viewer}}
\sphinxAtStartPar
To add an information viewer, add a \sphinxtitleref{QPlainTextEdit} from Input Widgets and name
it \sphinxtitleref{info\_pte}. Information messages from MSG, DEBUG and PRINT will show up in
the Information Viewer if it exists.

\sphinxAtStartPar
If \sphinxtitleref{info\_pte} is not found and the \sphinxtitleref{errors\_pte} is found, then information
messages will show up in the Error Viewer.

\sphinxAtStartPar
To clear the information viewer, add a QPushButton and name it \sphinxtitleref{clear\_info\_pb}.

\begin{sphinxadmonition}{warning}{Warning:}
\sphinxAtStartPar
The information viewer must be a QPlainTextEdit not a QTextEdit.
\end{sphinxadmonition}


\section{Speed \& Feed Calculators}
\label{\detokenize{misc:speed-feed-calculators}}
\sphinxAtStartPar
To add a milling Speeds and Feeds Calculator, add a \sphinxtitleref{QFrame} or \sphinxtitleref{QWidget} and
set the Object Name to \sphinxtitleref{fsc\_container}

\noindent{\hspace*{\fill}\sphinxincludegraphics{{fsc-02}.png}\hspace*{\fill}}

\sphinxAtStartPar
To make the entry boxes touch\sphinxhyphen{}screen aware, add a Dynamic Property called
\sphinxtitleref{mode} and set the value to \sphinxtitleref{touch}. Then when you touch an entry field, a
numeric popup will show up to allow you to enter the value without a keyboard.
See {\hyperref[\detokenize{property::doc}]{\sphinxcrossref{\DUrole{doc}{Dynamic Properties}}}}

\noindent{\hspace*{\fill}\sphinxincludegraphics{{fsc-01}.png}\hspace*{\fill}}

\sphinxAtStartPar
To add a Drill Feed and Speed calculator, add a \sphinxtitleref{QFrame} or \sphinxtitleref{QWidget} and set
the Object Name to \sphinxtitleref{dsf\_container}.

\sphinxAtStartPar
To make the entry boxes touch\sphinxhyphen{}screen aware, add a Dynamic Property called
\sphinxtitleref{mode} and set the value to \sphinxtitleref{touch}. Then when you touch it, a numeric popup
will appear, allowing you to enter the numbers

\noindent{\hspace*{\fill}\sphinxincludegraphics{{dsc-01}.png}\hspace*{\fill}}


\section{Help System}
\label{\detokenize{misc:help-system}}
\sphinxAtStartPar
A QPushButton can be setup to launch a Help dialog which contains text from a
file in the configuration directory. A help button can be placed on multiple
places with different file names. Only one Help dialog can be open at a time.


\begin{savenotes}\sphinxattablestart
\sphinxthistablewithglobalstyle
\raggedright
\sphinxcapstartof{table}
\sphinxthecaptionisattop
\sphinxcaption{Help Button Dynamic Properties}\label{\detokenize{misc:id2}}
\sphinxaftertopcaption
\begin{tabulary}{\linewidth}[t]{|T|T|T|}
\sphinxtoprule
\sphinxtableatstartofbodyhook
\sphinxAtStartPar
\sphinxstylestrong{Property Name}
&
\sphinxAtStartPar
\sphinxstylestrong{Type}
&
\sphinxAtStartPar
\sphinxstylestrong{Value}
\\
\sphinxhline
\sphinxAtStartPar
function
&
\sphinxAtStartPar
string
&
\sphinxAtStartPar
help
\\
\sphinxhline
\sphinxAtStartPar
file
&
\sphinxAtStartPar
string
&
\sphinxAtStartPar
file name
\\
\sphinxhline
\sphinxAtStartPar
topic
&
\sphinxAtStartPar
string
&
\sphinxAtStartPar
title of topic
\\
\sphinxhline
\sphinxAtStartPar
x\_pos
&
\sphinxAtStartPar
string
&
\sphinxAtStartPar
x location of upper left corner
\\
\sphinxhline
\sphinxAtStartPar
y\_pos
&
\sphinxAtStartPar
string
&
\sphinxAtStartPar
y location of upper left corner
\\
\sphinxhline
\sphinxAtStartPar
horz\_size
&
\sphinxAtStartPar
string
&
\sphinxAtStartPar
width
\\
\sphinxhline
\sphinxAtStartPar
vert\_size
&
\sphinxAtStartPar
string
&
\sphinxAtStartPar
height
\\
\sphinxbottomrule
\end{tabulary}
\sphinxtableafterendhook\par
\sphinxattableend\end{savenotes}

\begin{sphinxadmonition}{note}{Note:}
\sphinxAtStartPar
The x\_pos is from the left edge of the screen and the y\_pos is from
the top of the screen.
\end{sphinxadmonition}

\sphinxAtStartPar
Dynamic Properties

\noindent{\hspace*{\fill}\sphinxincludegraphics{{help-01}.png}\hspace*{\fill}}

\sphinxAtStartPar
Help Dialog

\noindent{\hspace*{\fill}\sphinxincludegraphics{{help-02}.png}\hspace*{\fill}}

\sphinxstepscope


\chapter{HAL Pins}
\label{\detokenize{hal:hal-pins}}\label{\detokenize{hal::doc}}
\sphinxAtStartPar
\sphinxhref{https://youtu.be/LU4914GyGXI}{HAL Tutorial}

\begin{sphinxadmonition}{note}{Note:}
\sphinxAtStartPar
Dynamic Property names are case sensitive and must be all lower case.
Hal types and directions are case sensitive and must be all caps. The
function value must be lower case.
\end{sphinxadmonition}

\begin{sphinxadmonition}{note}{Note:}
\sphinxAtStartPar
Hal pin names can containe a\sphinxhyphen{}z, A\sphinxhyphen{}Z, 0\sphinxhyphen{}9, underscore \_, or dash \sphinxhyphen{}.
\end{sphinxadmonition}


\section{Button}
\label{\detokenize{hal:button}}
\sphinxAtStartPar
Any QPushButton, QCheckBox or QRadioButton can be assigned to a HAL \sphinxtitleref{bit} pin by
adding four string type Dynamic Properties.  The pin\_name used will create a HAL
pin prefixed with \sphinxtitleref{flexhal.} A pin\_name of my\sphinxhyphen{}button would be in HAL
\sphinxtitleref{flexhal.my\sphinxhyphen{}button}. See {\hyperref[\detokenize{property::doc}]{\sphinxcrossref{\DUrole{doc}{Dynamic Properties}}}}


\begin{savenotes}\sphinxattablestart
\sphinxthistablewithglobalstyle
\centering
\sphinxcapstartof{table}
\sphinxthecaptionisattop
\sphinxcaption{HAL Push Button}\label{\detokenize{hal:id1}}
\sphinxaftertopcaption
\begin{tabulary}{\linewidth}[t]{|T|T|}
\sphinxtoprule
\sphinxtableatstartofbodyhook
\sphinxAtStartPar
\sphinxstylestrong{Property Name}
&
\sphinxAtStartPar
\sphinxstylestrong{Pin Value}
\\
\sphinxhline
\sphinxAtStartPar
function
&
\sphinxAtStartPar
hal\_pin
\\
\sphinxhline
\sphinxAtStartPar
pin\_name
&
\sphinxAtStartPar
any unique name
\\
\sphinxhline
\sphinxAtStartPar
hal\_type
&
\sphinxAtStartPar
HAL\_BIT
\\
\sphinxhline
\sphinxAtStartPar
hal\_dir
&
\sphinxAtStartPar
HAL\_OUT
\\
\sphinxbottomrule
\end{tabulary}
\sphinxtableafterendhook\par
\sphinxattableend\end{savenotes}


\section{Spinbox}
\label{\detokenize{hal:spinbox}}\label{\detokenize{hal:spinboxtag}}
\sphinxAtStartPar
Any QSpinBox or QDoubleSpinBox can be a HAL \sphinxtitleref{number} pin by adding four string
type Dynamic Properties. The pin\_name used will create a HAL pin prefixed with
\sphinxtitleref{flexhal.} A pin\_name of my\sphinxhyphen{}spinbox would be in HAL \sphinxtitleref{flexhal.my\sphinxhyphen{}spinbox}.


\begin{savenotes}\sphinxattablestart
\sphinxthistablewithglobalstyle
\centering
\sphinxcapstartof{table}
\sphinxthecaptionisattop
\sphinxcaption{HAL Spin Box}\label{\detokenize{hal:id2}}
\sphinxaftertopcaption
\begin{tabulary}{\linewidth}[t]{|T|T|}
\sphinxtoprule
\sphinxtableatstartofbodyhook
\sphinxAtStartPar
\sphinxstylestrong{Property Name}
&
\sphinxAtStartPar
\sphinxstylestrong{Pin Value}
\\
\sphinxhline
\sphinxAtStartPar
function
&
\sphinxAtStartPar
hal\_pin
\\
\sphinxhline
\sphinxAtStartPar
pin\_name
&
\sphinxAtStartPar
any unique name
\\
\sphinxhline
\sphinxAtStartPar
hal\_type
&
\sphinxAtStartPar
HAL\_FLOAT or HAL\_S32 or HAL\_U32
\\
\sphinxhline
\sphinxAtStartPar
hal\_dir
&
\sphinxAtStartPar
HAL\_OUT
\\
\sphinxbottomrule
\end{tabulary}
\sphinxtableafterendhook\par
\sphinxattableend\end{savenotes}

\begin{sphinxadmonition}{note}{Note:}
\sphinxAtStartPar
A QSpinBox can only be HAL\_S32 or HAL\_U32 data type. A QDoubleSpinBox
can only be HAL\_FLOAT data type.
\end{sphinxadmonition}


\section{Slider}
\label{\detokenize{hal:slider}}
\sphinxAtStartPar
A QSlider can be a HAL pin by adding these four string type Dynamic Properties.
The pin\_name used will create a HAL pin prefixed with \sphinxtitleref{flexhal.} A pin\_name of
my\sphinxhyphen{}slider would be in HAL \sphinxtitleref{flexhal.my\sphinxhyphen{}slider}. See {\hyperref[\detokenize{property::doc}]{\sphinxcrossref{\DUrole{doc}{Dynamic Properties}}}}


\begin{savenotes}\sphinxattablestart
\sphinxthistablewithglobalstyle
\centering
\sphinxcapstartof{table}
\sphinxthecaptionisattop
\sphinxcaption{HAL Slider}\label{\detokenize{hal:id3}}
\sphinxaftertopcaption
\begin{tabulary}{\linewidth}[t]{|T|T|}
\sphinxtoprule
\sphinxtableatstartofbodyhook
\sphinxAtStartPar
\sphinxstylestrong{Property Name}
&
\sphinxAtStartPar
\sphinxstylestrong{Pin Value}
\\
\sphinxhline
\sphinxAtStartPar
function
&
\sphinxAtStartPar
hal\_pin
\\
\sphinxhline
\sphinxAtStartPar
pin\_name
&
\sphinxAtStartPar
any unique name
\\
\sphinxhline
\sphinxAtStartPar
hal\_type
&
\sphinxAtStartPar
HAL\_S32 or HAL\_U32
\\
\sphinxhline
\sphinxAtStartPar
hal\_dir
&
\sphinxAtStartPar
HAL\_OUT
\\
\sphinxbottomrule
\end{tabulary}
\sphinxtableafterendhook\par
\sphinxattableend\end{savenotes}


\section{HAL I/O}
\label{\detokenize{hal:hal-i-o}}
\sphinxAtStartPar
A HAL I/O pin can be input and output functions.

\sphinxAtStartPar
A QPushButton (set to checkable), QCheckBox, QDoubleSpinBox and QSpinBox can be
HAL I/O pin.

\sphinxAtStartPar
Two I/O pins connected to the same signal will stay in
synchronization. The connected pins must be of the same type.


\begin{savenotes}\sphinxattablestart
\sphinxthistablewithglobalstyle
\centering
\sphinxcapstartof{table}
\sphinxthecaptionisattop
\sphinxcaption{HAL I/O}\label{\detokenize{hal:id4}}
\sphinxaftertopcaption
\begin{tabulary}{\linewidth}[t]{|T|T|}
\sphinxtoprule
\sphinxtableatstartofbodyhook
\sphinxAtStartPar
\sphinxstylestrong{Property Name}
&
\sphinxAtStartPar
\sphinxstylestrong{Pin Value}
\\
\sphinxhline
\sphinxAtStartPar
function
&
\sphinxAtStartPar
hal\_io
\\
\sphinxhline
\sphinxAtStartPar
pin\_name
&
\sphinxAtStartPar
any unique name
\\
\sphinxhline
\sphinxAtStartPar
hal\_type
&
\sphinxAtStartPar
HAL\_BIT for a QCheckBox or QPushButton
\\
\sphinxhline
\sphinxAtStartPar
hal\_type
&
\sphinxAtStartPar
HAL\_FLOAT for a QDoubleSpinBox
\\
\sphinxhline
\sphinxAtStartPar
hal\_type
&
\sphinxAtStartPar
HAL\_S32 or HAL\_U32 for a QSpinBox
\\
\sphinxhline
\sphinxAtStartPar
hal\_dir
&
\sphinxAtStartPar
HAL\_IO
\\
\sphinxbottomrule
\end{tabulary}
\sphinxtableafterendhook\par
\sphinxattableend\end{savenotes}


\section{Label}
\label{\detokenize{hal:label}}
\sphinxAtStartPar
A QLabel can be used to monitor HAL pins. HAL connections must be made in the
post gui HAL file. The pin\_name used will create a HAL pin prefixed with
\sphinxtitleref{flexhal.} A pin\_name of my\sphinxhyphen{}reader would be in HAL \sphinxtitleref{flexhal.my\sphinxhyphen{}reader}.


\begin{savenotes}\sphinxattablestart
\sphinxthistablewithglobalstyle
\centering
\sphinxcapstartof{table}
\sphinxthecaptionisattop
\sphinxcaption{HAL Label}\label{\detokenize{hal:id5}}
\sphinxaftertopcaption
\begin{tabulary}{\linewidth}[t]{|T|T|}
\sphinxtoprule
\sphinxtableatstartofbodyhook
\sphinxAtStartPar
\sphinxstylestrong{Property Name}
&
\sphinxAtStartPar
\sphinxstylestrong{Pin Value}
\\
\sphinxhline
\sphinxAtStartPar
function
&
\sphinxAtStartPar
hal\_pin
\\
\sphinxhline
\sphinxAtStartPar
pin\_name
&
\sphinxAtStartPar
any unique name
\\
\sphinxhline
\sphinxAtStartPar
hal\_type
&
\sphinxAtStartPar
HAL\_BIT or HAL\_FLOAT or HAL\_S32 or HAL\_U32
\\
\sphinxhline
\sphinxAtStartPar
hal\_dir
&
\sphinxAtStartPar
HAL\_IN
\\
\sphinxbottomrule
\end{tabulary}
\sphinxtableafterendhook\par
\sphinxattableend\end{savenotes}

\begin{sphinxadmonition}{note}{Note:}
\sphinxAtStartPar
A HAL\_FLOAT QLabel can have a string Dynamic Property called
\sphinxtitleref{precision} with a value of the number of decimal digits.
\end{sphinxadmonition}


\section{Bool Label}
\label{\detokenize{hal:bool-label}}
\sphinxAtStartPar
A QLabel of hal\_type HAL\_BIT can have True and False text by adding two
additional Dynamic Properties. See {\hyperref[\detokenize{property::doc}]{\sphinxcrossref{\DUrole{doc}{Dynamic Properties}}}}


\begin{savenotes}\sphinxattablestart
\sphinxthistablewithglobalstyle
\centering
\sphinxcapstartof{table}
\sphinxthecaptionisattop
\sphinxcaption{HAL Bool Label}\label{\detokenize{hal:id6}}
\sphinxaftertopcaption
\begin{tabulary}{\linewidth}[t]{|T|T|}
\sphinxtoprule
\sphinxtableatstartofbodyhook
\sphinxAtStartPar
\sphinxstylestrong{Property Name}
&
\sphinxAtStartPar
\sphinxstylestrong{Pin Value}
\\
\sphinxhline
\sphinxAtStartPar
function
&
\sphinxAtStartPar
hal\_pin
\\
\sphinxhline
\sphinxAtStartPar
pin\_name
&
\sphinxAtStartPar
any unique name
\\
\sphinxhline
\sphinxAtStartPar
hal\_type
&
\sphinxAtStartPar
HAL\_BIT
\\
\sphinxhline
\sphinxAtStartPar
hal\_dir
&
\sphinxAtStartPar
HAL\_IN
\\
\sphinxhline
\sphinxAtStartPar
true\_text
&
\sphinxAtStartPar
text to display when True
\\
\sphinxhline
\sphinxAtStartPar
false\_text
&
\sphinxAtStartPar
text to display when False
\\
\sphinxbottomrule
\end{tabulary}
\sphinxtableafterendhook\par
\sphinxattableend\end{savenotes}

\noindent{\hspace*{\fill}\sphinxincludegraphics{{hal-bool-label-01}.png}\hspace*{\fill}}


\section{Multi\sphinxhyphen{}State Label}
\label{\detokenize{hal:multi-state-label}}
\sphinxAtStartPar
A QLabel of hal\_type HAL\_U32 can have multiple text by adding as many Dynamic
Properties as needed. The \sphinxtitleref{text\_n} starts at 0 for example text\_0, text\_1 etc.


\begin{savenotes}\sphinxattablestart
\sphinxthistablewithglobalstyle
\centering
\sphinxcapstartof{table}
\sphinxthecaptionisattop
\sphinxcaption{HAL Multi\sphinxhyphen{}State Label}\label{\detokenize{hal:id7}}
\sphinxaftertopcaption
\begin{tabulary}{\linewidth}[t]{|T|T|}
\sphinxtoprule
\sphinxtableatstartofbodyhook
\sphinxAtStartPar
\sphinxstylestrong{Property Name}
&
\sphinxAtStartPar
\sphinxstylestrong{Pin Value}
\\
\sphinxhline
\sphinxAtStartPar
function
&
\sphinxAtStartPar
hal\_msl
\\
\sphinxhline
\sphinxAtStartPar
pin\_name
&
\sphinxAtStartPar
any unique name
\\
\sphinxhline
\sphinxAtStartPar
hal\_type
&
\sphinxAtStartPar
HAL\_U32
\\
\sphinxhline
\sphinxAtStartPar
hal\_dir
&
\sphinxAtStartPar
HAL\_IN
\\
\sphinxhline
\sphinxAtStartPar
text\_n
&
\sphinxAtStartPar
text to display when value is equal to n
\\
\sphinxbottomrule
\end{tabulary}
\sphinxtableafterendhook\par
\sphinxattableend\end{savenotes}

\begin{sphinxadmonition}{note}{Note:}
\sphinxAtStartPar
The text values must start at 0 and be sequencial.
\end{sphinxadmonition}

\noindent{\hspace*{\fill}\sphinxincludegraphics{{hal-msl}.png}\hspace*{\fill}}


\section{LCD}
\label{\detokenize{hal:lcd}}
\sphinxAtStartPar
A QLCDNumber can be used to monitor HAL pins. HAL connections must be made in
the post gui HAL file. The pin\_name used will create a HAL pin prefixed with
\sphinxtitleref{flexhal.} A pin\_name of my\sphinxhyphen{}reader would be in HAL \sphinxtitleref{flexhal.my\sphinxhyphen{}reader}.


\begin{savenotes}\sphinxattablestart
\sphinxthistablewithglobalstyle
\centering
\sphinxcapstartof{table}
\sphinxthecaptionisattop
\sphinxcaption{HAL LCD}\label{\detokenize{hal:id8}}
\sphinxaftertopcaption
\begin{tabulary}{\linewidth}[t]{|T|T|}
\sphinxtoprule
\sphinxtableatstartofbodyhook
\sphinxAtStartPar
\sphinxstylestrong{Property Name}
&
\sphinxAtStartPar
\sphinxstylestrong{Pin Value}
\\
\sphinxhline
\sphinxAtStartPar
function
&
\sphinxAtStartPar
hal\_pin
\\
\sphinxhline
\sphinxAtStartPar
pin\_name
&
\sphinxAtStartPar
any unique name
\\
\sphinxhline
\sphinxAtStartPar
hal\_type
&
\sphinxAtStartPar
HAL\_FLOAT or HAL\_S32 or HAL\_U32
\\
\sphinxhline
\sphinxAtStartPar
hal\_dir
&
\sphinxAtStartPar
HAL\_IN
\\
\sphinxbottomrule
\end{tabulary}
\sphinxtableafterendhook\par
\sphinxattableend\end{savenotes}

\begin{sphinxadmonition}{note}{Note:}
\sphinxAtStartPar
A HAL\_FLOAT QLCDNumber can have a string Dynamic Property called
\sphinxtitleref{precision} with a value of the number of decimal digits.
\end{sphinxadmonition}

\sphinxAtStartPar
Pin Types:

\begin{sphinxVerbatim}[commandchars=\\\{\}]
\PYG{n}{HAL\PYGZus{}BIT}
\PYG{n}{HAL\PYGZus{}FLOAT}
\PYG{n}{HAL\PYGZus{}S32}
\PYG{n}{HAL\PYGZus{}U32}
\end{sphinxVerbatim}

\sphinxAtStartPar
Pin Directions:

\begin{sphinxVerbatim}[commandchars=\\\{\}]
\PYG{n}{HAL\PYGZus{}IN}
\PYG{n}{HAL\PYGZus{}OUT}
\PYG{n}{HAL\PYGZus{}IO}
\end{sphinxVerbatim}

\sphinxAtStartPar
Currently only \sphinxtitleref{HAL\_BIT} with \sphinxtitleref{HAL\_OUT} have been tested.

\begin{sphinxadmonition}{warning}{Warning:}
\sphinxAtStartPar
By default, no QRadioButtons are checked unless you set one checked
in the Designer. Starting up with none checked could be a problem if you
expect one to be selected at startup.
\end{sphinxadmonition}


\section{Progress Bar}
\label{\detokenize{hal:progress-bar}}
\sphinxAtStartPar
A QProgressBar can be used to monitor HAL pins. HAL connections must be made in
the post gui HAL file. The pin\_name used will create a HAL pin prefixed with
\sphinxtitleref{flexhal.} A pin\_name of my\sphinxhyphen{}bar would be in HAL \sphinxtitleref{flexhal.my\sphinxhyphen{}bar}.


\begin{savenotes}\sphinxattablestart
\sphinxthistablewithglobalstyle
\centering
\sphinxcapstartof{table}
\sphinxthecaptionisattop
\sphinxcaption{HAL Progressbar}\label{\detokenize{hal:id9}}
\sphinxaftertopcaption
\begin{tabulary}{\linewidth}[t]{|T|T|}
\sphinxtoprule
\sphinxtableatstartofbodyhook
\sphinxAtStartPar
\sphinxstylestrong{Property Name}
&
\sphinxAtStartPar
\sphinxstylestrong{Pin Value}
\\
\sphinxhline
\sphinxAtStartPar
function
&
\sphinxAtStartPar
hal\_pin
\\
\sphinxhline
\sphinxAtStartPar
pin\_name
&
\sphinxAtStartPar
any unique name
\\
\sphinxhline
\sphinxAtStartPar
hal\_type
&
\sphinxAtStartPar
HAL\_S32 or HAL\_U32
\\
\sphinxhline
\sphinxAtStartPar
hal\_dir
&
\sphinxAtStartPar
HAL\_IN
\\
\sphinxbottomrule
\end{tabulary}
\sphinxtableafterendhook\par
\sphinxattableend\end{savenotes}


\section{Step by Step}
\label{\detokenize{hal:step-by-step}}
\begin{sphinxadmonition}{note}{Note:}
\sphinxAtStartPar
This example is for a QPushButton
\end{sphinxadmonition}

\sphinxAtStartPar
You can use a QPushButton as a momentary output, or with \sphinxtitleref{checkable} selected
for a toggle type output, or QCheckBox or QRadioButton for a HAL output control.

\sphinxAtStartPar
Drag the widget into the GUI and the widget can have any name you like; names
are not used by HAL controls in Flex GUI \sphinxhyphen{} it is the following that matters.

\sphinxAtStartPar
Click on the widget to select it then click on the green plus sign in the
Property Editor for that widget to add a Dynamic Property and select String.
See {\hyperref[\detokenize{property::doc}]{\sphinxcrossref{\DUrole{doc}{Dynamic Properties}}}}

\noindent{\hspace*{\fill}\sphinxincludegraphics{{hal-01}.png}\hspace*{\fill}}

\sphinxAtStartPar
Set the Property Name to \sphinxtitleref{function} and click Ok

\noindent{\hspace*{\fill}\sphinxincludegraphics{{hal-02}.png}\hspace*{\fill}}

\sphinxAtStartPar
Set the Value to \sphinxtitleref{hal\_pin}; this tells Flex GUI that this widget is going to be
for a HAL pin

\noindent{\hspace*{\fill}\sphinxincludegraphics{{hal-03}.png}\hspace*{\fill}}

\sphinxAtStartPar
Add another string Dynamic Property named \sphinxtitleref{pin\_name} and set the value to any
unique name

\noindent{\hspace*{\fill}\sphinxincludegraphics{{hal-04}.png}\hspace*{\fill}}

\sphinxAtStartPar
Add another Dynamic Property named \sphinxtitleref{hal\_type} and set the value to HAL\_BIT

\noindent{\hspace*{\fill}\sphinxincludegraphics{{hal-05}.png}\hspace*{\fill}}

\sphinxAtStartPar
Add another Dynamic Property named \sphinxtitleref{hal\_dir} and set the value to HAL\_OUT

\noindent{\hspace*{\fill}\sphinxincludegraphics{{hal-06}.png}\hspace*{\fill}}

\sphinxAtStartPar
If you added Show HAL to your menu, you can open up the \sphinxtitleref{Halshow} program and
view the pin names

\noindent{\hspace*{\fill}\sphinxincludegraphics{{hal-07}.png}\hspace*{\fill}}

\sphinxAtStartPar
The pin names will all start with \sphinxtitleref{flexhal} plus the unique name you gave them

\noindent{\hspace*{\fill}\sphinxincludegraphics{{hal-08}.png}\hspace*{\fill}}

\sphinxAtStartPar
Now you can connect the Flex HAL pin in the postgui.hal file like normal

\begin{sphinxVerbatim}[commandchars=\\\{\}]
\PYG{n}{net} \PYG{n}{some}\PYG{o}{\PYGZhy{}}\PYG{n}{signal}\PYG{o}{\PYGZhy{}}\PYG{n}{name} \PYG{n}{flexhal}\PYG{o}{.}\PYG{n}{hal}\PYG{o}{\PYGZhy{}}\PYG{n}{test}\PYG{o}{\PYGZhy{}}\PYG{l+m+mi}{01} \PYG{o}{=}\PYG{o}{\PYGZgt{}} \PYG{n}{some}\PYG{o}{\PYGZhy{}}\PYG{n}{other}\PYG{o}{\PYGZhy{}}\PYG{n}{pin}\PYG{o}{\PYGZhy{}}\PYG{o+ow}{in}
\end{sphinxVerbatim}

\sphinxAtStartPar
After installing Flex GUI, from the CNC menu, you can copy the Flex GUI examples
and look at the hal\sphinxhyphen{}btn example.


\section{Homed Required}
\label{\detokenize{hal:homed-required}}
\sphinxAtStartPar
If the HAL button requires all joints to be homed before being enabled, you can
specify that by adding a Dynamic Property named \sphinxtitleref{required} and set the value to
\sphinxtitleref{homed}.

\noindent{\hspace*{\fill}\sphinxincludegraphics{{hal-09}.png}\hspace*{\fill}}

\sphinxstepscope


\chapter{Touch Screens}
\label{\detokenize{touch:touch-screens}}\label{\detokenize{touch::doc}}
\sphinxAtStartPar
Some entry widgets like MDI and Touch\sphinxhyphen{}Off have a touch\sphinxhyphen{}screen popup available
to make it easier for those users to enter the data.


\section{Tool Bar}
\label{\detokenize{touch:tool-bar}}
\sphinxAtStartPar
To add a button to the tool bar without having a menu item that creates an
action, you just have to create the action yourself

\noindent{\hspace*{\fill}\sphinxincludegraphics{{new-action-01}.png}\hspace*{\fill}}

\sphinxAtStartPar
The action creating window, when you type in the Text name, the Object name is
created for you

\noindent{\hspace*{\fill}\sphinxincludegraphics{{new-action-02}.png}\hspace*{\fill}}

\begin{sphinxadmonition}{warning}{Warning:}
\sphinxAtStartPar
Make sure the Object name matches the Action Name created when
you create a menu item \sphinxhyphen{} see the {\hyperref[\detokenize{menu::doc}]{\sphinxcrossref{\DUrole{doc}{Menu}}}} section for the full list of
Action Names.
\end{sphinxadmonition}

\sphinxAtStartPar
Now you just drag the action into the tool bar to create a new tool bar button

\noindent{\hspace*{\fill}\sphinxincludegraphics{{new-action-03}.png}\hspace*{\fill}}

\sphinxAtStartPar
Another option is to just use QPushButtons in a QFrame, as every menu action
has a QPushButton as well that executes the same function.


\section{MDI}
\label{\detokenize{touch:mdi}}
\sphinxAtStartPar
To enable the popup entry dialogs for the MDI entry, the QLineEdit object name
must be \sphinxtitleref{mdi\_command\_le} and the Dynamic Property \sphinxtitleref{input} must be \sphinxtitleref{nccode} for
the NC codes popup or \sphinxtitleref{keyboard} for a full keyboard popup.

\sphinxAtStartPar
The G codes dialog will appear when you touch the MDI entry box

\noindent{\hspace*{\fill}\sphinxincludegraphics{{touch-01}.png}\hspace*{\fill}}

\sphinxAtStartPar
The arrow buttons change the letters section to different letters

\noindent{\hspace*{\fill}\sphinxincludegraphics{{touch-02}.png}\hspace*{\fill}}

\sphinxAtStartPar
The full keyboard

\noindent{\hspace*{\fill}\sphinxincludegraphics{{touch-03}.png}\hspace*{\fill}}


\section{Touch Off}
\label{\detokenize{touch:touch-off}}
\sphinxAtStartPar
The Coordinate System Touch\sphinxhyphen{}Off offset is a QLineEdit named \sphinxtitleref{touchoff\_le}.
To enable the number pad popup for the offset entry, add a Dynamic Property
named \sphinxtitleref{input} and set the value to \sphinxtitleref{number}

\noindent{\hspace*{\fill}\sphinxincludegraphics{{touch-04}.png}\hspace*{\fill}}

\sphinxAtStartPar
Touch\sphinxhyphen{}Off:

\noindent{\hspace*{\fill}\sphinxincludegraphics{{touch-05}.png}\hspace*{\fill}}


\section{Tool Touch\sphinxhyphen{}Off}
\label{\detokenize{touch:tool-touch-off}}
\sphinxAtStartPar
The Tool Touch\sphinxhyphen{}Off offset is a QLineEdit named \sphinxtitleref{tool\_touchoff\_le}. To enable
the number pad popup for the offset entry, add a Dynamic Property named
\sphinxtitleref{input} and set the value to \sphinxtitleref{number}.


\section{Spin Boxes}
\label{\detokenize{touch:spin-boxes}}
\sphinxAtStartPar
QDoubleSpinBox and QSpinBox can use the popup numbers keypad by adding a
Dynamic Property named \sphinxtitleref{input} and setting the value to \sphinxtitleref{number}. If you enter
a float value for a QSpinBox the value will get converted to an integer


\section{Line Edits}
\label{\detokenize{touch:line-edits}}
\sphinxAtStartPar
A QLineEdit can have a popup entry for numbers, G codes, or a full keyboard.
Add a Dynamic Property named \sphinxtitleref{input} and set the value to one of these
\sphinxtitleref{number}, \sphinxtitleref{nccode}, or \sphinxtitleref{keyboard}.


\section{File Navigator}
\label{\detokenize{touch:file-navigator}}
\sphinxAtStartPar
If a QListWidget with an objectName of \sphinxtitleref{file\_lw} is found, a touch\sphinxhyphen{}friendly
file selector is added. A Parent Directory and possibly a directory name with
an ellipsis can be used to change directories. Touch a file name and it is
loaded into the GUI.

\sphinxAtStartPar
If PROGRAM\_PREFIX is specified, that will be the starting directory:

\noindent{\hspace*{\fill}\sphinxincludegraphics{{touch-06}.png}\hspace*{\fill}}

\sphinxstepscope


\chapter{Master Layout}
\label{\detokenize{layout:master-layout}}\label{\detokenize{layout::doc}}
\sphinxAtStartPar
Starting with an empty Main Window, if you right\sphinxhyphen{}click in it you can add a tool
bar or remove the status bar

\noindent{\hspace*{\fill}\sphinxincludegraphics{{layout-01}.png}\hspace*{\fill}}

\sphinxAtStartPar
Next, if you have some items you want visible all the time you can add a QFrame
or QWidget then below that, add a QTabWidget

\noindent{\hspace*{\fill}\sphinxincludegraphics{{layout-02}.png}\hspace*{\fill}}

\sphinxAtStartPar
Now that you have at least one widget in the main window, you can right\sphinxhyphen{}click
and select the layout you want to use

\noindent{\hspace*{\fill}\sphinxincludegraphics{{layout-03}.png}\hspace*{\fill}}

\sphinxAtStartPar
Example layout:

\noindent{\hspace*{\fill}\sphinxincludegraphics{{layout-04}.png}\hspace*{\fill}}

\sphinxAtStartPar
To add more tabs to a tab widget, right click on the tab then select Insert
Page and where you want it to be inserted

\noindent{\hspace*{\fill}\sphinxincludegraphics{{layout-05}.png}\hspace*{\fill}}

\sphinxAtStartPar
To change the tab name, in the Property Editor QTabWidget section, change
the currenTabText value to the new desired name.

\noindent{\hspace*{\fill}\sphinxincludegraphics{{layout-06}.png}\hspace*{\fill}}

\sphinxstepscope


\chapter{GUI Tips}
\label{\detokenize{tips:gui-tips}}\label{\detokenize{tips::doc}}
\sphinxAtStartPar
To group items together, use a container like a QFrame, QGroupBox or a
QTabWidget.

\sphinxAtStartPar
If the contents of a container are rows and columns after adding at least one
widget, right click and select \sphinxtitleref{Layout} then \sphinxtitleref{Lay Out in a Grid}. Now you can
drag and drop widgets into the container. The blue line or red box indicate
where it will be placed in the grid.

\noindent{\hspace*{\fill}\sphinxincludegraphics{{tips-01}.png}\hspace*{\fill}}

\sphinxAtStartPar
When using a grid layout for items that may change like Dro labels, change the
text to represent the longest number including a minus sign. Next, in the
property editor look at the Width and select (I usually use the column title) a
widget and set the minimum width a tad bigger than the widest widget in that
column. This will prevent the column from resizing as the values change.

\sphinxAtStartPar
For example, the numbers in the Actual column can contain up to 8 characters
like \sphinxhyphen{}23.4567. In the next image no minimum width has been set

\noindent{\hspace*{\fill}\sphinxincludegraphics{{tips-02}.png}\hspace*{\fill}}

\sphinxAtStartPar
All the cells in the column will have the same width \sphinxhyphen{} here you can see it has a
width of 44

\noindent{\hspace*{\fill}\sphinxincludegraphics{{tips-03}.png}\hspace*{\fill}}

\sphinxAtStartPar
If we double\sphinxhyphen{}click in the label and add \sphinxhyphen{}23.4567 the width changes to 61

\noindent{\hspace*{\fill}\sphinxincludegraphics{{tips-04}.png}\hspace*{\fill}}

\sphinxAtStartPar
I usually set the title of a column width to be a bit wider than the widest
widget in the column

\noindent{\hspace*{\fill}\sphinxincludegraphics{{tips-05}.png}\hspace*{\fill}}

\sphinxAtStartPar
If you drag a container into another container that has a layout and it’s real
short, just set the minimum height to make it larger and easier to drag and drop
into.

\sphinxAtStartPar
Ctrl + left click to select several widgets at once to change all their
properties.

\sphinxAtStartPar
The Monospace font is good for numbers that need a fixed width like DRO values.

\sphinxstepscope


\chapter{StyleSheet}
\label{\detokenize{style:stylesheet}}\label{\detokenize{style::doc}}
\sphinxAtStartPar
You can use your own .qss style sheet by creating a valid .qss file in the
configuration directory and setting it in the {\hyperref[\detokenize{ini::doc}]{\sphinxcrossref{\DUrole{doc}{INI Settings}}}}.

\begin{sphinxVerbatim}[commandchars=\\\{\}]
\PYG{p}{[}\PYG{n}{DISPLAY}\PYG{p}{]}
\PYG{n}{QSS} \PYG{o}{=} \PYG{n}{name\PYGZus{}of\PYGZus{}file}\PYG{o}{.}\PYG{n}{qss}
\end{sphinxVerbatim}

\begin{sphinxadmonition}{note}{Note:}
\sphinxAtStartPar
If a THEME is found in the ini file the QSS entry is ignored
\end{sphinxadmonition}

\sphinxAtStartPar
The Qt \sphinxhref{https://doc.qt.io/qt-6/stylesheet-reference.html}{Style Sheets Reference}
and the \sphinxhref{https://doc.qt.io/qt-6/stylesheet-syntax.html}{Style Sheet Syntax}
and the \sphinxhref{https://doc.qt.io/qt-6.2/stylesheet-examples.html}{Style Sheet Examples}
are good references to use when creating your own stylesheets.

\begin{sphinxadmonition}{note}{Note:}
\sphinxAtStartPar
If there is an error in the stylesheet syntax, no warning is issued,
it is just ignored. So don’t forget the ; at the end of each setting. And do
not accidentally use any backslashes it will break the whole file.
\end{sphinxadmonition}

\begin{sphinxadmonition}{warning}{Warning:}
\sphinxAtStartPar
If you only set a background\sphinxhyphen{}color on a QPushButton, the background
may not appear unless you set the border property to some value, even if
border is set to none.
\end{sphinxadmonition}


\section{Colors}
\label{\detokenize{style:colors}}
\sphinxAtStartPar
Most colors can be specified using Hex, RGB or RGBA color model. RGB is
Red, Green, Blue and A means Alpha or transparency. The alpha parameter is a
number between 0.0 (fully transparent) and 1.0 (not transparent at all). Hex is
red, green blue in hexadecimal number pairs from 00 to ff.

\begin{sphinxVerbatim}[commandchars=\\\{\}]
\PYG{p}{\PYGZsh{}}\PYG{n+nn}{0000ff}
\PYG{n+nt}{rgb}\PYG{o}{(}\PYG{n+nt}{0}\PYG{o}{,}\PYG{+w}{ }\PYG{n+nt}{0}\PYG{o}{,}\PYG{+w}{ }\PYG{n+nt}{255}\PYG{o}{)}\PYG{+w}{ }\PYG{n+nt}{Blue}
\PYG{n+nt}{rgba}\PYG{o}{(}\PYG{n+nt}{0}\PYG{o}{,}\PYG{+w}{ }\PYG{n+nt}{0}\PYG{o}{,}\PYG{+w}{ }\PYG{n+nt}{255}\PYG{o}{,}\PYG{+w}{ }\PYG{n+nt}{25}\PYG{o}{\PYGZpc{}}\PYG{o}{)}\PYG{+w}{ }\PYG{n+nt}{Light}\PYG{+w}{ }\PYG{n+nt}{Blue}
\end{sphinxVerbatim}


\section{Examples}
\label{\detokenize{style:examples}}
\begin{sphinxVerbatim}[commandchars=\\\{\}]
\PYG{c}{/* Set the background color for all QPushButtons, border is required * /}
\PYG{c}{QPushButton \PYGZob{}}
\PYG{c}{        background\PYGZhy{}color: rgba(224, 224, 224, 50\PYGZpc{});}
\PYG{c}{        border: 1px;}
\PYG{c}{\PYGZcb{}}

\PYG{c}{/* Set the background color and style for all QPushButtons when Pressed * /}
\PYG{c}{QPushButton:pressed \PYGZob{}}
\PYG{c}{        background\PYGZhy{}color: rgba(192, 192, 192, 100\PYGZpc{});}
\PYG{c}{        border\PYGZhy{}style: inset;}
\PYG{c}{\PYGZcb{}}

\PYG{c}{/* Set settings for a QPushButton named exit\PYGZus{}pb * /}
\PYG{c}{QPushButton\PYGZsh{}exit\PYGZus{}pb \PYGZob{}}
\PYG{c}{        border: none;}
\PYG{c}{        background\PYGZhy{}color: rgba(0, 0, 0, 0);}
\PYG{c}{\PYGZcb{}}

\PYG{c}{/* Using sub controls * /}
\PYG{c}{QAbstractSpinBox::up\PYGZhy{}button \PYGZob{}}
\PYG{c}{        min\PYGZhy{}width: 30px;}
\PYG{c}{\PYGZcb{}}

\PYG{c}{/* Combining sub controls and state * /}
\PYG{c}{QTabBar::tab:selected \PYGZob{}}
\PYG{c}{        background: lightgray;}
\PYG{c}{\PYGZcb{}}

\PYG{c}{/* Target by Object Name starts with something common*/}
\PYG{n+nt}{QLabel}\PYG{o}{[}\PYG{n+nt}{objectName}\PYG{o}{*}\PYG{o}{=}\PYG{l+s+s2}{\PYGZdq{}dro\PYGZdq{}}\PYG{o}{]}\PYG{+w}{ }\PYG{p}{\PYGZob{}}
\PYG{+w}{        }\PYG{k}{font\PYGZhy{}family}\PYG{p}{:}\PYG{+w}{ }\PYG{n}{Courier}\PYG{p}{;}
\PYG{+w}{        }\PYG{k}{font\PYGZhy{}size}\PYG{p}{:}\PYG{+w}{ }\PYG{l+m+mi}{14}\PYG{k+kt}{pt}\PYG{p}{;}
\PYG{+w}{        }\PYG{k}{font\PYGZhy{}weight}\PYG{p}{:}\PYG{+w}{ }\PYG{l+m+mi}{700}\PYG{p}{;}
\PYG{p}{\PYGZcb{}}
\end{sphinxVerbatim}


\section{Tool Bar Buttons}
\label{\detokenize{style:tool-bar-buttons}}
\sphinxAtStartPar
A tool bar button created from a menu action can be styled by using the
QToolButton\textasciigrave{} selector:

\begin{sphinxVerbatim}[commandchars=\\\{\}]
\PYG{n+nt}{QToolButton}\PYG{p}{:}\PYG{n+nd}{hover}\PYG{+w}{ }\PYG{p}{\PYGZob{}}
\PYG{+w}{        }\PYG{k}{background\PYGZhy{}color}\PYG{p}{:}\PYG{+w}{ }\PYG{n+nb}{rgba}\PYG{p}{(}\PYG{l+m+mi}{255}\PYG{p}{,}\PYG{+w}{ }\PYG{l+m+mi}{0}\PYG{p}{,}\PYG{+w}{ }\PYG{l+m+mi}{0}\PYG{p}{,}\PYG{+w}{ }\PYG{l+m+mi}{75}\PYG{k+kt}{\PYGZpc{}}\PYG{p}{)}\PYG{p}{;}
\PYG{p}{\PYGZcb{}}
\end{sphinxVerbatim}
\phantomsection\label{\detokenize{style:refname}}
\sphinxAtStartPar
To set the style of a single tool bar button, you need to use the widget name
for that action. The tool bar button must exist in the tool bar.


\begin{savenotes}\sphinxattablestart
\sphinxthistablewithglobalstyle
\raggedright
\sphinxcapstartof{table}
\sphinxthecaptionisattop
\sphinxcaption{Tool Button Names}\label{\detokenize{style:id1}}
\sphinxaftertopcaption
\begin{tabulary}{\linewidth}[t]{|T|T|T|}
\sphinxtoprule
\sphinxtableatstartofbodyhook
\sphinxAtStartPar
\sphinxstylestrong{Menu Item}
&
\sphinxAtStartPar
\sphinxstylestrong{Action Name}
&
\sphinxAtStartPar
\sphinxstylestrong{Widget Name}
\\
\sphinxhline
\sphinxAtStartPar
Open
&
\sphinxAtStartPar
actionOpen
&
\sphinxAtStartPar
flex\_Open
\\
\sphinxhline
\sphinxAtStartPar
Edit
&
\sphinxAtStartPar
actionEdit
&
\sphinxAtStartPar
flex\_Edit
\\
\sphinxhline
\sphinxAtStartPar
Reload
&
\sphinxAtStartPar
actionReload
&
\sphinxAtStartPar
flex\_Reload
\\
\sphinxhline
\sphinxAtStartPar
Save As
&
\sphinxAtStartPar
actionSave\_As
&
\sphinxAtStartPar
flex\_Save\_As
\\
\sphinxhline
\sphinxAtStartPar
Edit Tool Table
&
\sphinxAtStartPar
actionEdit\_Tool\_Table
&
\sphinxAtStartPar
flex\_Edit\_Tool\_Table
\\
\sphinxhline
\sphinxAtStartPar
Reload Tool Table
&
\sphinxAtStartPar
actionReload\_Tool\_Table
&
\sphinxAtStartPar
flex\_Reload\_Tool\_Table
\\
\sphinxhline
\sphinxAtStartPar
Ladder Editor
&
\sphinxAtStartPar
actionLadder\_Editor
&
\sphinxAtStartPar
flex\_Ladder\_Editor
\\
\sphinxhline
\sphinxAtStartPar
Quit
&
\sphinxAtStartPar
actionQuit
&
\sphinxAtStartPar
flex\_Quit
\\
\sphinxhline
\sphinxAtStartPar
E Stop
&
\sphinxAtStartPar
actionE\_Stop
&
\sphinxAtStartPar
flex\_E\_Stop
\\
\sphinxhline
\sphinxAtStartPar
Power
&
\sphinxAtStartPar
action\_Power
&
\sphinxAtStartPar
flex\_Power
\\
\sphinxhline
\sphinxAtStartPar
Run
&
\sphinxAtStartPar
actionRun
&
\sphinxAtStartPar
flex\_Run
\\
\sphinxhline
\sphinxAtStartPar
Run From Line
&
\sphinxAtStartPar
actionRun\_From\_Line
&
\sphinxAtStartPar
flex\_Run\_From\_Line
\\
\sphinxhline
\sphinxAtStartPar
Step
&
\sphinxAtStartPar
actionStep
&
\sphinxAtStartPar
flex\_Step
\\
\sphinxhline
\sphinxAtStartPar
Pause
&
\sphinxAtStartPar
actionPause
&
\sphinxAtStartPar
flex\_Pause
\\
\sphinxhline
\sphinxAtStartPar
Resume
&
\sphinxAtStartPar
actionResume
&
\sphinxAtStartPar
flex\_Resume
\\
\sphinxhline
\sphinxAtStartPar
Stop
&
\sphinxAtStartPar
actionStop
&
\sphinxAtStartPar
flex\_Stop
\\
\sphinxhline
\sphinxAtStartPar
Clear MDI History
&
\sphinxAtStartPar
actionClear\_MDI\_History
&
\sphinxAtStartPar
flex\_Clear\_MDI\_History
\\
\sphinxhline
\sphinxAtStartPar
Copy MDI History
&
\sphinxAtStartPar
actionCopy\_MDI\_History
&
\sphinxAtStartPar
flex\_Copy\_MDI\_History
\\
\sphinxhline
\sphinxAtStartPar
Show HAL
&
\sphinxAtStartPar
actionShow\_HAL
&
\sphinxAtStartPar
flex\_Show\_HAL
\\
\sphinxhline
\sphinxAtStartPar
HAL Meter
&
\sphinxAtStartPar
actionHAL\_Meter
&
\sphinxAtStartPar
flex\_HAL\_Meter
\\
\sphinxhline
\sphinxAtStartPar
HAL Scope
&
\sphinxAtStartPar
actionHAL\_Scope
&
\sphinxAtStartPar
flex\_HAL\_Scope
\\
\sphinxhline
\sphinxAtStartPar
About
&
\sphinxAtStartPar
actionAbout
&
\sphinxAtStartPar
flex\_About
\\
\sphinxhline
\sphinxAtStartPar
Quick Reference
&
\sphinxAtStartPar
actionQuick\_Reference
&
\sphinxAtStartPar
flex\_Quick\_Reference
\\
\sphinxbottomrule
\end{tabulary}
\sphinxtableafterendhook\par
\sphinxattableend\end{savenotes}

\sphinxAtStartPar
The syntax to select a tool bar button by name (here the flex\_Quit button) is:

\begin{sphinxVerbatim}[commandchars=\\\{\}]
\PYG{n+nt}{QToolButton}\PYG{p}{\PYGZsh{}}\PYG{n+nn}{flex\PYGZus{}Quit}\PYG{p}{:}\PYG{n+nd}{hover}\PYG{+w}{ }\PYG{p}{\PYGZob{}}
\PYG{+w}{        }\PYG{k}{background\PYGZhy{}color}\PYG{p}{:}\PYG{+w}{ }\PYG{n+nb}{rgba}\PYG{p}{(}\PYG{l+m+mi}{255}\PYG{p}{,}\PYG{+w}{ }\PYG{l+m+mi}{0}\PYG{p}{,}\PYG{+w}{ }\PYG{l+m+mi}{0}\PYG{p}{,}\PYG{+w}{ }\PYG{l+m+mi}{75}\PYG{k+kt}{\PYGZpc{}}\PYG{p}{)}\PYG{p}{;}
\PYG{p}{\PYGZcb{}}
\end{sphinxVerbatim}

\sphinxstepscope


\chapter{Resources}
\label{\detokenize{resources:resources}}\label{\detokenize{resources::doc}}
\sphinxAtStartPar
To create a resources.py file with images to use with the .qss stylesheet, start
by placing all the images in a different directory than the configuration files.
A subdirectory in the configuration directory is fine

\begin{sphinxVerbatim}[commandchars=\\\{\}]
└── configs
   └── my\PYGZus{}mill
       └── images
\end{sphinxVerbatim}

\sphinxAtStartPar
Add the following library if not installed

\begin{sphinxVerbatim}[commandchars=\\\{\}]
\PYG{n}{sudo} \PYG{n}{apt} \PYG{n}{install} \PYG{n}{qtbase5}\PYG{o}{\PYGZhy{}}\PYG{n}{dev}\PYG{o}{\PYGZhy{}}\PYG{n}{tools}
\end{sphinxVerbatim}

\sphinxAtStartPar
After installing Flex GUI on the CNC menu run \sphinxtitleref{Flex Resources}

\noindent{\hspace*{\fill}\sphinxincludegraphics{{resources-00}.png}\hspace*{\fill}}

\sphinxAtStartPar
Startup

\noindent{\hspace*{\fill}\sphinxincludegraphics{{resources-01}.png}\hspace*{\fill}}

\sphinxAtStartPar
Next Select Images Directory

\noindent{\hspace*{\fill}\sphinxincludegraphics{{resources-02}.png}\hspace*{\fill}}

\sphinxAtStartPar
The selected directory is shown in the label

\noindent{\hspace*{\fill}\sphinxincludegraphics{{resources-03}.png}\hspace*{\fill}}

\sphinxAtStartPar
Next Select Image Files. To select all the images left click on the first one
and hold down the shift key and left click on the last one. To pick several
images but not all hold down the ctrl key while you left click on each one.

\noindent{\hspace*{\fill}\sphinxincludegraphics{{resources-04}.png}\hspace*{\fill}}

\sphinxAtStartPar
The images selected are shown below

\noindent{\hspace*{\fill}\sphinxincludegraphics{{resources-05}.png}\hspace*{\fill}}

\sphinxAtStartPar
Next Build QRC File

\noindent{\hspace*{\fill}\sphinxincludegraphics{{resources-06}.png}\hspace*{\fill}}

\sphinxAtStartPar
Next Select Config Directory

\noindent{\hspace*{\fill}\sphinxincludegraphics{{resources-07}.png}\hspace*{\fill}}

\begin{sphinxadmonition}{note}{Note:}
\sphinxAtStartPar
The Image directory and the configuration directory must be different
\end{sphinxadmonition}

\sphinxAtStartPar
Next Build Resources File

\noindent{\hspace*{\fill}\sphinxincludegraphics{{resources-08}.png}\hspace*{\fill}}

\sphinxAtStartPar
The Flex Resource Builder can be closed now. In the configuration directory you
will have a resources.py file that contains the images used by the stylesheet.

\sphinxAtStartPar
Next edit the ini file and in the {[}DISPLAY{]} section add the following line

\begin{sphinxVerbatim}[commandchars=\\\{\}]
\PYG{n}{RESOURCES} \PYG{o}{=} \PYG{n}{resources}\PYG{o}{.}\PYG{n}{py}
\end{sphinxVerbatim}

\sphinxAtStartPar
In the {[}DISPLAY{]} section add the style sheet

\begin{sphinxVerbatim}[commandchars=\\\{\}]
\PYG{n}{QSS} \PYG{o}{=} \PYG{n}{xyz}\PYG{o}{.}\PYG{n}{qss}
\end{sphinxVerbatim}

\sphinxAtStartPar
To add an image named my\sphinxhyphen{}image.png to a QPushButton with an object name of
my\_pb add the following to the qss file

\begin{sphinxVerbatim}[commandchars=\\\{\}]
\PYG{n}{QPushButton}\PYG{c+c1}{\PYGZsh{}my\PYGZus{}pb \PYGZob{}}
        \PYG{n+nb}{min}\PYG{o}{\PYGZhy{}}\PYG{n}{height}\PYG{p}{:} \PYG{l+m+mi}{80}\PYG{n}{px}\PYG{p}{;}
        \PYG{n+nb}{min}\PYG{o}{\PYGZhy{}}\PYG{n}{width}\PYG{p}{:} \PYG{l+m+mi}{80}\PYG{n}{px}\PYG{p}{;}
        \PYG{n}{margin}\PYG{p}{:} \PYG{l+m+mi}{2}\PYG{n}{px}\PYG{p}{;}
        \PYG{n}{background}\PYG{o}{\PYGZhy{}}\PYG{n}{position}\PYG{p}{:} \PYG{n}{center}\PYG{p}{;}
        \PYG{n}{background}\PYG{o}{\PYGZhy{}}\PYG{n}{origin}\PYG{p}{:} \PYG{n}{content}\PYG{p}{;}
        \PYG{n}{background}\PYG{o}{\PYGZhy{}}\PYG{n}{clip}\PYG{p}{:} \PYG{n}{padding}\PYG{p}{;}
        \PYG{n}{background}\PYG{o}{\PYGZhy{}}\PYG{n}{repeat}\PYG{p}{:} \PYG{n}{no}\PYG{o}{\PYGZhy{}}\PYG{n}{repeat}\PYG{p}{;}
        \PYG{n}{background}\PYG{o}{\PYGZhy{}}\PYG{n}{image}\PYG{p}{:} \PYG{n}{url}\PYG{p}{(}\PYG{p}{:}\PYG{n}{my}\PYG{o}{\PYGZhy{}}\PYG{n}{image}\PYG{o}{.}\PYG{n}{png}\PYG{p}{)}\PYG{p}{;}
\PYG{p}{\PYGZcb{}}
\end{sphinxVerbatim}

\sphinxAtStartPar
Now when you run the configuration the image will be on the QPushButton

\noindent{\hspace*{\fill}\sphinxincludegraphics{{resources-09}.png}\hspace*{\fill}}

\begin{sphinxadmonition}{note}{Note:}
\sphinxAtStartPar
Delete any text in the QPushButton or it will be on top of the image
\end{sphinxadmonition}



\renewcommand{\indexname}{Index}
\printindex
\end{document}